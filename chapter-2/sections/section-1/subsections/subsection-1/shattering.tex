\subsection{Shattering}

We begin with a purely combinatorial setup. Fix an infinite set $X$ and a collection of its subsets, $\mathcal{C} \subseteq \mathcal{P}(X)$. For any finite set $F\subseteq X$, we are interested in the subsets of $F$ that can be \emph{cut out} by the sets in $\mathcal{C}$. This is called the \emph{trace} of $\mathcal{C}$ on $F$.
\[
    \mathcal{C}\cap F=\{C\cap F:C\in\mathcal{C}\}.
\]
Since $F$ is finite, it has $2^n$ subsets, ie $|\mathcal{P}(F)| = 2^n$. If all of the subsets of $F$ can be cut out by $\mathcal{C}$, we say \emph{$\mathcal{C}$ shatters $F$}. We define this more formally below.

\begin{definition}[Shattering]
    Let $\mathcal{C} \subseteq \mathcal{P}(X)$. We say that $\mathcal{C}$ \emph{shatters} a finite set $F \subseteq X$ if the trace of $\mathcal{C}$ on $F$ is the entire power set of $F$. That is, if $\mathcal{C} \cap F = \mathcal{P}(F)$, or equivalently, $|\mathcal{C} \cap F| = 2^{|F|}$.
\end{definition}

Note that, shattering is the key concept that underlying the combinatorial dichotomy we will introduce below. It's also important to emphasize that, generally speaking, we are not interested in $\mathcal{C}$ shattering any specific finite subset, rather we want to know given $n\in \mathbb{N}$, can $\mathcal{C}$ shatter some finite set of size $n$. If the answer is "no" for given $n$, then we want to know what is the maximum number of subsets $\mathcal{C}$ can cut out from a finite set of size $n$. To measure this \textit{combinatorial complexity} of $\mathcal{C}$, we define its \emph{growth function}, which measures the maximum size of a trace on any set of size $n$.
\begin{equation}\label{eq:growth-func-def-1}
f_{\mathcal{C}}(n)\coloneq\max\bigl\{|\mathcal{C}\cap F|:F\subseteq X,\ |F|=n\bigr\}.
\end{equation}
Note that, we always have $f_{\mathcal{C}}(n) \le 2^n$. Following theorem states that the combinatorial complexity of $\mathcal{C}$, which is determined by its growth function, can only be polynomial or exponential.

\begin{theorem}[Combinatorial Dichotomy]
    \label{thm:vc-dichotomy}
    Let $X$ and $\mathcal{C}$ be given as above. Then exactly one of the following holds:
    \begin{enumerate}[label=(\roman*)]
        \item $f_{\mathcal{C}}(n)=2^{n}$ for every $n\in\mathbb{N}$.
        \item There exists an integer $d\ge 1$ such that
        \[
            f_{\mathcal{C}}(n)\le n^d
        \]
        for all sufficiently large $n \in \mathbb{N}$.
    \end{enumerate}
\end{theorem}

\begin{remark}
    In the first case of~\ref{thm:vc-dichotomy}, we say that $\mathcal{C}$ has \emph{exponential growth}. In the second case, we say that $\mathcal{C}$ has \emph{polynomial growth} and we call $d$ the \emph{degree} of the polynomial bound. The dichotomy is a powerful tool in combinatorial geometry and model theory, as it allows us to classify families of sets based on their growth behavior.

    We will prove this theorem using the following lemma.
\end{remark}