\subsection{Dependence property - Dual perspective into finite VC-dimension}

When a collection $\mathcal{C}$ is naturally parametrised by the elements of another set—such as in Example~\ref{ex:vc-infinite-dimension}, where $\mathcal{C}$ was indexed by the natural numbers—its VC-dimension can often be analysed via a corresponding \textit{model-theoretic} property of the parameter space, namely \textit{(non-in)dependence} propery, shortened as \textit{(N)IP}. In particular, the concept of dependence in model theory serves as a dual perspective to finite VC-dimension. We will see that a parametrised family of sets is dependent precisely when it has finite VC-dimension.

To make this correspondence precise, we adopt the following formal framework. Let $X$ and $Y$ be infinite sets, and let $\Phi \subseteq X \times Y$ be a binary relation. For each $x \in X$ and $y \in Y$, define the sets
\[
    \Phi_x = \{y \in Y : (x,y) \in \Phi\} \subseteq Y \quad \text{and} \quad \Phi^y = \{x \in X : (x,y) \in \Phi\} \subseteq X.
\]

\noindent
Then we define the collections of sets indexed by $Y$ and $X$ correspondingly:
\[
    \mathcal{C}_X = \{\Phi^y : y \in Y\}  \quad \text{and} \quad \mathcal{C}_Y = \{\Phi_x : x \in X\}.
\]
\noindent
Here, $Y$ is the parameter space for $\mathcal{C}_X\subseteq \mathcal{P}(X)$, and $X$ is the parameter space for $\mathcal{C}_Y \subseteq \mathcal{P}(Y)$.

We focus on the trace of $\mathcal{C}_X$ on a finite set $F \subseteq X$. A subset $E \subseteq F$ belongs to the trace $\mathcal{C}_X \cap F$ if and only if there exists some parameter $y \in Y$ such that $E = \Phi_y \cap F$. This condition can be rephrased:
\[
    \begin{aligned}
        E = \Phi_y \cap F \quad &\Leftrightarrow \quad [\forall x \in E, (x,y) \in \Phi] \text{ and } [\forall x \in F \setminus E, (x,y) \notin \Phi] \\
        &\Leftrightarrow \quad [\forall x \in E, y \in \Phi_x] \text{ and } [\forall x \in F \setminus E, y \notin \Phi_x] \\
    \end{aligned}
\]
Therefore,
\begin{equation}
    \label{eq:trace-condition}
    E \in \mathcal{C}_X \cap F \quad \Leftrightarrow \quad \left( \bigcap_{x \in E} \Phi_x \right) \cap \left( \bigcap_{x \in F \setminus E} (Y \setminus \Phi_x) \right) \neq \emptyset. \quad
\end{equation}


Recall that, \emph{atom} is a minimal non-empty set in a Boolean algebra of sets~\cite[Chap 1]{vandenDries1998} in terms of inclusion. If we consider the boolean algebra generated by the sets $\{\Phi_x : x \in F\}$, denoted as $\mathcal{B}(\{\Phi_x : x \in F\})$, the expression on the right-hand side of~\ref{eq:trace-condition} is precisely an atom of this Boolean algebra for each subset $E \in \mathcal{P}(F)$.

Hence, given $\mathcal{C}_X$ and a finite subset $F \subseteq X$ with $|F| = n$, we can use this duality described above to determine if $\mathcal{C}_X$ shatters $F$. More specifically, this can be summarized as follows:
\[
    \begin{aligned}
        \mathcal{C}_X \text{ shatters } F \quad &\Leftrightarrow \quad |\mathcal{C}_X \cap F| = 2^n \\
        &\Leftrightarrow \quad \mathcal{B}(\{\Phi_x : x \in F\}) \text{ has } 2^n \text{ atoms} \\
    \end{aligned}
\]

Note that, here duality emerges from the fact that $\mathcal{C}_X$ is collection of subsets of $X$ indexed by $Y$, while $\{\Phi_x : x \in F\}$ is a collection of subsets of $Y$ indexed by the finite set $F\subset X$. The atoms of the Boolean algebra $\mathcal{B}(\{\Phi_x : x \in F\})$ correspond to the subsets of $F$ that can be isolated by the parameters from $Y$. We will not use this argument based on the atoms of boolean algebra to define a model-theoretic property of $\mathcal{C}_Y$ called \textit{independence}.

Firstly, we need to introduce some notation that will be useful when we use dependence property of a collection of subsets. Let \(Y\) be an arbitrary infinite set, and let \(S \subseteq Y\). We introduce the following notation for any subset \(S\) of \(Y\):
\[
    S^1 \coloneqq S,
    \quad
    S^{-1} \coloneqq Y \setminus S.
\]
\noindent
Next, let \(G = \{1,\dots,n\}\subseteq\mathbb{N}\). For each \(H\subseteq G\), define the sign‐function
\[
    \begin{equation}
        \label{eq:sign-function}
        \mathds{1}_H : G \to \{1,-1\},
        \qquad
        \mathds{1}_H(i) =
        \begin{cases}
            1,  & i\in H,\\
            -1, & i\notin H.
        \end{cases}
    \end{equation}
\]
\noindent
Given subsets \(S_1,\dots,S_n \subseteq Y\) and any \(H\in\mathcal P(G)\), consider the intersection
\[
    S_1^{\mathds{1}_H(1)}
    \,\cap
    S_2^{\mathds{1}_H(2)}
    \,\cap\cdots\cap\,
    S_n^{\mathds{1}_H(n)}.
\]

There are \(2^n\) such intersections; they are pairwise disjoint and their union equals \(Y\). The sequence \((S_1,\dots,S_n)\) exhibits dependence precisely if at least one of these intersections is empty. More precisely, we have the following definition.


\begin{definition}[Independence property]
    Let $Y$ and $S_1, \ldots, S_n \subseteq Y$ be given.
    \begin{itemize}
        \item We say that the sequence $(S_1,  \ldots, S_n)$ is \emph{independent} (in $Y$) if the boolean algebra generated by the sets $S_1, \ldots, S_n$, denoted $\mathcal{B}(S_1, \ldots, S_n)$ has $2^n$ atoms. Otherwise, we say that the sequence is \emph{dependent} (in $Y$).

        \item If a collection $\mathcal{C}_Y \subseteq \mathcal{P}(Y)$ contains independent sequence of size $n$ for every $n \in \mathbb{N}$, we say that $\mathcal{C}_Y$ is \emph{independent} (in $Y$). Otherwise, we say that $\mathcal{C}_Y$ is \emph{dependent} (in $Y$).
    \end{itemize}
\end{definition}

Following proposition is a dual of ~\ref{cor:vc-dichotomy}.

\begin{proposition}{\label{prop:number-of-atoms}}
    Let $\mathcal{C}_Y \subseteq \mathcal{P}(Y)$ be a collection of subsets of an infinite set $Y$. If $\mathcal{C}_Y$ contains no independent sequence of size $d \ge 1$ for some $d$, then for all sequences $S_1, \ldots, S_n \in \mathcal{C}_Y$, the boolean algebra $\mathcal{B}(S_1, \dots, S_n)$ has at most $p_d(n)$ atoms.
\end{proposition}

\begin{proof}

    For $n < d$, $p_d(n) = 2^n$, so $C_Y$ shatters some set of size $n$. Assume, $n \geq d$ and let $S_1, \dots, S_n \in C_Y$. Also, let $F = \{1, \ldots, n\}$ and consider $\mathcal{D} \subseteq \mathcal{P}(F)$ where $\mathcal{D}$ contains subsets $D \subseteq F$ such that
    \[
        S_1^{\mathds{1}_D(1)} \cap S_2^{\mathds{1}_D(2)} \cap \cdots \cap S_n^{\mathds{1}_D(n)} \neq \emptyset.
    \]

    If $|\mathcal{D}| > p_d(n)$, then by the lemma~\ref{lem:dependent-boolean-comb}, there exists a subset $E \subseteq F$ of size $d$ such that $\mathcal{D}$ shatters $E$. Hence, the boolean algebra $\mathcal{B}(S_1, \dots, S_n)$ has $2^d$ atoms, which contradicts the assumption that $\mathcal{C}_Y$ contains no independent sequence of size $d$. Therefore, we must have $|\mathcal{D}| \leq p_d(n)$.
\end{proof}

One of the sufficient conditions for a sequence to be dependent is as follows:
\begin{lemma}{\label{lem:suff-cond-dependencty}}
    Let $\mathcal{C}_Y \subseteq \mathcal{P}(Y)$ be a collection and let $d \in \mathbb{N}$ be such that any non-empty intersection of $d$ sets from $\mathcal{C}_Y$, say $S_1, \dots, S_d \in \mathcal{C}_Y$, is equal to the intersection of at most $d-1$ of those sets $\mathcal{C}_Y$
    \[
        S_1 \cap S_2 \cap \cdots \cap S_d = \bigcap_{i\in D} S_i \quad \text{for some } D \subseteq \{1, 2, \ldots, d\} \text{ with } |D| < d.
    \]
    Then the collection $\mathcal{C}_Y$ is dependent in $Y$.
\end{lemma}

\begin{proof}

    It's enough to show that there is no independent sequence of size $d$ in $\mathcal{C}_Y$. In other words, for an arbitrary sequence $S_1, \dots, S_d$, the number of atoms of the boolean algebra generated by this sequence, $\mathcal{B}(S_1, \dots, S_d)$ is strictly less than $2^d$.

    Let $S_1, \dots, S_d$ be an arbitrary sequence of sets in $\mathcal{C}_Y$. If $S_1 \cap \dots \cap S_d$ is empty then we are done, since this intersection cannot be atom of $\mathcal{B}(S_1, \dots, S_d)$ and this boolean algebra has strictly less than $2^d$ atoms. On the other hand, if $S_1 \cap \dots \cap S_d$ is non-empty, then by assumption
    \begin{equation}
        \label{eq:proof-suff-cont-1}
        S_1 \cap \dots \cap S_d = \bigcap_{i \in D} S_i
    \end{equation}
    for some proper subset $D \subset \{1, \dots, d\}$. It follows that the intersection
    \begin{equation}
        \label{eq:proof-suff-cont-2}
        \begin{aligned}
            \bigl(\bigcap_{i \in D} S_i\bigr) \cap \bigl(\bigcap_{j \not \in D} S^{-1}_j\bigr) =  S_1^{\mathds{1}_D(1)} \cap S_2^{\mathds{1}_D(2)} \cap \cdots \cap S_n^{\mathds{1}_D(n)}\\
        \end{aligned}
    \end{equation}
    is also empty combining~\ref{eq:proof-suff-cont-1} and~\ref{eq:proof-suff-cont-2}, where $\mathds{1}_{D}: \{1, \dots, d\} \to \{-1, 1\}$ is a sign function defined in~\ref{eq:sign-function}. To see this, note that combining~\ref{eq:proof-suff-cont-1} and~\ref{eq:proof-suff-cont-2} we get intersection of two sets that are complement of each other, more precisely,
    \[
        \begin{aligned}
            \bigl(\bigcap_{i \in D} S_i\bigr) \cap \bigl(\bigcap_{j \not \in D} S^{-1}_j\bigr) = S_1 \cap \dots \cap S_d \cap \bigl(\bigcap_{j \not \in D} S^{-1}_j\bigr) = \emptyset.
        \end{aligned}
    \]
\end{proof}

\noindent
We can interpret the lemma~\ref{lem:suff-cond-dependencty} more informally as follows: Let $\mathcal{C}_Y$ and $d$ be as in the statement of the lemma. If we intersect any $d$ sets from $\mathcal{C}_Y$, the result can be expressed as an intersection of fewer than $d$ sets. We will use this lemma to prove the following lemma.

\begin{lemma}
    \label{lem:dependent-boolean-comb}
    Let $\mathcal{C}_Y$ and $d$ be given such that they satisfy the hypothesis of~\ref{lem:suff-cond-dependencty}. Let $\mathcal{C} \subseteq \mathcal{P}(Y)$ and suppose $e \in \mathbb{N}$ is such that each set in $\mathcal{C}$ is a boolean combination of at most $e$ sets in $\mathcal{C}_Y$. Then $\mathcal{C}$ is dependent. More precisely, $\mathcal{B}(A_1, \dots, A_n)$ has at most $p_d(en)$ atoms, for all $A_1, \dots, A_n \in \mathcal{C}$.
\end{lemma}

\begin{proof}

    Let $A_1, \dots, A_n$ be an arbitrary sequence of sets from $\mathcal{C}$. By hypothesis, for each $i \in \{1, \dots, n\}$, the set $A_i$ is a boolean combination of some collection of sets $\mathcal{G}_i \subseteq \mathcal{C}_Y$ where $|\mathcal{G}_i| \le e$.

    The boolean algebra generated by the sequence $(A_1, \dots, A_n)$, denoted $\mathcal{B}(A_1, \dots, A_n)$, is a subalgebra of the boolean algebra generated by the union of all these individual generating sets. Let $\mathcal{F} = \bigcup_{i=1}^n \mathcal{G}_i$. Then every set $A_i$ is in $\mathcal{B}(\mathcal{F})$, and consequently, $\mathcal{B}(A_1, \dots, A_n)$ is contained within $\mathcal{B}(\mathcal{F})$.

    The number of atoms in a subalgebra is at most the number of atoms in the larger algebra. Therefore, the number of atoms in $\mathcal{B}(A_1, \dots, A_n)$ is less than or equal to the number of atoms in $\mathcal{B}(\mathcal{F})$.

    The size of the set $\mathcal{F}$ can be bounded as follows:
    \[
        m \coloneq |\mathcal{F}| = \left|\bigcup_{i=1}^n \mathcal{G}_i\right| \le \sum_{i=1}^n |\mathcal{G}_i| \le ne.
    \]
    The collection $\mathcal{F}$ consists of $m$ sets, all of which are from $\mathcal{C}_Y$. From lemma~\ref{lem:suff-cond-dependencty}, we know that the collection $\mathcal{C}_Y$ is dependent and contains no independent sequence of size $d$.

    Since $\mathcal{F}$ is a collection of $m$ sets from the dependent collection $\mathcal{C}_Y$, we can apply the proposition~\ref{prop:number-of-atoms}. It states that the number of atoms in $\mathcal{B}(\mathcal{F})$ is at most $p_d(m)$. Combining this with our bound on $m$:
    \[
        \text{Number of atoms in } \mathcal{B}(A_1, \dots, A_n) \le \text{Number of atoms in } \mathcal{B}(\mathcal{F}) \le p_d(m) \le p_d(en).
    \]
    The final inequality holds because $m \le en$ and the polynomial function $p_d(x)$ is non-decreasing for $x \ge d-1$.

    Since the number of atoms in $\mathcal{B}(A_1, \dots, A_n)$ is bounded by $p_d(en)$, which is a polynomial in $n$, it cannot be equal to $2^n$ for all $n \in \mathbb{N}$. Therefore, the collection $\mathcal{C}$ is dependent, and more precisely, the boolean algebra $\mathcal{B}(A_1, \dots, A_n)$ has at most $p_d(en)$ atoms.
\end{proof}
