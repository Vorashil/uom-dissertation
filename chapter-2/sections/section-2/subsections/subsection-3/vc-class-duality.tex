\subsection{Proof of Duality of VC-Classes and Dependence}

In the preceding subsections, we have introduced the Vapnik-Chervonenkis property from the perspective of a collection of sets $\C$ cutting out subsets from a point set $X$. We have also introduced dual notion of dependence and explained the duality between finite VC-dimension and dependence of parameter space. The main result of this subsection is to demonstrate that these two perspectives are, in fact, equivalent. We begin by formalizing the complexity measure for the dual case.

\begin{definition}[Dependency Growth Function and Index]
    Let $Y$ be an infinite set and let $\mathcal{C}_Y \subseteq \mathcal{P}(Y)$ be a collection of subsets of $Y$.
    \begin{enumerate}
        \item The \textbf{dependency growth function} of $\mathcal{C}_Y$, denoted $f^{\mathcal{C}_Y}: \mathbb{N} \to \mathbb{N}$, is defined as
        \[
            f^{\mathcal{C}_Y}(n) \coloneq \max \left\{ \text{number of atoms of } \mathcal{B}(S_1, \dots, S_n) \mid S_1, \dots, S_n \in \mathcal{C}_Y \right\},
        \]
        where $\mathcal{B}(S_1, \dots, S_n)$ is the Boolean algebra of subsets of $Y$ generated by $S_1, \dots, S_n$.
        \item If $\mathcal{C}_Y$ is dependent, its \textbf{dependency index}, denoted $D(\mathcal{C}_Y)$, is the smallest integer $d \in \mathbb{N}$ such that $f^{\mathcal{C}_Y}(d) < 2^d$. If $\mathcal{C}_Y$ is independent, we set $D(\mathcal{C}_Y) = \infty$.
    \end{enumerate}
\end{definition}

Now, we can state the main proposition connecting the VC-index with dependency index.

\begin{proposition}[VC-Dependence Duality]\label{prop:vc-dependence-duality}
    Suppose $X$ and $Y$ are infinite sets and $\Phi \subseteq X \times Y$ is a binary relation. Let $\mathcal{C}_X = \Phi^Y \coloneqq \{\Phi_y \mid y \in Y\} \subseteq \mathcal{P}(X)$ and $\mathcal{C}_Y = \Phi_X \coloneqq \{\Phi_x \mid x \in X\} \subseteq \mathcal{P}(Y)$. Then the complexity measures of these two collections are identical. Specifically:
    \begin{enumerate}
        \item The growth functions are equal: $f_{\mathcal{C}_X}(n) = f^{\mathcal{C}_Y}(n)$ for all $n \in \mathbb{N}$.
        \item The VC-index of $\mathcal{C}_X$ equals the dependency index of $\mathcal{C}_Y$: $V(\mathcal{C}_X) = D(\mathcal{C}_Y)$.
        \item $\mathcal{C}_X$ is a VC-class if and only if $\mathcal{C}_Y$ is a dependent collection.
    \end{enumerate}
\end{proposition}

\begin{proof}

    The latter two claims follow directly from the first. We therefore focus on proving that $f_{\mathcal{C}_X}(n) = f^{\mathcal{C}_Y}(n)$.

    Let $n \in \mathbb{N}$. To determine $f_{\mathcal{C}_X}(n)$, we must consider an arbitrary $n$-element subset $F = \{x_1, \dots, x_n\} \subseteq X$ and count the number of distinct subsets of $F$ that can be cut out by elements of $\mathcal{C}_X$. A subset $E \subseteq F$ is cut out by $\mathcal{C}_X$ if there exists a $y \in Y$ such that $E = F \cap \Phi_y$.

    To determine $f^{\mathcal{C}_Y}(n)$, we consider an arbitrary collection of $n$ sets from $\mathcal{C}_Y$, say $\{\Phi_{x_1}, \dots, \Phi_{x_n}\}$, and count the number of non-empty atoms in the Boolean algebra they generate. An atom is a set of the form $\bigcap_{i=1}^n \Phi_{x_i}^{\epsilon_i}$, where $\epsilon_i \in \{1, -1\}$, $\Phi_{x_i}^1 = \Phi_{x_i}$, and $\Phi_{x_i}^{-1} = Y \setminus \Phi_{x_i}$.

    The equivalence arises from the fact that these are two descriptions of the same underlying condition. Let $F = \{x_1, \dots, x_n\} \subseteq X$. A subset $E \subseteq F$ is cut out by some $\Phi_y \in \C$ if and only if:
    $$ \exists y \in Y \text{ such that } ( \forall x_i \in E, x_i \in \Phi_y ) \land ( \forall x_j \in F \setminus E, x_j \notin \Phi_y ) $$
    By the fundamental definition of the dual collections, the condition $x \in \Phi_y$ is equivalent to $(x,y) \in \Phi$, which is in turn equivalent to $y \in \Phi_x$. Applying this translation, the condition above becomes:
    $$ \exists y \in Y \text{ such that } ( \forall x_i \in E, y \in \Phi_{x_i} ) \land ( \forall x_j \in F \setminus E, y \notin \Phi_{x_j} ) $$
    This is precisely the definition that the atom corresponding to the set $E$, given by
    $$ \left( \bigcap_{x_i \in E} \Phi_{x_i} \right) \cap \left( \bigcap_{x_j \in F\setminus E} (Y \setminus \Phi_{x_j}) \right) $$
    is non-empty.

    Thus, for any finite set $F = \{x_1, \dots, x_n\} \subseteq X$, there is a one-to-one correspondence between the subsets of $F$ that can be cut out by $\mathcal{C}_X$ and the non-empty atoms of the Boolean algebra generated by the corresponding sets $\{\Phi_{x_1}, \dots, \Phi_{x_n}\} \subseteq \mathcal{C}_Y$. Since the counts are equal for any choice of $n$ elements, their maximum possible values must also be equal. Therefore, $f_{\mathcal{C}_X}(n) = f^{\mathcal{C}_Y}(n)$.
\end{proof}

This duality is symmetric. A less obvious but equally important result is that the notion of independence is also symmetric.

\begin{proposition}\label{prop:indep-equiv}
    With the notation above, the collection $\Phi^Y$ is independent if and only if the collection $\Phi_X$ is independent.
\end{proposition}

\begin{proof}

    By symmetry of the relation, we need only prove one direction. We will show that if the collection $\Phi_X$ is independent, then the collection $\Phi^Y$ must also be independent.

    Assume $\Phi_X$ is independent. By definition, this means that for any integer $n \in \mathbb{N}$, there exists an independent sequence of $n$ sets from $\Phi_X$. Let us choose such a sequence, $\{\Phi_{x_1}, \Phi_{x_2}, \dots, \Phi_{x_n}\}$, for some distinct $x_1, \dots, x_n \in X$. The independence of this sequence implies that for every possible choice of signs, represented by a vector $\epsilon = (\epsilon_1, \dots, \epsilon_n) \in \{1, -1\}^n$, the corresponding atom $\bigcap_{i=1}^n \Phi_{x_i}^{\epsilon_i}$ is a non-empty subset of $Y$.

    For each of the $2^n$ such non-empty atoms, we can select a witness element from $Y$. Let $y(\epsilon)$ be an element chosen from the atom corresponding to the vector $\epsilon$. By construction, $y(\epsilon)$ has the property that for each $i \in \{1, \dots, n\}$, we have $y(\epsilon) \in \Phi_{x_i}$ if and only if $\epsilon_i = 1$.

    Using the duality of the relation $\Phi$, this is equivalent to $(x_i, y(\epsilon)) \in \Phi$, which in turn is equivalent to $x_i \in \Phi_{y(\epsilon)}$. This demonstrates that for any desired subset of $\{x_1, \dots, x_n\}$, we can find a set in the collection $\Phi^Y$ that cuts it out precisely. Therefore, the set $\{x_1, \dots, x_n\}$ is shattered by $\Phi^Y$.

    Since we can find such a shattered set of size $n$ for any $n \in \mathbb{N}$, the collection $\Phi^Y$ is, by definition, independent.
\end{proof}

Given these equivalences, it is natural to assign these properties to the binary relation $\Phi$ itself.

\begin{definition}{\label{def:relation-dependence}}
    A binary relation $\Phi \subseteq X \times Y$ is said to be \textbf{dependent} if the collection $\Phi_X \subseteq \mathcal{P}(Y)$ is dependent. Otherwise, $\Phi$ is \textbf{independent}. We define the \textbf{dependency index of $\Phi$} as $D(\Phi) \coloneqq D(\Phi_X)$.
\end{definition}

By the results of this subsection, the following are all equivalent:
\begin{itemize}
    \item $\Phi$ is dependent.
    \item $\Phi_X$ is a dependent collection.
    \item $\Phi^Y$ is a VC-class.
    \item $\Phi_X$ is not independent.
    \item $\Phi^Y$ is not independent.
\end{itemize}
This robust set of equivalences allows us to freely switch between the language of VC-classes and dependence when analyzing definable sets.