\subsection{Boolean Operations on Relations}

The framework of dependence is robust under logical operations. In this subsubsection, we address exercises from the source text~\cite[Chap 5, \S 2]{vandenDries1998} to show that the class of dependent relations is closed under finite Boolean combinations.


Given two relations $\Phi, \Psi \subseteq X \times Y$, we define their negation, union (disjunction), and intersection (conjunction) as follows:
\begin{itemize}
    \item \textbf{Negation ($\neg\Phi$):} $(x,y) \in \neg\Phi$ if and only if $(x,y) \notin \Phi$.
    \item \textbf{Union ($\Phi \lor \Psi$):} $(x,y) \in \Phi \lor \Psi$ if and only if $(x,y) \in \Phi$ or $(x,y) \in \Psi$.
    \item \textbf{Intersection ($\Phi \land \Psi$):} $(x,y) \in \Phi \land  \Psi$ if and only if $(x,y) \in \Phi$ and $(x,y) \in \Psi$.
\end{itemize}
Note that for any relation $\Phi$, its negation $\neg\Phi$ has the same dependency growth function. This is because the atoms of the Boolean algebra generated by $\{\Phi_{x_i}\}_{i=1}^n$ are in one-to-one correspondence with the atoms generated by their complements $\{\neg\Phi_{x_i}\}_{i=1}^n$; taking complements merely relabels which atoms are "inside" or "outside" the generating sets, without changing the total number of non-empty regions. Thus, $f^\Phi = f^{\neg\Phi}$.

\begin{lemma}
    Let $\Phi, \Psi \subseteq X \times Y$ be two binary relations. Then for all $n \in \mathbb{N}$, the dependency growth functions satisfy:
    \[
        f^{\Phi \lor \Psi}(n) \le f^\Phi(n) \cdot f^\Psi(n) \quad \text{and} \quad f^{\Phi  \land  \Psi}(n) \le f^\Phi(n) \cdot f^\Psi(n).
    \]
\end{lemma}

\begin{proof}

    We first prove the inequality for the union, $f^{\Phi \lor \Psi}(n) \le f^\Phi(n) \cdot f^\Psi(n)$. The proof of the second inequality follow from the first one and the De Morgan's law.

    Let $n \in \mathbb{N}$ and choose an arbitrary sequence of $n$ points, $x_1, \dots, x_n \in X$. Let
    \[
        \mathcal{A}_\Phi = \mathcal{B}(\Phi_{x_1}, \dots, \Phi_{x_n}) \text{ and } \mathcal{A}_\Psi = \mathcal{B}(\Psi_{x_1}, \dots, \Psi_{x_n})
    \]
    be the Boolean algebras generated by the corresponding fibers. We denote the set of atoms of these algebras by $\text{atoms}(\mathcal{A}_\Phi)$ and $\text{atoms}(\mathcal{A}_\Psi)$, each form a partition of $Y$. Both of these sets of atoms form partition of $Y$ and are disjoint by definition. Also note that, by the definition of the growth function, we have:
    \[
        |\text{atoms}(\mathcal{A}_\Phi)| \le f^\Phi(n) \quad \text{and} \quad |\text{atoms}(\mathcal{A}_\Psi)| \le f^\Psi(n).
    \]
    Consider the collection
    \[
        Z \coloneq \{A \cap B \mid A \in \text{atom}(\mathcal{A}_\Phi) \land B \in \text{atom}(\mathcal{A}_\Psi) \land A \cap B \neq \emptyset \}.
    \]
    $Z$ forms a refined partition of $Y$, since every point $y \in Y$ belongs to exactly one such intersection. The number of elements in this refined partition is bounded above
    \[
        |Z| \leq |\text{atoms}(\mathcal{A}_\Phi)| \cdot |\text{atoms}(\mathcal{A}_\Psi)|.
    \]

    Now, let $\mathcal{A}_{\Phi \lor \Psi} = \mathcal{B}((\Phi \lor \Psi)_{x_1}, \dots, (\Phi \lor \Psi)_{x_n})$. The proof is complete if show that the number of atoms of $\mathcal{A}_{\Phi \lor \Psi}$ is bounded above by $|Z|$. Since $\text{atom}(\mathcal{A}_{\Phi \lor \Psi})$ also paritions $Y$, every element of $Z$ must have non-empty intersection with an element of $\text{atom}(\mathcal{A}_{\Phi \lor \Psi})$. Hence, if we show that each element of $Z$ is contained entirely in an atom of $\mathcal{A}_{\Phi \lor \Psi}$ we get the desired the desired upper bound for $|\text{atom}(\mathcal{A}_{\Phi \lor \Psi})|$.

    Let $S \coloneq A \cap B$ be an element of $Z$, where $A \in \text{atoms}(\mathcal{A}_\Phi)$ and $B \in \text{atoms}(\mathcal{A}_\Psi)$. Take any two points $y_1, y_2 \in S$. Then
    \[
        y_1, y_2 \in A \quad \Rightarrow \quad \forall i \in \{1, \dots, n\}, \quad (y_1 \in \Phi_{x_i} \iff y_2 \in \Phi_{x_i}).
    \]
    Similarly,
    \[
        y_1, y_2 \in B \quad \Rightarrow \quad \forall i \in \{1, \dots, n\}, \quad (y_1 \in \Psi_{x_i} \iff y_2 \in \Psi_{x_i}).
    \]
    Then combining these two implications,
    \[
        y_1, y_2 \in A \cap B \quad \Rightarrow \forall i \in \{1, \dots, n\}, (y_1 \in \Phi_{x_i} \lor y_1 \in \Psi_{x_i} \iff y_2 \in \Phi_{x_i} \lor y_2 \in \Psi_{x_i}).
    \]
    It follows that for each $i$, ($y_1 \in \Phi_{x_i} \cup \Psi_{x_i} \iff y_2 \in \Phi_{x_i} \cup \Psi_{x_i}$). This means that $y_1$ and $y_2$ belong to the same atom of $\mathcal{A}_{\Phi \lor \Psi}$, since $\Phi_{x_i} \cup \Psi_{x_i} = (\Phi \lor \Psi)_{x_i}$. As this holds for any pair of points in $S$, the entire set $S$ must be contained within a single atom of $\mathcal{A}_{\Phi \lor \Psi}$.

    This implies that, by pigeonhole principle the partition of $Y$ formed by $\text{atoms}(\mathcal{A}_{\Phi \lor \Psi})$ consists of fewer elements than the partition formed by $Z$. Therefore, the number of atoms is bounded:
    \[
        |\text{atoms}(\mathcal{A}_{\Phi \lor \Psi})| \leq |Z| \leq |\text{atoms}(\mathcal{A}_\Phi)| \cdot |\text{atoms}(\mathcal{A}_\Psi)| \le f^\Phi(n) \cdot f^\Psi(n).
    \]
    Since this inequality holds for any arbitrary choice of $x_1, \dots, x_n$, it must also hold for the maximum value, which gives $f^{\Phi \lor \Psi}(n) \le f^\Phi(n) \cdot f^\Psi(n)$.

    The second inequality, for $\Phi  \land  \Psi$, follows directly from the first inequality and De Morgan's laws:
    \[
        f^{\Phi \land  \Psi}(n) = f^{\neg(\neg\Phi \lor \neg\Psi)}(n) = f^{\neg\Phi \lor \neg\Psi}(n) \le f^{\neg\Phi}(n) \cdot f^{\neg\Psi}(n) = f^\Phi(n) \cdot f^\Psi(n).
    \]
\end{proof}

\begin{lemma}{\label{lem:dependent-union-intersection}}
    If $\Phi$ and $\Psi$ are dependent relations, then their union $\Phi \lor \Psi$ and intersection $\Phi \land  \Psi$ are also dependent.
\end{lemma}

\begin{proof}

    If $\Phi$ is dependent, its growth function $f^\Phi(n)$ is bounded by a polynomial in $n$ for all sufficiently large $n$. That is, there exists a polynomial $P(n)$ and an integer $N_\Phi$ such that $f^\Phi(n) \le P(n)$ for all $n \ge N_\Phi$. Similarly, if $\Psi$ is dependent, there exists a polynomial $Q(n)$ and an integer $N_\Psi$ such that $f^\Psi(n) \le Q(n)$ for all $n \ge N_\Psi$.

    From the previous lemma, we know that $f^{\Phi \lor \Psi}(n) \le f^\Phi(n) \cdot f^\Psi(n)$. For $n \ge \max(N_\Phi, N_\Psi)$, this implies:
    \[
        f^{\Phi \lor \Psi}(n) \le P(n) \cdot Q(n).
    \]
    The product of two polynomials is another polynomial. Since any polynomial function of $n$ grows slower than the exponential function $2^n$, there must exist an integer $d$ such that for all sufficiently large $n$, $f^{\Phi \lor \Psi}(n) < 2^n$. This is the definition of a dependent relation.

    The exact same argument holds for the relation $\Phi \land \Psi$, since its growth function is also bounded by the polynomial $P(n) \cdot Q(n)$. Therefore, dependence is preserved under these operations.
\end{proof}

