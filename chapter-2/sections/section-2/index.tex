\section{Non-independence property}

In this section, we build on the notions introduced earlier—specifically, \textit{shattering} and the \textit{growth function}—to formalise the Non-Independence Property (NIP) for families of sets. Our exposition follows~\cite[Chap.~5]{vandenDries1998}. We begin by defining the Vapnik–Chervonenkis (VC) dimension of a family of sets in terms of shattering. We then show that, when the VC-dimension is finite, this property admits an equivalent characterisation via a model-theoretic notion of \textit{dependence}. These dual perspectives on finite VC-dimension will serve as the foundation for the main result of this section: informally, that if a collection of subsets is dependent in lower dimensions, then this dependence persists in higher dimensions.

\subsection{VC-Classes and VC-Dimension}

\begin{definition}[VC-Class and VC-Dimension]\label{def:vc-class-dimension}
    A collection $\mathcal{C} \subseteq \mathcal{P}(X)$ is called a \emph{Vapnik-Chervonenkis class} (or \emph{VC-class}) if there exists some $d \in \mathbb{N}$ such that $\mathcal{C}$ does not shatter any set of size $d$. The \emph{VC-dimension} of $\mathcal{C}$, denoted $V(\mathcal{C})$, is defined as:
    \[
        V(\mathcal{C}) \coloneq \min \{|F| : F \subseteq X \text{ is finite and is not shattered by } \mathcal{C}\}.
    \]
    If $\mathcal{C}$ is a VC-class, then $V(\mathcal{C})$ is the size of the largest finite set shattered by $\mathcal{C}$. If $\mathcal{C}$ is not a VC-class (i.e., it shatters finite sets of arbitrarily large size), we set $V(\mathcal{C}) = \infty$.
\end{definition}

Thus, a collection $\mathcal{C}$ is a VC-class if and only if its VC-dimension is finite. The dichotomy theorem states that $V(\mathcal{C}) < \infty$ if and only if $f_{\mathcal{C}}:\mathbb{N} \to \mathbb{N}$ has polynomial growth, while $V(\mathcal{C}) = \infty$ if and only if $f_{\mathcal{C}}(n) = 2^n$ for all $n$.

\begin{example}[Intervals]
    Let $X = \mathbb{R}$ and let $\mathcal{C}$ be the collection of all closed intervals $[a, b]$.
    \begin{itemize}
        \item $\mathcal{C}$ shatters the set $F_1 = \{5\}$. The trace is $\{\emptyset, \{5\}\}$ (using intervals like $[0,1]$ and $[4,6]$).
        \item $\mathcal{C}$ shatters the set $F_2 = \{5, 10\}$. To get all four subsets:
        \begin{itemize}
            \item $\emptyset$: use $[0,1]$
            \item $\{5\}$: use $[4,6]$
            \item $\{10\}$: use $[9,11]$
            \item $\{5, 10\}$: use $[4,11]$
        \end{itemize}
        \item $\mathcal{C}$ does \emph{not} shatter $F_3 = \{5, 10, 15\}$. It is impossible to find an interval that contains $\{5, 15\}$ but not $\{10\}$.
    \end{itemize}
    The largest set shattered has size 2, so $V(\mathcal{C}) = 2$.
\end{example}

\begin{example}[An Infinite VC-Dimension Class]
    \label{ex:vc-infinite-dimension}
    Let $X = \{p \in \mathbb{N} \mid n \textrm{ is prime}\}$ be the given set. Consider the collection of subsets
    \[
        \mathcal{C} = \{C_n : n \in \mathbb{N}\} \textrm{ where } C_n = \{p \in X : p \text{ divides } n\}
    \]
    We claim $V(\mathcal{C}) = \infty$. To prove this, we must show that $\mathcal{C}$ can shatter any finite set of primes $F = \{p_1, p_2, \dots, p_d\}$.

    Let $E = \{q_1, \dots, q_k\}$ be an arbitrary subset of $F$. We need to find an integer $n$ such that $C_n \cap F = E$. The choice is straightforward: let $n = q_1 \cdot q_2 \cdot \dots \cdot q_k$. By construction, every prime in $E$ divides $n$. Furthermore, no prime in $F \setminus E$ can divide $n$, by the fundamental theorem of arithmetic. Thus, $C_n \cap F = E$. Since we can do this for any subset $E \subseteq F$, $\mathcal{C}$ shatters $F$. As $F$ was an arbitrary finite set of primes, $\mathcal{C}$ shatters sets of all sizes, and $V(\mathcal{C}) = \infty$.
\end{example}
\subsection{Dependence property - Dual perspective into finite VC-dimension}

When a collection $\mathcal{C}$ is naturally parametrised by the elements of another set—such as in Example~\ref{ex:vc-infinite-dimension}, where $\mathcal{C}$ was indexed by the natural numbers—its VC-dimension can often be analysed via a corresponding \textit{model-theoretic} property of the parameter space, namely \textit{(non-in)dependence} propery, shortened as \textit{(N)IP}. In particular, the concept of dependence in model theory serves as a dual perspective to finite VC-dimension. We will see that a parametrised family of sets is dependent precisely when it has finite VC-dimension.

To make this correspondence precise, we adopt the following formal framework. Let $X$ and $Y$ be infinite sets, and let $\Phi \subseteq X \times Y$ be a binary relation. For each $x \in X$ and $y \in Y$, define the sets
\[
    \Phi_x = \{y \in Y : (x,y) \in \Phi\} \subseteq Y \quad \text{and} \quad \Phi^y = \{x \in X : (x,y) \in \Phi\} \subseteq X.
\]

\noindent
Then we define the collections of sets indexed by $Y$ and $X$ correspondingly:
\[
    \mathcal{C}_X = \{\Phi^y : y \in Y\}  \quad \text{and} \quad \mathcal{C}_Y = \{\Phi_x : x \in X\}.
\]
\noindent
Here, $Y$ is the parameter space for $\mathcal{C}_X\subseteq \mathcal{P}(X)$, and $X$ is the parameter space for $\mathcal{C}_Y \subseteq \mathcal{P}(Y)$.

We focus on the trace of $\mathcal{C}_X$ on a finite set $F \subseteq X$. A subset $E \subseteq F$ belongs to the trace $\mathcal{C}_X \cap F$ if and only if there exists some parameter $y \in Y$ such that $E = \Phi_y \cap F$. This condition can be rephrased:
\[
    \begin{aligned}
        E = \Phi_y \cap F \quad &\Leftrightarrow \quad [\forall x \in E, (x,y) \in \Phi] \text{ and } [\forall x \in F \setminus E, (x,y) \notin \Phi] \\
        &\Leftrightarrow \quad [\forall x \in E, y \in \Phi_x] \text{ and } [\forall x \in F \setminus E, y \notin \Phi_x] \\
    \end{aligned}
\]
Therefore,
\begin{equation}
    \label{eq:trace-condition}
    E \in \mathcal{C}_X \cap F \quad \Leftrightarrow \quad \left( \bigcap_{x \in E} \Phi_x \right) \cap \left( \bigcap_{x \in F \setminus E} (Y \setminus \Phi_x) \right) \neq \emptyset. \quad
\end{equation}


Recall that, \emph{atom} is a minimal non-empty set in a Boolean algebra of sets~\cite[Chap 1]{vandenDries1998} in terms of inclusion. If we consider the boolean algebra generated by the sets $\{\Phi_x : x \in F\}$, denoted as $\mathcal{B}(\{\Phi_x : x \in F\})$, the expression on the right-hand side of~\ref{eq:trace-condition} is precisely an atom of this Boolean algebra for each subset $E \in \mathcal{P}(F)$.

Hence, given $\mathcal{C}_X$ and a finite subset $F \subseteq X$ with $|F| = n$, we can use this duality described above to determine if $\mathcal{C}_X$ shatters $F$. More specifically, this can be summarized as follows:
\[
    \begin{aligned}
        \mathcal{C}_X \text{ shatters } F \quad &\Leftrightarrow \quad |\mathcal{C}_X \cap F| = 2^n \\
        &\Leftrightarrow \quad \mathcal{B}(\{\Phi_x : x \in F\}) \text{ has } 2^n \text{ atoms} \\
    \end{aligned}
\]

Note that, here duality emerges from the fact that $\mathcal{C}_X$ is collection of subsets of $X$ indexed by $Y$, while $\{\Phi_x : x \in F\}$ is a collection of subsets of $Y$ indexed by the finite set $F\subset X$. The atoms of the Boolean algebra $\mathcal{B}(\{\Phi_x : x \in F\})$ correspond to the subsets of $F$ that can be isolated by the parameters from $Y$. We will not use this argument based on the atoms of boolean algebra to define a model-theoretic property of $\mathcal{C}_Y$ called \textit{independence}.

Firstly, we need to introduce some notation that will be useful when we use dependence property of a collection of subsets. Let \(Y\) be an arbitrary infinite set, and let \(S \subseteq Y\). We introduce the following notation for any subset \(S\) of \(Y\):
\[
    S^1 \coloneqq S,
    \quad
    S^{-1} \coloneqq Y \setminus S.
\]
\noindent
Next, let \(G = \{1,\dots,n\}\subseteq\mathbb{N}\). For each \(H\subseteq G\), define the sign‐function
\[
    \begin{equation}
        \label{eq:sign-function}
        \mathds{1}_H : G \to \{1,-1\},
        \qquad
        \mathds{1}_H(i) =
        \begin{cases}
            1,  & i\in H,\\
            -1, & i\notin H.
        \end{cases}
    \end{equation}
\]
\noindent
Given subsets \(S_1,\dots,S_n \subseteq Y\) and any \(H\in\mathcal P(G)\), consider the intersection
\[
    S_1^{\mathds{1}_H(1)}
    \,\cap
    S_2^{\mathds{1}_H(2)}
    \,\cap\cdots\cap\,
    S_n^{\mathds{1}_H(n)}.
\]

There are \(2^n\) such intersections; they are pairwise disjoint and their union equals \(Y\). The sequence \((S_1,\dots,S_n)\) exhibits dependence precisely if at least one of these intersections is empty. More precisely, we have the following definition.


\begin{definition}[Independence property]
    Let $Y$ and $S_1, \ldots, S_n \subseteq Y$ be given.
    \begin{itemize}
        \item We say that the sequence $(S_1,  \ldots, S_n)$ is \emph{independent} (in $Y$) if the boolean algebra generated by the sets $S_1, \ldots, S_n$, denoted $\mathcal{B}(S_1, \ldots, S_n)$ has $2^n$ atoms. Otherwise, we say that the sequence is \emph{dependent} (in $Y$).

        \item If a collection $\mathcal{C}_Y \subseteq \mathcal{P}(Y)$ contains independent sequence of size $n$ for every $n \in \mathbb{N}$, we say that $\mathcal{C}_Y$ is \emph{independent} (in $Y$). Otherwise, we say that $\mathcal{C}_Y$ is \emph{dependent} (in $Y$).
    \end{itemize}
\end{definition}

Following proposition is a dual of ~\ref{cor:vc-dichotomy}.

\begin{proposition}{\label{prop:number-of-atoms}}
    Let $\mathcal{C}_Y \subseteq \mathcal{P}(Y)$ be a collection of subsets of an infinite set $Y$. If $\mathcal{C}_Y$ contains no independent sequence of size $d \ge 1$ for some $d$, then for all sequences $S_1, \ldots, S_n \in \mathcal{C}_Y$, the boolean algebra $\mathcal{B}(S_1, \dots, S_n)$ has at most $p_d(n)$ atoms.
\end{proposition}

\begin{proof}

    For $n < d$, $p_d(n) = 2^n$, so $C_Y$ shatters some set of size $n$. Assume, $n \geq d$ and let $S_1, \dots, S_n \in C_Y$. Also, let $F = \{1, \ldots, n\}$ and consider $\mathcal{D} \subseteq \mathcal{P}(F)$ where $\mathcal{D}$ contains subsets $D \subseteq F$ such that
    \[
        S_1^{\mathds{1}_D(1)} \cap S_2^{\mathds{1}_D(2)} \cap \cdots \cap S_n^{\mathds{1}_D(n)} \neq \emptyset.
    \]

    If $|\mathcal{D}| > p_d(n)$, then by the lemma~\ref{lem:dependent-boolean-comb}, there exists a subset $E \subseteq F$ of size $d$ such that $\mathcal{D}$ shatters $E$. Hence, the boolean algebra $\mathcal{B}(S_1, \dots, S_n)$ has $2^d$ atoms, which contradicts the assumption that $\mathcal{C}_Y$ contains no independent sequence of size $d$. Therefore, we must have $|\mathcal{D}| \leq p_d(n)$.
\end{proof}

One of the sufficient conditions for a sequence to be dependent is as follows:
\begin{lemma}{\label{lem:suff-cond-dependencty}}
    Let $\mathcal{C}_Y \subseteq \mathcal{P}(Y)$ be a collection and let $d \in \mathbb{N}$ be such that any non-empty intersection of $d$ sets from $\mathcal{C}_Y$, say $S_1, \dots, S_d \in \mathcal{C}_Y$, is equal to the intersection of at most $d-1$ of those sets $\mathcal{C}_Y$
    \[
        S_1 \cap S_2 \cap \cdots \cap S_d = \bigcap_{i\in D} S_i \quad \text{for some } D \subseteq \{1, 2, \ldots, d\} \text{ with } |D| < d.
    \]
    Then the collection $\mathcal{C}_Y$ is dependent in $Y$.
\end{lemma}

\begin{proof}

    It's enough to show that there is no independent sequence of size $d$ in $\mathcal{C}_Y$. In other words, for an arbitrary sequence $S_1, \dots, S_d$, the number of atoms of the boolean algebra generated by this sequence, $\mathcal{B}(S_1, \dots, S_d)$ is strictly less than $2^d$.

    Let $S_1, \dots, S_d$ be an arbitrary sequence of sets in $\mathcal{C}_Y$. If $S_1 \cap \dots \cap S_d$ is empty then we are done, since this intersection cannot be atom of $\mathcal{B}(S_1, \dots, S_d)$ and this boolean algebra has strictly less than $2^d$ atoms. On the other hand, if $S_1 \cap \dots \cap S_d$ is non-empty, then by assumption
    \begin{equation}
        \label{eq:proof-suff-cont-1}
        S_1 \cap \dots \cap S_d = \bigcap_{i \in D} S_i
    \end{equation}
    for some proper subset $D \subset \{1, \dots, d\}$. It follows that the intersection
    \begin{equation}
        \label{eq:proof-suff-cont-2}
        \begin{aligned}
            \bigl(\bigcap_{i \in D} S_i\bigr) \cap \bigl(\bigcap_{j \not \in D} S^{-1}_j\bigr) =  S_1^{\mathds{1}_D(1)} \cap S_2^{\mathds{1}_D(2)} \cap \cdots \cap S_n^{\mathds{1}_D(n)}\\
        \end{aligned}
    \end{equation}
    is also empty combining~\ref{eq:proof-suff-cont-1} and~\ref{eq:proof-suff-cont-2}, where $\mathds{1}_{D}: \{1, \dots, d\} \to \{-1, 1\}$ is a sign function defined in~\ref{eq:sign-function}. To see this, note that combining~\ref{eq:proof-suff-cont-1} and~\ref{eq:proof-suff-cont-2} we get intersection of two sets that are complement of each other, more precisely,
    \[
        \begin{aligned}
            \bigl(\bigcap_{i \in D} S_i\bigr) \cap \bigl(\bigcap_{j \not \in D} S^{-1}_j\bigr) = S_1 \cap \dots \cap S_d \cap \bigl(\bigcap_{j \not \in D} S^{-1}_j\bigr) = \emptyset.
        \end{aligned}
    \]
\end{proof}

\noindent
We can interpret the lemma~\ref{lem:suff-cond-dependencty} more informally as follows: Let $\mathcal{C}_Y$ and $d$ be as in the statement of the lemma. If we intersect any $d$ sets from $\mathcal{C}_Y$, the result can be expressed as an intersection of fewer than $d$ sets. We will use this lemma to prove the following lemma.

\begin{lemma}
    \label{lem:dependent-boolean-comb}
    Let $\mathcal{C}_Y$ and $d$ be given such that they satisfy the hypothesis of~\ref{lem:suff-cond-dependencty}. Let $\mathcal{C} \subseteq \mathcal{P}(Y)$ and suppose $e \in \mathbb{N}$ is such that each set in $\mathcal{C}$ is a boolean combination of at most $e$ sets in $\mathcal{C}_Y$. Then $\mathcal{C}$ is dependent. More precisely, $\mathcal{B}(A_1, \dots, A_n)$ has at most $p_d(en)$ atoms, for all $A_1, \dots, A_n \in \mathcal{C}$.
\end{lemma}

\begin{proof}

    Let $A_1, \dots, A_n$ be an arbitrary sequence of sets from $\mathcal{C}$. By hypothesis, for each $i \in \{1, \dots, n\}$, the set $A_i$ is a boolean combination of some collection of sets $\mathcal{G}_i \subseteq \mathcal{C}_Y$ where $|\mathcal{G}_i| \le e$.

    The boolean algebra generated by the sequence $(A_1, \dots, A_n)$, denoted $\mathcal{B}(A_1, \dots, A_n)$, is a subalgebra of the boolean algebra generated by the union of all these individual generating sets. Let $\mathcal{F} = \bigcup_{i=1}^n \mathcal{G}_i$. Then every set $A_i$ is in $\mathcal{B}(\mathcal{F})$, and consequently, $\mathcal{B}(A_1, \dots, A_n)$ is contained within $\mathcal{B}(\mathcal{F})$.

    The number of atoms in a subalgebra is at most the number of atoms in the larger algebra. Therefore, the number of atoms in $\mathcal{B}(A_1, \dots, A_n)$ is less than or equal to the number of atoms in $\mathcal{B}(\mathcal{F})$.

    The size of the set $\mathcal{F}$ can be bounded as follows:
    \[
        m \coloneq |\mathcal{F}| = \left|\bigcup_{i=1}^n \mathcal{G}_i\right| \le \sum_{i=1}^n |\mathcal{G}_i| \le ne.
    \]
    The collection $\mathcal{F}$ consists of $m$ sets, all of which are from $\mathcal{C}_Y$. From lemma~\ref{lem:suff-cond-dependencty}, we know that the collection $\mathcal{C}_Y$ is dependent and contains no independent sequence of size $d$.

    Since $\mathcal{F}$ is a collection of $m$ sets from the dependent collection $\mathcal{C}_Y$, we can apply the proposition~\ref{prop:number-of-atoms}. It states that the number of atoms in $\mathcal{B}(\mathcal{F})$ is at most $p_d(m)$. Combining this with our bound on $m$:
    \[
        \text{Number of atoms in } \mathcal{B}(A_1, \dots, A_n) \le \text{Number of atoms in } \mathcal{B}(\mathcal{F}) \le p_d(m) \le p_d(en).
    \]
    The final inequality holds because $m \le en$ and the polynomial function $p_d(x)$ is non-decreasing for $x \ge d-1$.

    Since the number of atoms in $\mathcal{B}(A_1, \dots, A_n)$ is bounded by $p_d(en)$, which is a polynomial in $n$, it cannot be equal to $2^n$ for all $n \in \mathbb{N}$. Therefore, the collection $\mathcal{C}$ is dependent, and more precisely, the boolean algebra $\mathcal{B}(A_1, \dots, A_n)$ has at most $p_d(en)$ atoms.
\end{proof}

\subsection{Proof of Duality of VC-Classes and Dependence}

In the preceding subsections, we have introduced the Vapnik-Chervonenkis property from the perspective of a collection of sets $\C$ cutting out subsets from a point set $X$. We have also introduced dual notion of dependence and explained the duality between finite VC-dimension and dependence of parameter space. The main result of this subsection is to demonstrate that these two perspectives are, in fact, equivalent. We begin by formalizing the complexity measure for the dual case.

\begin{definition}[Dependency Growth Function and Index]
    Let $Y$ be an infinite set and let $\mathcal{C}_Y \subseteq \mathcal{P}(Y)$ be a collection of subsets of $Y$.
    \begin{enumerate}
        \item The \textbf{dependency growth function} of $\mathcal{C}_Y$, denoted $f^{\mathcal{C}_Y}: \mathbb{N} \to \mathbb{N}$, is defined as
        \[
            f^{\mathcal{C}_Y}(n) \coloneq \max \left\{ \text{number of atoms of } \mathcal{B}(S_1, \dots, S_n) \mid S_1, \dots, S_n \in \mathcal{C}_Y \right\},
        \]
        where $\mathcal{B}(S_1, \dots, S_n)$ is the Boolean algebra of subsets of $Y$ generated by $S_1, \dots, S_n$.
        \item If $\mathcal{C}_Y$ is dependent, its \textbf{dependency index}, denoted $D(\mathcal{C}_Y)$, is the smallest integer $d \in \mathbb{N}$ such that $f^{\mathcal{C}_Y}(d) < 2^d$. If $\mathcal{C}_Y$ is independent, we set $D(\mathcal{C}_Y) = \infty$.
    \end{enumerate}
\end{definition}

Now, we can state the main proposition connecting the VC-index with dependency index.

\begin{proposition}[VC-Dependence Duality]\label{prop:vc-dependence-duality}
    Suppose $X$ and $Y$ are infinite sets and $\Phi \subseteq X \times Y$ is a binary relation. Let $\mathcal{C}_X = \Phi^Y \coloneqq \{\Phi_y \mid y \in Y\} \subseteq \mathcal{P}(X)$ and $\mathcal{C}_Y = \Phi_X \coloneqq \{\Phi_x \mid x \in X\} \subseteq \mathcal{P}(Y)$. Then the complexity measures of these two collections are identical. Specifically:
    \begin{enumerate}
        \item The growth functions are equal: $f_{\mathcal{C}_X}(n) = f^{\mathcal{C}_Y}(n)$ for all $n \in \mathbb{N}$.
        \item The VC-index of $\mathcal{C}_X$ equals the dependency index of $\mathcal{C}_Y$: $V(\mathcal{C}_X) = D(\mathcal{C}_Y)$.
        \item $\mathcal{C}_X$ is a VC-class if and only if $\mathcal{C}_Y$ is a dependent collection.
    \end{enumerate}
\end{proposition}

\begin{proof}

    The latter two claims follow directly from the first. We therefore focus on proving that $f_{\mathcal{C}_X}(n) = f^{\mathcal{C}_Y}(n)$.

    Let $n \in \mathbb{N}$. To determine $f_{\mathcal{C}_X}(n)$, we must consider an arbitrary $n$-element subset $F = \{x_1, \dots, x_n\} \subseteq X$ and count the number of distinct subsets of $F$ that can be cut out by elements of $\mathcal{C}_X$. A subset $E \subseteq F$ is cut out by $\mathcal{C}_X$ if there exists a $y \in Y$ such that $E = F \cap \Phi_y$.

    To determine $f^{\mathcal{C}_Y}(n)$, we consider an arbitrary collection of $n$ sets from $\mathcal{C}_Y$, say $\{\Phi_{x_1}, \dots, \Phi_{x_n}\}$, and count the number of non-empty atoms in the Boolean algebra they generate. An atom is a set of the form $\bigcap_{i=1}^n \Phi_{x_i}^{\epsilon_i}$, where $\epsilon_i \in \{1, -1\}$, $\Phi_{x_i}^1 = \Phi_{x_i}$, and $\Phi_{x_i}^{-1} = Y \setminus \Phi_{x_i}$.

    The equivalence arises from the fact that these are two descriptions of the same underlying condition. Let $F = \{x_1, \dots, x_n\} \subseteq X$. A subset $E \subseteq F$ is cut out by some $\Phi_y \in \C$ if and only if:
    $$ \exists y \in Y \text{ such that } ( \forall x_i \in E, x_i \in \Phi_y ) \land ( \forall x_j \in F \setminus E, x_j \notin \Phi_y ) $$
    By the fundamental definition of the dual collections, the condition $x \in \Phi_y$ is equivalent to $(x,y) \in \Phi$, which is in turn equivalent to $y \in \Phi_x$. Applying this translation, the condition above becomes:
    $$ \exists y \in Y \text{ such that } ( \forall x_i \in E, y \in \Phi_{x_i} ) \land ( \forall x_j \in F \setminus E, y \notin \Phi_{x_j} ) $$
    This is precisely the definition that the atom corresponding to the set $E$, given by
    $$ \left( \bigcap_{x_i \in E} \Phi_{x_i} \right) \cap \left( \bigcap_{x_j \in F\setminus E} (Y \setminus \Phi_{x_j}) \right) $$
    is non-empty.

    Thus, for any finite set $F = \{x_1, \dots, x_n\} \subseteq X$, there is a one-to-one correspondence between the subsets of $F$ that can be cut out by $\mathcal{C}_X$ and the non-empty atoms of the Boolean algebra generated by the corresponding sets $\{\Phi_{x_1}, \dots, \Phi_{x_n}\} \subseteq \mathcal{C}_Y$. Since the counts are equal for any choice of $n$ elements, their maximum possible values must also be equal. Therefore, $f_{\mathcal{C}_X}(n) = f^{\mathcal{C}_Y}(n)$.
\end{proof}

This duality is symmetric. A less obvious but equally important result is that the notion of independence is also symmetric.

\begin{proposition}\label{prop:indep-equiv}
    With the notation above, the collection $\Phi^Y$ is independent if and only if the collection $\Phi_X$ is independent.
\end{proposition}

\begin{proof}

    By symmetry of the relation, we need only prove one direction. We will show that if the collection $\Phi_X$ is independent, then the collection $\Phi^Y$ must also be independent.

    Assume $\Phi_X$ is independent. By definition, this means that for any integer $n \in \mathbb{N}$, there exists an independent sequence of $n$ sets from $\Phi_X$. Let us choose such a sequence, $\{\Phi_{x_1}, \Phi_{x_2}, \dots, \Phi_{x_n}\}$, for some distinct $x_1, \dots, x_n \in X$. The independence of this sequence implies that for every possible choice of signs, represented by a vector $\epsilon = (\epsilon_1, \dots, \epsilon_n) \in \{1, -1\}^n$, the corresponding atom $\bigcap_{i=1}^n \Phi_{x_i}^{\epsilon_i}$ is a non-empty subset of $Y$.

    For each of the $2^n$ such non-empty atoms, we can select a witness element from $Y$. Let $y(\epsilon)$ be an element chosen from the atom corresponding to the vector $\epsilon$. By construction, $y(\epsilon)$ has the property that for each $i \in \{1, \dots, n\}$, we have $y(\epsilon) \in \Phi_{x_i}$ if and only if $\epsilon_i = 1$.

    Using the duality of the relation $\Phi$, this is equivalent to $(x_i, y(\epsilon)) \in \Phi$, which in turn is equivalent to $x_i \in \Phi_{y(\epsilon)}$. This demonstrates that for any desired subset of $\{x_1, \dots, x_n\}$, we can find a set in the collection $\Phi^Y$ that cuts it out precisely. Therefore, the set $\{x_1, \dots, x_n\}$ is shattered by $\Phi^Y$.

    Since we can find such a shattered set of size $n$ for any $n \in \mathbb{N}$, the collection $\Phi^Y$ is, by definition, independent.
\end{proof}

Given these equivalences, it is natural to assign these properties to the binary relation $\Phi$ itself.

\begin{definition}{\label{def:relation-dependence}}
    A binary relation $\Phi \subseteq X \times Y$ is said to be \textbf{dependent} if the collection $\Phi_X \subseteq \mathcal{P}(Y)$ is dependent. Otherwise, $\Phi$ is \textbf{independent}. We define the \textbf{dependency index of $\Phi$} as $D(\Phi) \coloneqq D(\Phi_X)$.
\end{definition}

By the results of this subsection, the following are all equivalent:
\begin{itemize}
    \item $\Phi$ is dependent.
    \item $\Phi_X$ is a dependent collection.
    \item $\Phi^Y$ is a VC-class.
    \item $\Phi_X$ is not independent.
    \item $\Phi^Y$ is not independent.
\end{itemize}
This robust set of equivalences allows us to freely switch between the language of VC-classes and dependence when analyzing definable sets.
\subsection{Boolean Operations on Relations}

The framework of dependence is robust under logical operations. In this subsubsection, we address exercises from the source text~\cite[Chap 5, \S 2]{vandenDries1998} to show that the class of dependent relations is closed under finite Boolean combinations.


Given two relations $\Phi, \Psi \subseteq X \times Y$, we define their negation, union (disjunction), and intersection (conjunction) as follows:
\begin{itemize}
    \item \textbf{Negation ($\neg\Phi$):} $(x,y) \in \neg\Phi$ if and only if $(x,y) \notin \Phi$.
    \item \textbf{Union ($\Phi \lor \Psi$):} $(x,y) \in \Phi \lor \Psi$ if and only if $(x,y) \in \Phi$ or $(x,y) \in \Psi$.
    \item \textbf{Intersection ($\Phi \land \Psi$):} $(x,y) \in \Phi \land  \Psi$ if and only if $(x,y) \in \Phi$ and $(x,y) \in \Psi$.
\end{itemize}
Note that for any relation $\Phi$, its negation $\neg\Phi$ has the same dependency growth function. This is because the atoms of the Boolean algebra generated by $\{\Phi_{x_i}\}_{i=1}^n$ are in one-to-one correspondence with the atoms generated by their complements $\{\neg\Phi_{x_i}\}_{i=1}^n$; taking complements merely relabels which atoms are "inside" or "outside" the generating sets, without changing the total number of non-empty regions. Thus, $f^\Phi = f^{\neg\Phi}$.

\begin{lemma}
    Let $\Phi, \Psi \subseteq X \times Y$ be two binary relations. Then for all $n \in \mathbb{N}$, the dependency growth functions satisfy:
    \[
        f^{\Phi \lor \Psi}(n) \le f^\Phi(n) \cdot f^\Psi(n) \quad \text{and} \quad f^{\Phi  \land  \Psi}(n) \le f^\Phi(n) \cdot f^\Psi(n).
    \]
\end{lemma}

\begin{proof}

    We first prove the inequality for the union, $f^{\Phi \lor \Psi}(n) \le f^\Phi(n) \cdot f^\Psi(n)$. The proof of the second inequality follow from the first one and the De Morgan's law.

    Let $n \in \mathbb{N}$ and choose an arbitrary sequence of $n$ points, $x_1, \dots, x_n \in X$. Let
    \[
        \mathcal{A}_\Phi = \mathcal{B}(\Phi_{x_1}, \dots, \Phi_{x_n}) \text{ and } \mathcal{A}_\Psi = \mathcal{B}(\Psi_{x_1}, \dots, \Psi_{x_n})
    \]
    be the Boolean algebras generated by the corresponding fibers. We denote the set of atoms of these algebras by $\text{atoms}(\mathcal{A}_\Phi)$ and $\text{atoms}(\mathcal{A}_\Psi)$, each form a partition of $Y$. Both of these sets of atoms form partition of $Y$ and are disjoint by definition. Also note that, by the definition of the growth function, we have:
    \[
        |\text{atoms}(\mathcal{A}_\Phi)| \le f^\Phi(n) \quad \text{and} \quad |\text{atoms}(\mathcal{A}_\Psi)| \le f^\Psi(n).
    \]
    Consider the collection
    \[
        Z \coloneq \{A \cap B \mid A \in \text{atom}(\mathcal{A}_\Phi) \land B \in \text{atom}(\mathcal{A}_\Psi) \land A \cap B \neq \emptyset \}.
    \]
    $Z$ forms a refined partition of $Y$, since every point $y \in Y$ belongs to exactly one such intersection. The number of elements in this refined partition is bounded above
    \[
        |Z| \leq |\text{atoms}(\mathcal{A}_\Phi)| \cdot |\text{atoms}(\mathcal{A}_\Psi)|.
    \]

    Now, let $\mathcal{A}_{\Phi \lor \Psi} = \mathcal{B}((\Phi \lor \Psi)_{x_1}, \dots, (\Phi \lor \Psi)_{x_n})$. The proof is complete if show that the number of atoms of $\mathcal{A}_{\Phi \lor \Psi}$ is bounded above by $|Z|$. Since $\text{atom}(\mathcal{A}_{\Phi \lor \Psi})$ also paritions $Y$, every element of $Z$ must have non-empty intersection with an element of $\text{atom}(\mathcal{A}_{\Phi \lor \Psi})$. Hence, if we show that each element of $Z$ is contained entirely in an atom of $\mathcal{A}_{\Phi \lor \Psi}$ we get the desired the desired upper bound for $|\text{atom}(\mathcal{A}_{\Phi \lor \Psi})|$.

    Let $S \coloneq A \cap B$ be an element of $Z$, where $A \in \text{atoms}(\mathcal{A}_\Phi)$ and $B \in \text{atoms}(\mathcal{A}_\Psi)$. Take any two points $y_1, y_2 \in S$. Then
    \[
        y_1, y_2 \in A \quad \Rightarrow \quad \forall i \in \{1, \dots, n\}, \quad (y_1 \in \Phi_{x_i} \iff y_2 \in \Phi_{x_i}).
    \]
    Similarly,
    \[
        y_1, y_2 \in B \quad \Rightarrow \quad \forall i \in \{1, \dots, n\}, \quad (y_1 \in \Psi_{x_i} \iff y_2 \in \Psi_{x_i}).
    \]
    Then combining these two implications,
    \[
        y_1, y_2 \in A \cap B \quad \Rightarrow \forall i \in \{1, \dots, n\}, (y_1 \in \Phi_{x_i} \lor y_1 \in \Psi_{x_i} \iff y_2 \in \Phi_{x_i} \lor y_2 \in \Psi_{x_i}).
    \]
    It follows that for each $i$, ($y_1 \in \Phi_{x_i} \cup \Psi_{x_i} \iff y_2 \in \Phi_{x_i} \cup \Psi_{x_i}$). This means that $y_1$ and $y_2$ belong to the same atom of $\mathcal{A}_{\Phi \lor \Psi}$, since $\Phi_{x_i} \cup \Psi_{x_i} = (\Phi \lor \Psi)_{x_i}$. As this holds for any pair of points in $S$, the entire set $S$ must be contained within a single atom of $\mathcal{A}_{\Phi \lor \Psi}$.

    This implies that, by pigeonhole principle the partition of $Y$ formed by $\text{atoms}(\mathcal{A}_{\Phi \lor \Psi})$ consists of fewer elements than the partition formed by $Z$. Therefore, the number of atoms is bounded:
    \[
        |\text{atoms}(\mathcal{A}_{\Phi \lor \Psi})| \leq |Z| \leq |\text{atoms}(\mathcal{A}_\Phi)| \cdot |\text{atoms}(\mathcal{A}_\Psi)| \le f^\Phi(n) \cdot f^\Psi(n).
    \]
    Since this inequality holds for any arbitrary choice of $x_1, \dots, x_n$, it must also hold for the maximum value, which gives $f^{\Phi \lor \Psi}(n) \le f^\Phi(n) \cdot f^\Psi(n)$.

    The second inequality, for $\Phi  \land  \Psi$, follows directly from the first inequality and De Morgan's laws:
    \[
        f^{\Phi \land  \Psi}(n) = f^{\neg(\neg\Phi \lor \neg\Psi)}(n) = f^{\neg\Phi \lor \neg\Psi}(n) \le f^{\neg\Phi}(n) \cdot f^{\neg\Psi}(n) = f^\Phi(n) \cdot f^\Psi(n).
    \]
\end{proof}

\begin{lemma}{\label{lem:dependent-union-intersection}}
    If $\Phi$ and $\Psi$ are dependent relations, then their union $\Phi \lor \Psi$ and intersection $\Phi \land  \Psi$ are also dependent.
\end{lemma}

\begin{proof}

    If $\Phi$ is dependent, its growth function $f^\Phi(n)$ is bounded by a polynomial in $n$ for all sufficiently large $n$. That is, there exists a polynomial $P(n)$ and an integer $N_\Phi$ such that $f^\Phi(n) \le P(n)$ for all $n \ge N_\Phi$. Similarly, if $\Psi$ is dependent, there exists a polynomial $Q(n)$ and an integer $N_\Psi$ such that $f^\Psi(n) \le Q(n)$ for all $n \ge N_\Psi$.

    From the previous lemma, we know that $f^{\Phi \lor \Psi}(n) \le f^\Phi(n) \cdot f^\Psi(n)$. For $n \ge \max(N_\Phi, N_\Psi)$, this implies:
    \[
        f^{\Phi \lor \Psi}(n) \le P(n) \cdot Q(n).
    \]
    The product of two polynomials is another polynomial. Since any polynomial function of $n$ grows slower than the exponential function $2^n$, there must exist an integer $d$ such that for all sufficiently large $n$, $f^{\Phi \lor \Psi}(n) < 2^n$. This is the definition of a dependent relation.

    The exact same argument holds for the relation $\Phi \land \Psi$, since its growth function is also bounded by the polynomial $P(n) \cdot Q(n)$. Therefore, dependence is preserved under these operations.
\end{proof}


\subsection{Propagation of Dependence Across Dimensions}

The central aim of this subsection is to establish our main theorem, which asserts a powerful inheritance property for dependence within a model-theoretic structure. We will prove that if all definable relations linking a $p$-dimensional set to a 1-dimensional set are dependent, then this property extends to relations between a $p$-dimensional set and any $q$-dimensional set. This result is a cornerstone of the argument, as it provides the inductive mechanism to generalize from a simple case to arbitrary dimensions. To achieve this, we will employ combinatorial methods rooted in Ramsey's Theorem. The core strategy involves using this theorem to extract highly uniform sequences, known as indiscernible sequences, which allow us to control the behavior of the definable relations in question.

\begin{theorem}{\label{thm:main-theorem-ch-2}}
    Let $\mathcal{R} = (R, \dots)$ be an infinite model-theoretic structure and suppose all definable relations $\Phi \subseteq R^{p+1}$, for all $p > 0$, are dependent. Then all definable relations $\Phi \subseteq R^{p+q}$, for all $p, q > 0$, are dependent.
\end{theorem}

To establish this we will use combinatorial techniques, so we need to introduce some notation. Let $X$ be a set and $r \in \mathbb{N}$ be given. We put
\[
    X^{(r)} \coloneq \text{ the collection of all $r$-element subsets of $X$}
\]

\subsubsection{Indiscernibility}

To establish this powerful reduction, we must first introduce a fundamental result from combinatorics: Ramsey's Theorem.

\begin{theorem}[Ramsey's Theorem]
    \label{thm:ramsey}
    Given positive integers $M, r, k$, there exists a positive integer $N = N(M, r, k)$ such that if $X$ is a set with
    \[
        |X| \ge N \text{ and } X^{(r)} = P_1 \cup P_2 \cup \dots \cup P_k,
    \]
    then there exists a subset $Y \subseteq X$ with $|Y| = M$ that satisfies $Y^{(r)} \subseteq P_j$ for some $j \in \{1, \dots, k\}$.
\end{theorem}



\begin{proof}

    It is sufficient to prove the theorem for $k=2$, as we only care about one of the partitions in the general case where $k \geq 2$. The proof will be an induction on $r$.

    The base case $r=1$ is a direct application of the pigeonhole principle. Take $N = 2M-1$. For any set $X$ with $|X| \geq N$ and partition $X^{(1)} = X = P_1 \cup P_2$. By the pigeonhole principle, one of $P_1$ or $P_2$ contains at least $M$ elements. Let $Y$ be a set of $M$ elements from the larger part. Then we can just pick any of the partitions with at least $M$ elements and choose a subset $Y \subseteq P_j$ for $j \in \{1, 2\}$ trivially.

    Assume, the theorem holds for a fixed $r$, so we have $N = N(M, r, 2)$ such that for any set $X'$ with $|X'| \geq N$ and $X'^{(r)} = P'_1 \cup P'_2$, we can find a subset $Y' \subseteq X'$ with $|Y'| = M$ satisfying $Y'^{(r)} \subseteq P'_1$ or $Y^{(r)} \subseteq P'_2$.

    To show the theorem holds for $r+1$, we define a sequence of integers recursively
    \[
        N_{2M} = 1 \text{ and } N_{i} \coloneq N(N_{i+1}, r, 2) + 1 \text{ for } 1 \leq i \leq M.
    \]
    Set $N = N_{1} = N(M, r+1, 2)$. Let $X$ be a set with $|X| \geq N_{1}$ and let $X^{(r+1)} = P_1 \cup P_2$ be a partition. We will inductively construct a descending sequence of subsets and elements
    \[
        X \supset A_1 \supset A_2 \supset \cdots \supset A_{2M}
    \]
    and distinct points $a_1,\dots,a_{2M}$ with the property that for $1 \le i < 2M$:
    \begin{enumerate}
        \item $a_i \in A_i$ and $a_i \notin A_{i+1}$,
        \item $|A_i| \ge N_i$,
        \item all $(r+1)$-sets in $A_i \setminus\{a_i\}$ containing $a_i$ are all in $P_1$ or all in $P_2$.{\label{enum-item:clause-4}}
    \end{enumerate}


    \textit{Construction of $A_2$ from $A_1$: }Set $A_1=X$ and choose $a_1\in A_1$ arbitrarily. Partition $(A_1\setminus\{a_1\})^{(r)}$ into
    \[
        \begin{aligned}
            Q_1 &\coloneq \{R\in (A_1\setminus\{a_1\})^{(r)}:\ \{a_1\}\cup R\in P_1\},\\
            Q_2 &\coloneq \{R\in (A_1\setminus\{a_1\})^{(r)}:\ \{a_1\}\cup R\in P_2\}.
        \end{aligned}
    \]
    Since $|A_1\setminus\{a_1\}|=|A_1|-1\ge N_1-1\ge N(N_2,r,2)$, we can apply the induction hypothesis for $r$ to $A_1\setminus\{a_1\}$ with $(A_1\setminus\{a_1\})^{(r)}=Q_1\cup Q_2$ to obtain a subset $A_2 \subset A_1$
    \[
        A_2\subseteq A_1\setminus\{a_1\} \subseteq A_1 \quad\text{with}\quad |A_2|\ge N_2
    \]
    such that either $A_2^{(r)}\subseteq Q_1$ or $A_2^{(r)}\subseteq Q_2$. Hence every $(r+1)$-subset of $A_2\cup\{a_1\}$ containing $a_1$ lies in the same part $P_j$ ($j\in\{1,2\}$). Then choose any $a_2\in A_2$ and continue construction.

    \textit{Construction of $A_{i+1}$ from $A_i$: }Now suppose $A_i$ and $a_i$ have been chosen with $|A_i|\ge N_i$ and clause~\ref{enum-item:clause-4} holds for $a_i$.
    \[
        \begin{aligned}
            Q_1&\coloneqq\{R\in (A_i\setminus\{a_i\})^{(r)}:\ \{a_i\}\cup R\in P_1\},\\
            Q_2&\coloneqq\{R\in (A_i\setminus\{a_i\})^{(r)}:\ \{a_i\}\cup R\in P_2\}.
        \end{aligned}
    \]
    Since $|A_i\setminus\{a_i\}|=|A_i|-1\ge N_i-1\ge N(N_{i+1},r,2)$, the induction hypothesis for $r$ gives a subset
    \[
        A_{i+1}\subseteq A_i\setminus\{a_i\} \quad\text{with}\quad |A_{i+1}|\ge N_{i+1}
    \]
    and $A_{i+1}^{(r)}\subseteq Q_1$ or $A_{i+1}^{(r)}\subseteq Q_2$. In particular, every $(r+1)$-subset of $A_{i+1}\cup\{a_i\}$ containing $a_i$ lies entirely in one of $P_1,P_2$. Choose any $a_{i+1}\in A_{i+1}$.

    This yields distinct points $a_1,\dots,a_{2M}$. For each $i$ let $j(i)\in\{1,2\}$ be the index guaranteed by clause~\ref{enum-item:clause-4}. Define
    \[
        Y_1\coloneqq\{a_i:\ j(i)=1\},\qquad Y_2\coloneqq\{a_i:\ j(i)=2\}.
    \]
    Then $|Y_1|+|Y_2|=2M$, so one of $Y_1,Y_2$ has size at least $M$; call that set $Y$ and let $j$ be its associated index.

    We claim $Y^{(r+1)}\subseteq P_j$. Take any $S\in Y^{(r+1)}$ and let $a_i$ be the unique element of $S$ with smallest index. The remaining $r$ elements of $S$ lie in $A_{i+1}\subseteq A_i\setminus\{a_i\}$, so $S=\{a_i\}\cup R$ with $R\in (A_i\setminus\{a_i\})^{(r)}$. By clause~\ref{enum-item:clause-4} for $a_i$, we have $S\in P_j$. Hence $Y^{(r+1)}\subseteq P_j$ and  $|Y|=M$ can be obtained by discarding extra elements if necessary, as required.

\end{proof}

Let $X$ be an infinite set.
If $A \subseteq X^r$ is an $r$-ary relation, a finite sequence $x_1, \dots, x_M$ of elements of $X$ is called \emph{$A$-indiscernible} if for all increasing $r$-tuples of indices
\[
    1 \le i(1) < \cdots < i(r) \le M, \quad 1 \le j(1) < \cdots < j(r) \le M,
\]
we have
\[
    \bigl(x_{i(1)}, \dots, x_{i(r)}\bigr) \in A
    \quad\Longleftrightarrow\quad
    \bigl(x_{j(1)}, \dots, x_{j(r)}\bigr) \in A.
\]
In other words, whether a given $r$-tuple from the sequence lies in $A$ depends only on the positions chosen, not on which specific elements occupy those positions.

If $\mathcal{A}$ is a finite family of relations on $X$, each $A \in \mathcal{A}$ having its own arity $r(A)$, we say that $x_1, \dots, x_M$ is \emph{$\mathcal{A}$-indiscernible} if it is $A$-indiscernible for every $A \in \mathcal{A}$.

\begin{remarknl}
    Indiscernibility means that the truth values of all relations in $\mathcal{A}$ on subtuples of the sequence are constant across all choices of indices of the same length.
    For example, if $A$ is binary, $A$-indiscernibility says: for any two pairs $(x_i, x_j)$ and $(x_{i'}, x_{j'})$ with $i<j$ and $i'<j'$, either both pairs are in $A$ or both are not.
\end{remarknl}

\begin{example}
    Let $X = \mathbb{Z}$ and $A \subseteq X^2$ be the binary relation
    \[
        A(m, n) \quad\text{means}\quad m < n \ \text{ and } \ m+n \ \text{is even}.
    \]
    Consider the sequence
    \[
        x_1 = 2,\quad x_2 = 4,\quad x_3 = 6,\quad x_4 = 8.
    \]
    For any $i < j$, $x_i + x_j$ is even, so $(x_i, x_j) \in A$.
    This holds for \emph{every} pair of indices in the sequence, so it is $A$-indiscernible: the truth value of $A(x_i, x_j)$ is constant (true) for all index pairs.

    Similarly, the sequence $1, 3, 5, 7$ is $A$-indiscernible since all sums are even again.
    In contrast, the sequence $2, 3, 4, 5$ is \emph{not} $A$-indiscernible, because some pairs have even sum and others have odd sum.
\end{example}

\begin{corollary}{\label{cor:ramsey-cor}}
    Let $X$ be infinite and $\mathcal{A}$ a finite family of relations on $X$. For every $M \in \mathbb{N}$ there exists $N \in \mathbb{N}$ such that every sequence in $X$ of length $N$ contains an $\mathcal{A}$-indiscernible subsequence of length $M$.
\end{corollary}

\begin{proof}

    It's enough to show this holds for $|\mathcal{A}| = 1$, since for $|\mathcal{A}| > 1$ we can find indiscernible sequence for one of the relations $A \in \mathcal{A}$ and repeatedly apply Ramsey's theorem~\ref{thm:ramsey} for remaining relations in $\mathcal{A}$ one by one to obtain subsequence of the $A$-indiscernible sequence.

    Suppose $\mathcal{A} = \{A\}$ with $A \subseteq X^r$ for some $r \in \mathbb{N}$ and let $x_1, \dots, x_N$ be a sequence in $X$ where $N = N(M, r, 2)$ and write $Z \coloneq\{1, \dots, N\}$. Consider partition of $Z^{(r)} = P_1 \cup P_2$ given by
    \[
        \begin{aligned}
            P_1 &\coloneq \left\{ \{i_1, \dots, i_r\} \mid 1 \leq i_1 \leq \dots \leq i_r \leq N \text{ and } (x_{i_1}, \dots, x_{i_r}) \in A\right\} \\
            P_2 &\coloneq \left\{ \{i_1, \dots, i_r\} \mid 1 \leq i_1 \leq \dots \leq i_r \leq N \text{ and } (x_{i_1}, \dots, x_{i_r}) \notin A\right\} \\
        \end{aligned}
    \]
    Hence, by Ramsey's theorem~\ref{thm:ramsey} there is a subset
    \[
        \{i_1, \dots, i_M\} \subseteq Z \text{ with } i_1 < \dots < i_M,
    \]
    such that $\{i_1, \dots, i_M\}^{(r)} \subseteq P_1$ or $P_2$. Then, the sequence $x_{i_1}, \dots, x_{i_M}$ is $A$-indiscernible.
\end{proof}


\begin{examplenl}
    Let $X = \mathbb{R}$ and let $\mathcal{A} = \{A_1, A_2\}$ where $A_1(x, y)$ means $x < y$ and $A_2(x, y, z)$ means $x + y > z$.
    Then $\mathcal{A}$-indiscernibility for a sequence $x_1, \dots, x_M$ means:
    \begin{itemize}
        \item all ordered pairs $(x_i, x_j)$ with $i < j$ satisfy $A_1$ or none do, and
        \item all triples $(x_i, x_j, x_k)$ with $i < j < k$ satisfy $A_2$ or none do.
    \end{itemize}
    Corollary~\ref{cor:ramsey-cor} says that if the starting sequence is long enough, we can always extract a subsequence where these two uniformity conditions hold simultaneously.
\end{examplenl}

For the rest of this section we fix infinite sets $X$ and $Y$ and a binary relation $\Phi \subseteq X \times Y$.

\begin{lemma}{\label{lem:indiscernible-seq-1}}
    Suppose $\Phi$ is independent and $\mathcal{A}$ is a finite collection of relations on $X$. Then, for each $M \in \mathbb{N}$ there are $\mathcal{A}$-indiscernible sequence
    \[
        a_1, \dots, a_M \in X \text{ and } b \in Y,
    \]
    such that for all $m \in \{1, \dots, M\}$, we have $(a_m, b) \in \Phi \iff m \text{ is even.}$
\end{lemma}

\begin{proof}

    Let $M \in \mathbb{N}$ be given. By Corollary~\ref{cor:ramsey-cor}, we can find a natural number $N$ such that each sequence $x_1, \dots, x_N \in X$ contains $\mathcal{A}$-indiscernible subsequence of length $M$.

    Recall that, by Definition~\ref{def:relation-dependence}, $\Phi \subseteq X \times Y$ is independent means $\Phi_X \subseteq \mathcal{P}(Y)$ is independent. Hence, we can find elements $x_1, \dots, x_N$ such that the sets $\Phi_{x_1}, \dots, \Phi_{x_N} \in \mathcal{P}(Y)$ are independent. Hence, for any subset $W \in \mathcal{P}(\{1, \dots, N\})$, the intersection
    \begin{equation}
        \label{eq:lemma-independence-indiscernible}
        \Bigl(\bigcap_{i \in W} \Phi_{x_i}\Bigr) \cap \Bigl( \bigcap_{i \notin W} Y \setminus \Phi_{x_i} \Bigr)
    \end{equation}
    is non-empty.

    Applying Ramsey's theorem to the sequence $x_1, \dots, x_N \in X$, we obtain $\mathcal{A}$-indiscernible subsequence $x_{i_1}, \dots, x_{i_M} \in X$, where $1 \leq i_1 \leq \dots \leq i_M \leq N$.

    Let the sequence be $a_m \coloneq x_{i_m}$ for $m \in \{1, \dots, M\}$. Now, choose the set of indices $W \subseteq \{1, \dots, N\}$ to be $W \coloneq \{i_m \mid m \in \{1, \dots, M\} \text{ is even}\}$. Since the sets $\Phi_{x_i}$ are independent, the intersection in (\ref{eq:lemma-independence-indiscernible}) for this $W$ is non-empty; let $b$ be an element of this intersection. By this choice, $b \in \Phi_{x_{i_m}}$ if and only if its index $i_m$ is in $W$, which occurs precisely when $m$ is even. This is equivalent to $(a_m, b) \in \Phi \iff m \text{ is even}$, as required.

\end{proof}

Next we introduce certain relations that we can define from $\Phi$. Let $x_1, \dots, x_M \in X$ and $y \in Y$ be given, where $M \in \mathbb{N}$. For any set $U \subseteq \{1, \dots, M\}$, we write the formula defining a subset $\Phi_{U} \subseteq X^{M}\timesY$ as below:
\begin{equation}{\label{eq:new-way-of-relation}}
    \Phi_{U}(x_1, \dots, x_M; y) \coloneq \Biggl( \bigwedge_{i \in U} \Phi(x_i, y) \Biggr) \land \Biggl( \bigwedge_{i \notin U} \neg\Phi(x_i, y)\Biggr).
\end{equation}
The image of $\Phi_{U}$ under the projection map $\pi: X^M\times Y \to X^M$ is defined by the formula
\[
    \exists y \Phi_{U}(x_1, \dots, x_M; y)
\]
We denote the collection of $M$-ary relation of this form as follows:
\[
    \mathcal{A}_{\Phi, M} \coloneq \{\exists y \Phi_{U}(x_1, \dots, x_M; y) \mid U \subseteq \{1, \dots, M\} \}
\]

Following lemma is the converse of~\ref{lem:indiscernible-seq-1}.

\begin{lemma}{\label{lem:indiscernible-seq-2}}
    Let $a_1, \dots, a_N \in X$ be an $\mathcal{A}_{\Phi, M}$-indiscernible sequence where, $N \geq 2M$. Suppose, $a_{i_1}, \dots, a_{i_{2M}}$ is a subsequence and $b \in Y$ such that
    \begin{equation}{\label{eq:lem-indiscerible-seq-2-asumption}}
        \forall m \in \{1, \dots, 2M\}, (a_{i_m}, b) \in \Phi \iff m \text{ is even}.
    \end{equation}
    Then, $D(\Phi) > M$.
\end{lemma}

\begin{proof}

    Recall, that $D(\Phi) > M$ means $f_{\Phi_X}(M) = 2^M$, so it's sufficient to show that $\Phi_{a_1}, \dots, \Phi_{a_M} \subseteq Y$ are independent. Write $E \subseteq \{1, \dots, 2M\}$ for the subset of even integers, then by assumption following holds:
    \begin{equation}{\label{eq:indiscernible-lemma-2-eq-1}}
        \bigwedge_{i \in E}\Phi(a_i, b) = \Phi(a_2, b) \land \dots \land \Phi(a_{2M}, b)
    \end{equation}
    Given any subset $U \subseteq \{1, \dots, M\}$, take a sequence $k_1, \dots, k_M \in \{1, \dots, 2M\}$ such that
    \begin{equation}{\label{eq:indiscernible-lemma-2-eq-2}}
        1 \leq k_1 \leq \dots \leq k_M \leq 2M, \quad \begin{cases}
                                                          \text{ for } i \in U, k_i \text{ is even} \\
                                                          \text{ for } i \notin U, k_i \text{ is odd}
        \end{cases}
    \end{equation}
    Combining,~\ref{eq:indiscernible-lemma-2-eq-1} and~\ref{eq:indiscernible-lemma-2-eq-2} we get that
    \[
        \Phi_{U}(a_{i_{k_1}}, \dots,a_{i_{k_M}}; b) = \Bigl(\bigwedge_{i \in U} \Phi(a_{k_i}; b) \Bigr) \land \Bigl(\bigwedge_{i \notin U} \neg \Phi(a_{k_i}; b)\Bigr)
    \]
    holds, which implies $\exists y \Phi_{U}(a_{i_{k_1}}, \dots,a_{i_{k_M}}; y)$ holds. Hence, by $\mathcal{A}_{\Phi, M}$-indiscerniblity of  $a_1, \dots, a_N \in X$
    \[
        \exists y \Phi_{U}(a_{1}, \dots,a_{M}; y)
    \]
    holds. Since, $U \subseteq \{1, \dots, M\}$ was arbitrary, this shows that $\Phi_{a_1}, \dots, \Phi_{a_M} \subseteq Y$ is independent sequence.
\end{proof}

\begin{remark}
    Informally speaking, the Lemma~\ref{lem:indiscernible-seq-2} provides a sufficient condition for the boolean algebra $\mathcal{B}(\Phi_{a_1}, \dots, \Phi_{a_M})$ to have $2^M$ atoms. To observe this, note that, $\mathcal{A}_{\Phi, M}$-indiscernibility of $a_1, \dots, a_N$ is not strong enough condition for us to conclude the desired result, as there is a possibility that for any subsequence $a_{i_1}, \dots, a_{i_M}$ and any relation $A \in \mathcal{A}_{\Phi, M}$, $A(a_{i_1}, \dots, a_{i_M})$ does not hold. Hence, we need extra condition~\ref{eq:lem-indiscerible-seq-2-asumption} for ensuring that any subsequence of size $M$ satisfies any relation in $\mathcal{A}_{\Phi, M}$. Once, this is established we use $\mathcal{A}_{\Phi, M}$-indiscernibility to show $\mathcal{B}(\Phi_{a_1}, \dots, \Phi_{a_M})$ has $2^M$ atoms. Choice of $a_1, \dots, a_M$ here was arbitrary, since we could also argue that  $\mathcal{B}(\Phi_{a_{M+1}}, \dots, \Phi_{a_{2M}})$ also has $2^M$ atoms.
\end{remark}

Next, we assume that $Y = Y_1 \times Y_2$, where both $Y_1, Y_2$ are infinite. Write $y_1 \in Y_1$ and $y_2 \in Y_2$ for arbitrary elements and consider the formula $\Phi(x; y_1, y_2)$ and corresponding relation $\Phi$

From now on assume that the parameter set splits as a Cartesian product
\[
    Y \;=\; Y_1\times Y_2,
\]
with both $Y_1$ and $Y_2$ infinite. We regard the original binary relation
\[
    \Phi \;\subseteq\; X\times Y \;=\; X\times (Y_1\times Y_2)
\]
as a ternary relation and we write it as a formula $\Phi(x;y_1,y_2)$ with
$x\in X$, $y_1\in Y_1$, and $y_2\in Y_2$. It is convenient to ``re-index’’
$\Phi$ by grouping $(x,y_1)$ together and keeping $y_2$ separate:
\[
    \Phi^\ast \;\subseteq\; (X\times Y_1)\times Y_2,
    \qquad
    \Phi^\ast\bigl((x,y_1);y_2\bigr) \iff \Phi(x;y_1,y_2).
\]
In words, $\Phi^\ast$ parametrizes a family of subsets of $Y_2$ indexed by
$X\times Y_1$:
\[
    \bigl(\Phi^\ast\bigr)_{(x,y_1)}
    \;=\; \{\,y_2\in Y_2 : \Phi(x;y_1,y_2)\,\}
    \;\subseteq\; Y_2.
\]
Thus the dependence/independence of $\Phi^\ast$ is the usual VC/NIP property
for a family of subsets of $Y_2$ with index set $X\times Y_1$. In particular,
it makes sense to assert that $\Phi^\ast$ is dependent (equivalently, that the
family $\{(\Phi^\ast)_{(x,y_1)} : (x,y_1)\in X\times Y_1\}$ has finite VC-dimension).


Fix $M\in\mathbb{N}$ and a subset $U\subseteq \{1,\dots,M\}$. For variables
$x_1,\dots,x_M\in X$ and $y_1\in Y_1$ we define
\[
    \Phi_{U}(x_1,\dots,x_M; y_1,y_2)\coloneq\Bigl(\bigwedge_{i\in u}\Phi(x_i;y_1,y_2)\Bigr)\ \land\
    \Bigl(\bigwedge_{i\notin u}\neg\Phi(x_i;y_1,y_2)\Bigr).
\]
We now existentially eliminate $y_2$ and obtain a relation on $X^M\times Y_1$:
\[
    \Gamma_{\Phi,U}(x_1,\dots,x_M; y_1) \coloneq \exists y_2\, \Phi_{U}(x_1,\dots,x_M; y_1,y_2).
\]
The subset $\Gamma_{\Phi,U}$ parametrises a collection of subsets of $Y_1$ with index set $X^M$, hence we can talk about the dependence of $\Gamma_{\Phi,U}$.

\begin{theorem}{\label{thm:main-theorem-2}}
    Suppose there are positive integers $M, N \in \mathbb{N}$ such that
    \[
        D(\Phi^{*}) \leq M \text{ and } D(\Gamma_{\Phi, U}) \leq N
    \]
    for all $U \subseteq \{1, \dots, M\}$. Then $\Phi$ is dependent.
\end{theorem}

\begin{proof}

    Let $U \subseteq \{1, \dots, M\}$ and $V \subseteq \{1, \dots, N\}$ be subsets and $\Psi_{U, V} \subseteq X^{MN}$ be a subset defined by the formula
    \[
        \Psi_{U, V}(\overline{x}_1, \dots, \overline{x}_N) \coloneq \exists y_1 \Biggl(\Bigl(\bigwedge_{j \in V} \Gamma_{\Phi, U} (\overline{x}_j, y_1)\Bigr) \land \Bigl(\bigwedge_{j \notin V} \Gamma_{\Phi, U} (\overline{x}_j, y_1)\Bigr) \Biggr)
    \]
    where $\overline{x}_j = (x_1, \dots, x_M) \in X^M$ for $j \in \{1, \dots, N\}$. Note that, with the notation defined earlier in~\ref{eq:new-way-of-relation}, $\Psi_{U, V}$ can be expressed as
    \[
        \Psi_{U, V}(\overline{x}_1, \dots, \overline{x}_N) = \exists y_1 (\Gamma_{\Phi, U})_{V}(\overline{x}_1, \dots, \overline{x}_N, y_1).
    \]
    Write $\mathcal{A}_{U} \coloneq \{\Psi_{U, V} \mid V \subseteq \{1, \dots, N\}\}$ and
    \[
        \mathcal{A} \coloneq \{\Psi_{U, V} \mid U \subseteq \{1, \dots, M\}, V \subseteq \{1, \dots, N\}\}.
    \]
    Hence,
    \[
        \mathcal{A} = \bigcup_{U \subseteq \{1, \dots, M\}} \mathcal{A}_U.
    \]
    We will use argument by contradiction to show $\Phi$ is dependent. Assume $\Phi$ is independent. Write
    \[
        K \coloneq (2N)^{2^M} \cdot 2M
    \]
    Since $\Phi$ is independent, by assumption, and $\mathcal{A}$ is finite collection of relations on $X$, we can apply~\ref{lem:indiscernible-seq-1} to get $\mathcal{A}$-indiscernible sequence $a_1, \dots, a_K \in X$ and an element $b = (b_1, b_2) \in Y = Y_1 \times Y_2$ such that
    \[
        (a_k, b) \in \Phi \iff k \text{ is even.}
    \]
    Note that, each relation in $\mathcal{A}_{\Phi^{*}, M}$ is of the form
    \[
        A^*_{U} \coloneq \exists y_2 \Phi^{*}_{U}\bigl((x_1, y_{1, 1}), \dots,(x_M, y_{1, M}), y_2 \bigr)
    \]
    where $U \subseteq \{1, \dots, M\}$ and $(x_i, y_{1, i}) \in X \times Y_1$ for $i \in \{1, \dots, M\}$.
    Since, $D(\Phi^{*}) \leq M$, we can apply the contrapositive of~\ref{lem:indiscernible-seq-2} to derive that there is no sequence $(a_{i_1}, b_1), \dots, (a_{i_{2M}}, b_1) \in X \times Y_1$ that is $\mathcal{A}_{\Phi^{*}, M}$-indiscernible.

    Note that,
    \[
        \begin{aligned}
            &\begin{array}{c}
            (a_1, b_1)
                 , \dots, (a_{2M}, b_1) \\
                 \text{is $A^*_{U}$-indiscernible}
            \end{array}
            \iff
            \begin{array}{c}
                a_1, \dots, a_{2M} \\
                \text{is $\Gamma_{\Phi,U}(x_1, \dots, x_M, b_1)$-indiscernible}
            \end{array}
        \end{aligned}
    \]
    Hence, there is no sequence $a_1, \dots, a_{2M}$ that is $\Gamma_{\Phi,U}(x_1, \dots, x_M, b)$-indiscernible for all $U \subseteq \{1, \dots, M\}$. We will show that such a sequence must exist, using the following claim, leading to a contradiction.



    \textbf{Claim:} Let $P, Q \in \mathbb{N}$ with $Q \geq 2NP$. Let $I \subseteq \{1, \dots, K\}$ be an interval of length $Q$. For any given $U \subseteq \{1, \dots, M\}$, there exists a subinterval $J \subseteq I$ of length $P$ such that the sequence $(a_k)_{k \in J}$ is $\Gamma_{\Phi,U}(x_1, \dots, x_M ; b_1)$-indiscernible.

    \begin{subproof}[Proof of the Claim]

        Let $I = \{k \mid i_0 \leq k < i_0 + Q\}$. We prove the claim by contradiction. Assume no such subinterval $J$ exists.

        We partition $I$ into $2N$ disjoint consecutive subintervals $J(j)$ for $j \in \{0, \dots, 2N-1\}$, each of length $P$:
        \[
            J(j) \coloneq \{k \mid i_0 + jP \leq k < i_0 + (j+1)P\}.
        \]
        By our assumption, for each $j$, the sequence $(a_k)_{k \in J(j)}$ is \emph{not} $\Gamma_{\Phi,U}(x_1, \dots, x_M ; b_1)$-indiscernible. This means that for each $j$, there must exist at least two strictly increasing sequences of indices of length $M$ within $J(j)$ that behave differently under $\Gamma_{\Phi,U}$. That is, there exist subsequences of $(a_k)_{k \in J(j)}$ of length $M$ for which $\Gamma_{\Phi,U}$ holds, and others for which it fails.

        We can therefore choose, for each $j \in \{0, \dots, 2N-1\}$, a strictly increasing sequence of indices $k(j,1) < \dots < k(j,M)$ from $J(j)$ as follows:
        \begin{itemize}
            \item If $j$ is even, choose the indices so that $\Gamma_{\Phi,U}(a_{k(j,1)}, \dots, a_{k(j,M)}; b_1)$ is \textbf{true}.
            \item If $j$ is odd, choose the indices so that $\Gamma_{\Phi,U}(a_{k(j,1)}, \dots, a_{k(j,M)}; b_1)$ is \textbf{false}.
        \end{itemize}
        Let $a_j^* \coloneq (a_{k(j,1)}, \dots, a_{k(j,M)}) \in X^M$. We have constructed a new sequence of $M$-tuples $a_0^*, \dots, a_{2N-1}^*$ such that $\Gamma_{\Phi,U}(a_j^*; b_1)$ holds if and only if $j$ is even.

        Since the original sequence $(a_k)_{k \in \{1,\dots,K\}}$ is $\mathcal{A}$-indiscernible, and each $a_j^*$ is formed by taking elements from $(a_k)$ with strictly increasing indices, the resulting sequence of tuples $(a_j^*)_{0 \le j < 2N}$ is $\mathcal{A}_U$-indiscernible. But we have just constructed it to have an alternating property with respect to $\Gamma_{\Phi,U}$. By Lemma~\ref{lem:indiscernible-seq-2}, this implies that the relation $\Gamma_{\Phi,U}$ must be independent with $D(\Gamma_{\Phi,U}) > N$.

        This contradicts our initial hypothesis that $D(\Gamma_{\Phi,U}) \leq N$. Thus, our assumption was false, and the claim must hold.
    \end{subproof}

    We now use the claim to derive the final contradiction. Let $\{U_1, \dots, U_s\}$ be an enumeration of all $s=2^M$ subsets of $\{1, \dots, M\}$.

    Let $I_0 \coloneq \{1, \dots, K\}$. The length of $I_0$ is $Q_0 = K = (2N)^{2^M} \cdot 2M$.

    \textbf{Step 1:} Apply the claim to the interval $I_0$ and the set $U_1$. Let $P_1 = Q_0 / (2N) = (2N)^{s-1} \cdot 2M$. The claim guarantees the existence of a subinterval $I_1 \subseteq I_0$ of length $P_1$ such that $(a_k)_{k \in I_1}$ is $\Gamma_{\Phi,U_1}(\overline{x}; b_1)$-indiscernible.

    \textbf{Step 2:} Apply the claim to the interval $I_1$ (with length $Q_1 = P_1$) and the set $U_2$. Let $P_2 = Q_1 / (2N) = (2N)^{s-2} \cdot 2M$. The claim gives a subinterval $I_2 \subseteq I_1$ of length $P_2$ where $(a_k)_{k \in I_2}$ is $\Gamma_{\Phi,U_2}(\overline{x}; b_1)$-indiscernible. Since $I_2 \subseteq I_1$, the sequence $(a_k)_{k \in I_2}$ remains $\Gamma_{\Phi,U_1}(\overline{x}; b_1)$-indiscernible.

    We repeat this process $s = 2^M$ times. At the final step $s$, we obtain an interval $I_s \subseteq I_{s-1} \subseteq \dots \subseteq I_0$. The length of this interval is
    \[
        |I_s| = \frac{K}{(2N)^s} = \frac{(2N)^{2^M} \cdot 2M}{(2N)^{2^M}} = 2M.
    \]
    By construction, the sequence $(a_k)_{k \in I_s}$ is $\Gamma_{\Phi,U_i}(\overline{x}; b_1)$-indiscernible for all $i \in \{1, \dots, s\}$. In other words, we have found a subsequence of $(a_k)_{k \in [K]}$ of length $2M$ that is simultaneously $\Gamma_{\Phi,U}(\overline{x}; b_1)$-indiscernible for all $U \subseteq \{1, \dots, M\}$.

    This is the exact condition that we showed was impossible before the claim, as it contradicts the hypothesis $D(\Phi^*) \leq M$.

    Our initial assumption that $\Phi$ is independent has led to a contradiction. Therefore, $\Phi$ must be dependent.


\end{proof}

\begin{remark}
    It is worth noting the structure of this proof. The argument establishes the dependence of a relation $\Phi \subseteq X \times (Y_1 \times Y_2)$ by relying solely on the assumed dependence of the auxiliary relations $\Phi^* \subseteq (X \times Y_1) \times Y_2$ and $\Gamma_{\Phi,U} \subseteq X^M \times Y_1$. These relations are constructed directly from $\Phi$ and live in simpler spaces, forming the basis of the inductive step from dimension $q$ to $q+1$. Moreover, the definitions of these auxiliary relations do not introduce new quantifiers over the domain $X$. This property is vital for ???

\end{remark}

\begin{proof}[Proof of~\ref{thm:main-theorem-ch-2}]

    The proof will be an induction on $q$ and application of~\ref{thm:main-theorem-2}. Note that, the base case of indiction, i.e $q=1$ it the assumption of this theorem. So we will assume the statement of the this theorem holds for some $q > 0$ and derive that it also holds for $q+1$.

    Let $\Phi \subseteq R^{p} \times R^{q+1}$ be definable for all $p>0$. Write
    \[
        X \coloneq R^p, \quad Y_1 \coloneq R^q, \quad Y_2 \coloneq R.
    \]
    Then $Y \coloneq Y_1 \times Y_2$ and $\Phi \subseteq X \times Y$. By the definability of $\Phi$, we can deduce that
    \[
        \Phi^{*} \subseteq (X \times Y_1) \times Y_2 = R^{p+q} \times R
    \]
    is also definable. Applying the hypothesis of this theorem to $\Phi^{*} \subseteq R^{p+q} \times R$, we deduce that $\Phi^{*}$ is dependent, say $D(\Phi^{*}) \leq M$ for some $M \in \mathbb{N}$. Then
    \[
        \Gamma_{\Phi, U} \subseteq (X^M \times Y_1) = R^{pM} \times R^q
    \]
    is also definable and we can apply the inductive hypothesis for $q > 0$, to deduce that $D(\Gamma_{\Phi, U}) \leq N$ for some $N \in \mathbb{N}$ and for all $U \subseteq [M]$. Then, we can apply the theorem~\ref{thm:main-theorem-ch-2} to deduce that $\Phi$ is dependent as required.
\end{proof}

