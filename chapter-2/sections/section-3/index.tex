\section{NIP property of O-minimal structures}

This section is dedicated to proving one of the most significant results in model theory: that all o-minimal structures possess the NIP (Not Independence Property), which is equivalent to establishing that every definable relation is dependent. Our approach is constructive, building from a foundational base case to the general theorem. The argument is structured in three key stages.

\begin{enumerate}
    \item We begin by establishing the crucial base case for our induction. In the first subsection, we will prove that any definable relation in a space of the form $R^m \times R$ is dependent. This is achieved by leveraging the core properties of o-minimality, namely the Cell Decomposition Theorem and the well-behaved structure of definable sets in one dimension.

    \item Next, to build intuition, we will focus on the canonical example of an o-minimal structure: the field of real numbers with its semialgebraic sets. We will prove that all semialgebraic relations are dependent by connecting the problem to the finite dimensionality of vector spaces of polynomials—a classic result from VC-theory. This provides a concrete illustration of dependence in a familiar setting.

    \item Finally, we will synthesize these results to prove the main theorem of this section. Armed with the base case from the first step, we will apply our main theorem on the propagation of dependence from the previous section (Theorem~\ref{thm:main-theorem-ch-2}). This theorem provides the powerful combinatorial engine to inductively generalize the result from relations in $R^p \times R$ to arbitrary dimensions $R^p \times R^q$, thereby completing the proof and establishing the NIP property for all o-minimal structures.
\end{enumerate}

Beyond its intrinsic importance in model theory, this result provides the essential groundwork for the subsequent chapter on learnability. The NIP property is a key combinatorial condition that tames the complexity of definable sets, making them amenable to the frameworks of statistical learning theory. The conclusion that o-minimal structures have NIP will be directly leveraged to show that concept classes defined within them are PAC (Probably Approximately Correct) learnable, bridging the gap between abstract model theory and concrete machine learning guarantees.

\subsection{Dependence of relations in dimension 1}


We first consider the application of lemmas~\ref{lem:suff-cond-dependencty} and \ref{lem:dependent-boolean-comb} in O-minimal structures. We will assume the reader is familiar with axioms and properties of o-minimal structures, including the cell decomposition theorem from~\cite[Chap 1, 3]{vandenDries1998}.



We denote an arbitrary o-minimal structure by the tuple $(R, \mathcal{S})$ where $R$ is densely linearly ordered set without endpoints and $\mathcal{S}$ is a collection of definable sets in $R$. Let $\Psi \subseteq R^{m+1}$ be a definable set and consider the definable family $\mathcal{C} = \{\Psi_x: x \in R^m\}$ of fibers of $\Psi$ over $R^m$. The primary goal of this subsection will be to prove that the collection $\mathcal{C}$ is dependent with polynomial bound on the . More formally,

\begin{lemma}{\label{lem:o-minimal-dependence-hypothesis}}
    Let $(R, \mathcal{S})$ be an o-minimal structure and let $\Psi \subseteq R^m \times R$ be a definable for any $m > 0$. Then $\mathcal{C} = \{\Psi_x: x \in R^m\} \subseteq \mathcal{P}(R)$ is dependent.
\end{lemma}

To obtain the desired result using~\ref{lem:dependent-boolean-comb} we need two components. Firstly, we need collection of subsets of $R$ that satisfies the hypothesis of the lemma~\ref{lem:suff-cond-dependencty} with some positive integer $d$. For this we will consider the collection of definably connected sets in $R$ and show that it satisfies the statement of the lemma~\ref{lem:suff-cond-dependencty} with $d=3$. Secondly, we will use cell decomposition theorem to show each set in $\mathcal{C}$ is boolean combination of at most $e \in \mathbb{N}$ definably connected sets. Following two lemmas, will give us the desired result of this subsection.

Recall that, in $R$, the definable connected sets are points and intervals. In the proof of this lemma, we consider points also as intervals, since any point $a \in R$, can be expressed as interval $[a, a] \subset R$.
\begin{lemma}
    Let $\mathcal{G}$ be a collection of definably connected sets of $R$. Then any collection of 3 sets from $\mathcal{G}$ is dependent.
\end{lemma}

\begin{proof}{\label{lem:o-minimality-application-1}}

    Let $S_1, S_2, S_3$ be three definably connected sets in $\mathcal{G}$ and assume their intersection $I = S_1 \cap S_2 \cap S_3$ is non-empty. Since the intersection of any number of intervals is itself an interval, $I$ is a non-empty interval.

    Let $l = \inf(I)$ and $r = \sup(I)$. By the definition of intersection, the endpoints of $I$ are determined by the endpoints of $S_1, S_2, S_3$ as follows:
    $$ l = \max\{\inf(B_1), \inf(B_2), \inf(B_3)\} $$
    $$ r = \min\{\sup(B_1), \sup(B_2), \sup(B_3)\} $$
    By the properties of $\max$ and $\min$, there must exist indices $i, j \in \{1, 2, 3\}$ such that:
    $$ \inf(B_i) = l \quad \text{and} \quad \sup(B_j) = r $$
    Now, consider the intersection of just these two intervals, $B_i \cap B_j$. The infimum of this intersection is $\max\{\inf(B_i), \inf(B_j)\} = \max\{l, \inf(B_j)\}$. Since $l$ is the maximum of all three infima, $l \ge \inf(B_j)$, so $\max\{l, \inf(B_j)\} = l$.

    Similarly, the supremum of $B_i \cap B_j$ is $\min\{\sup(B_i), \sup(B_j)\} = \min\{\sup(B_i), r\}$. Since $r$ is the minimum of all three suprema, $r \le \sup(B_i)$, so $\min\{\sup(B_i), r\} = r$.

    Thus, the interval $B_i \cap B_j$ has infimum $l$ and supremum $r$, which means it is identical to the original intersection $I$.
    $$ B_1 \cap B_2 \cap B_3 = I = B_i \cap B_j $$
    The set of indices $\{i, j\}$ is a subset of $\{1, 2, 3\}$ of size at most 2, and therefore a proper subset.
\end{proof}

The second part of proof utilizes a powerful consequence of the Cell Decomposition Theorem (CDT). Recall that, by CDT there is a decomposition of $R^{m+1}$ that partitions $\Psi \subseteq R^{m+1}$ into cells. The following result establishes a uniform bound on the number of cells of the fibers of a definable set.

\begin{lemma}{\label{lem:o-minimality-application-2}}
    Let $S \subseteq R^m \times R^n$ be definable. Then there is a number $e_S \in \mathbb{N}$ such that for each $a \in R^m$, the set $S_a$ has a partition into at most $e_S$ cells. In particular, each fiber $S_a$ has at most $e_S$ definably connected components.
\end{lemma}

\begin{proof}

    Consider the decomposition $\mathcal{D}$ of $R^{m+n}$ partitioning $S$. We can then consider for each $a \in R^m$ the decomposition $\mathcal{D}_a = \{C_a: C \in \mathcal{D}, a \in \pi(C) \subseteq R^m\}$ of $R^n$, where $\pi: R^{m+n} \to R^m$ is a projection onto first $m$ coordinates. Since, each $C_a$ is still a cell, $\mathcal{D}_a$ decomposes $S_a$ and has at most $|\mathcal{D}|$ cells. Hence, we can take $e_S = |\mathcal{D}|$.
\end{proof}

\begin{proof}[Proof of~\ref{lem:o-minimal-dependence-hypothesis}]

    Applying the lemma~\ref{lem:o-minimality-application-2} to the definable set $\Psi$ and its fibers, we get that each set in $\mathcal{C} = \{\Psi_x: x \in R^m\}$ is boolean combination of at most $e_{\Psi}$ definably connected sets. Hence, we can apply the Lemma~\ref{lem:dependent-boolean-comb} to $\mathcal{G}$ with $d=3$ and $\mathcal{C}$, we conclude that $\mathcal{C}$ is dependent collection.
\end{proof}

\subsection{Dependence in semialgebraic sets}

Our main aim in this chapter is to show that definable sets in o-minimal structures are dependent. We will start with a simple example, which is a special case of the main theorem of this section.

\begin{proposition}{\label{prop:vector-space-dependent}}
    Let $\mathcal{V}$ be an $m$-dimensional real vector space of real valued functions $X \to \mathbb{R}$, where $X$ is an infinite set. For each $f \in \mathcal{V}$, consider following:
    \[
        \text{pos}(f) \coloneq \{x \in X \mid f(x) > 0\}.
    \]
    Then, the collection $\text{pos}(\mathcal{V}) \coloneq \{\text{pos}(f) \mid f \in \mathcal{V}\} \subseteq \mathcal{P}(X)$ is VC-class with VC-index $m+1$.
\end{proposition}

Before proceeding to the proof, it is useful to recall some fundamental concepts from linear algebra. For a more detailed treatment, one can consult~\cite{axler1997} or~\cite{strang2016}. Let $V$ be a finite-dimensional real vector space. An \textbf{inner product} on $V$ is a function $\langle \cdot, \cdot \rangle: V \times V \to \mathbb{R}$ that is positive-definite, symmetric, and bilinear. The standard inner product on $\mathbb{R}^n$ for vectors $u=(u_1, \dots, u_n)$ and $v=(v_1, \dots, v_n)$ is the dot product: $\langle u, v \rangle = \sum_{i=1}^n u_i v_i$. Two vectors $u, v \in V$ are \textbf{orthogonal} if $\langle u, v \rangle = 0$. For any subspace $W \subseteq V$, its \textbf{orthogonal complement}, denoted $W^\perp$, is the set of all vectors in $V$ that are orthogonal to every vector in $W$. A crucial result states that $V$ can be decomposed as the direct sum $V = W \oplus W^\perp$, which implies $\dim(V) = \dim(W) + \dim(W^\perp)$. If $W$ is a \emph{proper} subspace of $V$ (i.e., $\dim(W) < \dim(V)$), then its orthogonal complement $W^\perp$ must be non-trivial and contain non-zero vectors.

\begin{proof}

    If $m = 0$, then $\mathcal{V}$ is trivial vector space and $\text{pos}(\mathcal{V}) =\{\emptyset\}$. Since, $V(\{\emptyset\}) = 1$ the statement holds trivially. So, assume $m \geq 1$. We will show $V(\text{pos}(\mathcal{V})) \leq m +1$ and $V(\text{pos}(\mathcal{V})) > m $ separately to establish the desired result.

    To show $V(\text{pos}(\mathcal{V})) \leq m +1 $, let $A = \{a_1, \dots, a_{m+1}\} \subseteq X$ be a finite subset with $m+1$ elements. We will show that $\text{pos}(\mathcal{V})$ cannot shatter $A$. Consider the linear map
    \[
        \begin{aligned}
            \chi: \mathcal{V} &\to \mathbb{R}^A \\
            f &\mapsto f \mid_A,
        \end{aligned}
    \]
    where $\mathbb{R}^A$ is a vector space of all real valued functions defined on the set $A$ and $f \mid_A$ is a restriction of the function $f \in \mathcal{V}$. It's straightforward to verify that $\chi$ is a linear map. Note that, each element of $g \in \mathbb{R}^A$ is completely determined by the $m+1$ values it takes on each element of $A$, so $g$ can be represented by the vector
    \[
        (g(a_1), \dots, g(a_{m+1})).
    \]
    Hence, $\mathbb{R}^A$ is isomorphic to $\mathbb{R}^{m+1}$ and $\dim(\mathbb{R}^A) = m+1$.

    The image of $\chi$, denoted as $\mathcal{V}|_A \coloneqq \{f|_A : f \in \mathcal{V}\}$, is a vector subspace of $\mathbb{R}^A$ as it inherits vector space structure from $\mathcal{V}$. By the rank-nullity theorem [CITE], the dimension of the image $\mathcal{V}|_A$ cannot exceed the dimension of its domain $\mathcal{V}$. Therefore:
    \begin{equation}
        \label{eq:orthogonal-compliment}
        \dim(\mathcal{V}|_A) \le \dim(\mathcal{V}) = m < m+1 = \dim(\mathbb{R}^A).
    \end{equation}
    It follows from~\ref{eq:orthogonal-compliment} that, $\mathcal{V}|_A$ is a proper subspace of $\mathbb{R}^A$. This means $\mathcal{V}|_A$ has a non-trivial orthogonal complement within $\mathbb{R}^A$, denoted as $(\mathcal{V}|_A)^\perp$. Hence, there exists a non-zero vector $w \in \mathbb{R}^A$ such that for any $f \in \mathcal{V}$, the orthogonality condition holds, ie
    \[
        \langle f|_A, w \rangle = \sum_{i=1}^{m+1} f(a_i) w(a_i) = 0.
    \]
    Let $A^+ = \{a \in A \mid w(a) > 0\}$. Since $w$ is non-zero, we can assume $A^+$ is non-empty (otherwise, we replace $w$ with $-w$). We claim $A^+$ cannot be cut out. Assume for contradiction there exists an $f_0 \in \mathcal{V}$ such that $\text{pos}(f_0) \cap A = A^+$. This implies $f_0(a) > 0$ for $a \in A^+$ and $f_0(a) \le 0$ for $a \in A \setminus A^+$.
    However, the inner product
    \[
        \langle f_0|_A, w \rangle = \underbrace{\sum_{a \in A^+} f_0(a) w(a)}_{> 0} + \underbrace{\sum_{a \in A \setminus A^+} f_0(a) w(a)}_{\geq 0}
    \]
    is a sum of strictly positive terms and non-negative terms, so it must be strictly positive. This contradicts that the inner product must be zero. Thus, no set of size $m+1$ can be shattered, which means $V(\text{pos}(\mathcal{V})) \le m+1$.

    To show $V(\text{pos}(\mathcal{V})) > m $, we need to find an $m$-element subset of $X$ that $\text{pos}(\mathcal{V})$ shatters. We will build this set inductively. Firstly, note that since $m \geq 1$, $\mathcal{V}$ contains non-zero function, call it $f_1: X \to \mathbb{R}$. Hence, there exists $b_1 \in X$ such that $f_1(b_1) \neq 0$. Write $\mathcal{V}_0 \coloneq \mathcal{V}$ and consider the linear evaluation function
    \[
        \begin{aligned}
            \text{ev}_{b_1}: \mathcal{V}_0 &\to \mathbb{R}, \\
            f &\mapsto f(b_1).
        \end{aligned}
    \]
    Since, $\text{ev}_{b_1}(f_1) \neq 0$, the image of this function has dimension 1. Write $\mathcal{V}_1 \coloneq \{f \in \mathcal{V}_0 \mid f(b_1) = 0\}$. Then $\mathcal{V}_1$ is precisely the kernel of $\text{ev}_{b_1}$. Hence, by rank nullity theorem
    \begin{equation}
        \label{eq:lower-bound}
        \dim(\mathcal{V}_0) = \dim(\mathcal{V}_1) + \dim(im(\text{ev}_{b_1})).
    \end{equation}
    It follows from~\ref{eq:lower-bound} that $\dim(\mathcal{V}_1) = m - 1$. For the inductive step, assume we have found points $\{b_1, \dots, b_{m-1}\}$ and a subspace $\mathcal{V}_{m-1} \coloneq \{f \in \mathcal{V} \mid f(b_1) = f(b_2) = \dots = f(b_{m-1}) = 0\}$ with $\dim(\mathcal{V}_{k-1}) = m - (m - 1) = 1$. Then, since $\dim(\mathcal{V}_{k-1}) = 1 > 0$, we can find $f_m \in \mathcal{V}_{k-1}$ such that there exists $b_m \in X$ such that $f_m(b_m) \neq 0$. Then, we construct $\mathcal{C}_m$ similarly and $\dim(\mathcal{C}_m) = 0$. Write $B \coloneq \{b_1, \dots, b_m\}$ for all the points we have found and define the linear map
    \[
        \begin{aligned}
            \phi: \mathcal{V} &\to \mathbb{R}^B, \\
            f &\mapsto f\mid_B.
        \end{aligned}
    \]
    Then, the kernel of $\phi$ is trivial, since for any $f \in \mathcal{V}$, if $\phi(f) = f\mid_B = 0$, then $f$ is zero function in $\mathcal{V}$ by construction of $B$. Since, we $\dim(im(\phi)) = \dim(\mathbb{R}^B)$, it follows from the last two arguments that $\phi$ is an isomorphism. This completes proof, since for any subset $D \subset B$, we can find a sign function~\ref{eq:sign-function} $\mathds{1}_D: B \to \mathbb{R}$ such that
    \[
        \mathds{1}_D(b) =
        \begin{cases}
            1,  & b\in D,\\
            -1, & b\notin D.
        \end{cases}
    \]
    and then $\mathds{1}_D$ is necessarily a restriction of a unique function in $\mathcal{V}$.
\end{proof}

Below, we present an application of Proposition~\ref{prop:vector-space-dependent}.

\begin{example}
    Let $X = \mathbb{R}^N$ and consider the vector space $\mathcal{W}$ of real polynomial functions $P:\mathbb{R}^N \to \mathbb{R}$ with $\deg(P) \leq d$ for some $d \in \mathbb{N}$. This space is finite-dimensional, as it is spanned by the finite set of monomials of degree at most $d$. Therefore, by Proposition~\ref{prop:vector-space-dependent}, the collection of positivity sets $\{\text{pos}(P) : P \in \mathcal{W}\}$ is a dependent collection (a VC-class).
\end{example}

This result for polynomials forms the base case for proving the following important lemma, which corresponds to a special case of the main theorem of this chapter about o-minimal structures. Later, we will present the theorem generalizes this result from the semialgebraic setting of $(\mathbb{R}, +, \cdot, <)$ to any o-minimal structure.

\begin{lemma}[Semialgebraic relations are dependent]
    \label{lem:semialgebraic-dependent}
    Every semialgebraic relation $\Phi \subset \mathbb{R}^M \times \mathbb{R}^N$ is dependent.
\end{lemma}

\begin{proof}

    Let $\Phi \subseteq \mathbb{R}^M \times \mathbb{R}^N$ be a semialgebraic relation. By the definition of a semialgebraic set, $\Phi$ is a finite Boolean combination of atomic sets, which are the solution sets of single polynomial inequalities. That is, $\Phi$ can be constructed from a finite number of "atomic" relations of the form
    \[
        \Psi_P = \{(x, y) \in \mathbb{R}^M \times \mathbb{R}^N \mid P(x, y) > 0\},
    \]
    where $P$ is a polynomial in the variables $(x, y) \in \mathbb{R}^M \times \mathbb{R}^N$. The Boolean operations correspond to taking unions, intersections, and complements of these atomic relations.

    The proof proceeds in two main steps. First, we show that every such atomic relation $\Psi_P$ is dependent. Second, we use the lemmas proven in the previous subsubsection, which state that the property of being dependent is closed under Boolean operations, to conclude that $\Phi$ must be dependent.

    To show atomic relations are dependent, let $P(x, y)$ be a polynomial in $M+N$ variables with real coefficients, and let $d$ be its total degree. Consider the atomic relation $\Psi_P \subseteq \mathbb{R}^M \times \mathbb{R}^N$ defined by the inequality $P(x, y) > 0$.

    To show that $\Psi_P$ is dependent, we will use the duality established previously and analyze the collection of its fibers. Let $X = \mathbb{R}^M$ and $Y = \mathbb{R}^N$. We consider the collection of fibers indexed by $x \in X$:
    \[
        (\Psi_P)_X \coloneqq \{(\Psi_P)_x \mid x \in \mathbb{R}^M\} \subseteq \mathcal{P}(\mathbb{R}^N),
    \]
    where each fiber is given by $(\Psi_P)_x = \{y \in \mathbb{R}^N \mid P(x, y) > 0\}$. The relation $\Psi_P$ is dependent if and only if this collection of fibers is a VC-class (i.e., has finite VC-dimension).

    For each fixed $x \in \mathbb{R}^M$, the function $P_x: \mathbb{R}^N \to \mathbb{R}$, defined as $y \mapsto P(x, y)$, is a polynomial in the $N$ variables of $y\coloneq  (y_1, \dots, y_N)$. We can write $P_x(y)$ as a sum over multi-indices $\beta \coloneq(\beta_1, \dots, \beta_N) \in \mathbb{Z}_{\geq}^N$:
    \[
        P_x(y) = P(x, y) = \sum_{|\beta| \le d} C_\beta(x) y^\beta,
    \]
    where each coefficient $C_\beta(x)$ is itself a polynomial in the $M$ variables of $x$. Here, we use shorthand notation $y^{\beta} = y_1^{\beta_1}\cdot\dots\cdot y_N^{\beta_N}$ and $|\beta| = \beta_1 + \dots + \beta_N$.
    Hence, for any fixed $x$, the function $P_x(y)$ is a polynomial in $y$ of degree at most $d$.

    Let $\mathcal{W}$ be the real vector space of all polynomial functions from $\mathbb{R}^N$ to $\mathbb{R}$ of degree at most $d$. This vector space is finite-dimensional with basis $B$ given as
    \[
        B \coloneq \{ y_1^{\beta_1} \cdot \ \dots \ \cdot y_N^{\beta_N} \mid   \beta_1 + \dots + \beta_N \leq d \text{ and } \beta_i \in \mathbb{Z}_{\geq} \}.
    \]
    Write $m \coloneq |B| = \dim(\mathcal{W})$, which is given by the number of monomials in $N$ variables of degree up to $d$, i.e., $m = \binom{N+d}{d}$.

    For every $x \in \mathbb{R}^M$, the function $P_x(y)$ is an element of this vector space $\mathcal{W}$. The fiber $(\Psi_P)_x$ is precisely the positivity set of the function $P_x(y)$, that is, $(\Psi_P)_x = \text{pos}(P_x)$. Therefore, the entire collection of fibers $(\Psi_P)_X$ is a subcollection of the set of all possible positivity sets derived from functions in $\mathcal{W}$:
    \[
        (\Psi_P)_X = \{\text{pos}(P_x) \mid x \in \mathbb{R}^M\} \subseteq \{\text{pos}(Q) \mid Q \in \mathcal{W}\} = \text{pos}(\mathcal{W}).
    \]
    By Proposition~\ref{prop:vector-space-dependent}, the collection $\text{pos}(\mathcal{W})$ is a VC-class with VC-dimension at most $m+1$. Since the growth function of a subcollection cannot exceed that of the larger collection, $(\Psi_P)_X$ must also be a VC-class, and hence is dependent. By the VC-Dependence Duality, this implies that the relation $\Psi_P$ is dependent.

    An atomic relation could also be defined by a polynomial equality, $P(x, y) = 0$. Such a relation can be expressed as the negation of a union of two strict inequalities:
    \[
        \{(x,y) \mid P(x,y) = 0\} = \mathbb{R}^{M+N} \setminus \left( \{(x,y) \mid P(x,y) > 0\} \cup \{(x,y) \mid -P(x,y) > 0\} \right).
    \]
    Since we have shown that relations defined by strict inequalities are dependent, and we know from the Lemma~\ref{lem:dependent-union-intersection} that the class of dependent relations is closed under union and negation, it follows that relations defined by polynomial equalities are also dependent.

    To show any semialgebraic relation $\Phi$ is dependent recall that, by definition, a finite Boolean combination of atomic relations of the forms just discussed. Since we have established that all such atomic relations are dependent, a repeated application of the Lemma~\ref{lem:dependent-union-intersection} stating that the union and intersection of two dependent relations are dependent (and that the negation of a dependent relation is dependent) allows us to conclude that the entire semialgebraic relation $\Phi$ is dependent. This completes the proof.
\end{proof}

\subsection{Dependence in o-minimal structures}



\begin{corollary}
    Let $(R, \mathcal{S})$ be an o-minimal structure and $\Phi \subseteq R^{p} \times R^q$ for all $p, q > 0$ be a definable relation. Then $\Phi$ is dependent. In particular, there is $d_{\Phi} \in \mathbb{N}$ such that for all sufficiently large $n$ each subset $F \subseteq R^q$ with $|F| = n$ has at most $n^{d_{\Phi}}$ subsets of the form $\Phi_{x} \cap F$ with $x \in R^p$.
\end{corollary}

\begin{proof}

    We use~\ref{lem:o-minimal-dependence-hypothesis} to deduce that for any definable set $\Phi \subseteq R^{p}\times R $ is dependent. This satisfies the assumption of the Theorem~\ref{thm:main-theorem-ch-2}, hence $\Phi \subseteq R^{p}\times R^{q}$ is dependent for any $q > 0$. Hence,
    \[
        \Phi_{X} \coloneq \{\Phi_x \mid  x \in R^p\}
    \]
    is VC-class.
\end{proof}

This corollary represents the central achievement of this section, formally establishing that o-minimal structures possess the NIP (Not Independence Property). The statement that every definable relation is dependent is a profound regularity theorem. It asserts that the geometric complexity of sets definable within these structures is fundamentally constrained. Unlike more 'wild' mathematical settings where one can define sets of arbitrary complexity, o-minimal structures only permit sets that behave, in a combinatorial sense, like simple geometric objects such as lines, planes, and their finite Boolean combinations. This inherent tameness prevents the definable families of sets from shattering arbitrarily large finite sets, a key insight into their well-behaved nature.

The second part of the corollary translates this abstract property into a concrete, quantitative guarantee. It reveals that while the family of fibers $\{\Phi_x \mid x \in R^p\}$ may be infinite, its expressive power on any finite set is severely limited. Given any finite 'test set' $F$ of $n$ points, the number of distinct patterns that can be 'cut out' by the fibers does not grow exponentially, but only polynomially, as $n^{d_\Phi}$. This polynomial bound ensures that the concept class defined by $\Phi$ has a finite VC-dimension, $d_\Phi$. As we will explore in the next chapter, this finite VC-dimension is the key parameter that guarantees a concept class is efficiently learnable from a finite number of examples, directly linking the logical property of o-minimality to the statistical property of PAC learnability.