\chapter{Literature review}
\chaptermark{Background} %shortened running header

This chapter discusses some  related research.
This chapter is optional and the discussion about
the related research could be done in the Introduction.

Another intermediate page to demonstrate the shortened running header.
Another intermediate page to demonstrate the shortened running header.
Another intermediate page to demonstrate the shortened running header.
Another intermediate page to demonstrate the shortened running header.
Another intermediate page to demonstrate the shortened running header.
Another intermediate page to demonstrate the shortened running header.
Another intermediate page to demonstrate the shortened running header.
Another intermediate page to demonstrate the shortened running header.
Another intermediate page to demonstrate the shortened running header.
Another intermediate page to demonstrate the shortened running header.
Another intermediate page to demonstrate the shortened running header.
Another intermediate page to demonstrate the shortened running header.
Another intermediate page to demonstrate the shortened running header.
Another intermediate page to demonstrate the shortened running header.
Another intermediate page to demonstrate the shortened running header.
Another intermediate page to demonstrate the shortened running header.
Another intermediate page to demonstrate the shortened running header.
Another intermediate page to demonstrate the shortened running header.
Another intermediate page to demonstrate the shortened running header.
Another intermediate page to demonstrate the shortened running header.

\chapter{Another chapter}

\section[My short title]{This is a really really long title}
\label{sec:My short title}

Lets write an equation that will get a label
\begin{equation}
    g(z) = \int_0^z\int_0^{x^2} \frac{\pi}{x^2+y^2} e^{-x^2} dy dx 	,
    \label{eqn:g(z) function}
\end{equation}
where $g(z)$ is some function we wish to calculate.
You can refer to equation~(\ref{eqn:g(z) function}) using this label.

We can also add a list of things that we would like to do in this demonstration:
\begin{itemize}
    \item \textbf{structure} a large tex document,
    \item \textbf{cross-reference} within your document,
    \item include \textbf{tables} and \textbf{figures} in your document,
    \item create and maintain a \textbf{bibliography}.
\end{itemize}
We could also add a numbered list, such as list of tasks you could try out.
\begin{enumerate}
    \item Add the packages \texttt{fullpage}, \texttt{amsmath} and  \texttt{amssymb} to the \texttt{main.tex} file.
    \item Create your own list, what happens if you create a list inside a list? Be careful to end environments that you start.
    \item Try out writing equations on multiple lines using the following environments, \texttt{eqnarray}, \texttt{align} and \texttt{gather}.
\end{enumerate}


Now we are going to create a table. First create the tabular environment and put some values in.

\begin{table}
    \begin{center} % center the table and caption
        \begin{tabular}{c|ccc}
            & \multicolumn{3}{c}{Attendance at Course} \\
            Week & MATH10121 & MATH10212 & MATH10222 \\
            \hline
            1 & 234 & 157 & 153 \\
            2 & 223 & 167 & 148 \\
            3 & 227 & 129 & 133 \\
            4 & 236 & 134 & 113 \\
            5 & 212 & 127 & 101
        \end{tabular}
        \caption{My table showing preparations times} % caption for the table (gives it a number)
        \label{tab:myTable}
    \end{center}
\end{table}

\section{Another Section}

Can drill down further into subsections \cite{Feller71}.

\subsection{Case 1}

Can drill down further into subsections using the method in preprint~\cite{Perelman}.
This is a reference to a journal paper~\cite{Akaike} or~\cite{GreengardRokhlin1987}.

\subsection{Case 2}

Can drill down further into subsections.

\subsection{Case 3}

Can drill down further into subsections.

\chapter{Conclusions}
And here is the final chapter summarising what you have learned/found out by doing this dissertation project.

