\section{Proof of the Fundamental Theorem of Statistical Learning}

In this section, we provide the proof for the "only if" direction of the Fundamental Theorem of Statistical Learning (\ref{thm:fundamental-theorem}). We will then briefly outline the argument for the converse, which relies on the Uniform Convergence Property.

\subsection{PAC Learnability Implies Finite VC Dimension}

We will prove the following statement, which is one half of the Fundamental Theorem.

\begin{theorem}
    Let the learning framework $(\mathcal{X}, \Sigma_{\mathcal{Z}}, \mathcal{D}, \mathcal{H})$ be given as in Theorem~\ref{thm:fundamental-theorem}. If $\mathcal{H}$ is PAC learnable, then $\operatorname{vc}(\mathcal{H}) < \infty$.
\end{theorem}

\begin{proof}

    The proof proceeds by contrapositive. We assume that $\mathcal{H}$ has an infinite VC dimension and show that it cannot be PAC learnable. To show that $\mathcal{H}$ is not PAC learnable, we must negate its definition. That is, we must show that there exist fixed accuracy and confidence parameters, $\varepsilon > 0$ and $\delta > 0$, such that for any sample size $m \in \mathbb{N}$, there exists a distribution $\mathbb{D} \in \mathcal{D}$ for which any learning algorithm $\mathcal{A}$ fails to meet the PAC guarantee. Specifically, the probability of the algorithm returning a hypothesis with an excess error greater than $\varepsilon$ will be at least $\delta$.

    Let us fix the PAC parameters $\varepsilon = 1/8$ and $\delta = 1/7$. Let $\mathcal{A}$ be an arbitrary learning function and $m \in \mathbb{N}$ be any sample size. Our goal is to find a distribution $\mathbb{D} \in \mathcal{D}$ such that the hypothesis $\mathcal{A}(\overline{z})$ returned by the algorithm on a sample $\overline{z} \sim \mathbb{D}^m$ fails the PAC condition, i.e.,
    \[
        \mathbb{D}^m\left(\left\{ \overline{z} \in \mathcal{Z}^m \mid \operatorname{er}_{\mathbb{D}}(\mathcal{A}(\overline{z})) - \operatorname{opt}_{\mathbb{D}}(\mathcal{H}) > \frac{1}{8} \right\}\right) \ge \frac{1}{7}.
    \]
    This will demonstrate that no sample size $m_0$ can satisfy the PAC definition for these fixed $\varepsilon$ and $\delta$, proving that $\mathcal{H}$ is not PAC learnable.

    First, we use the assumption of an infinite VC dimension to construct a specific learning problem that is designed to be difficult for any algorithm. Since $\operatorname{vc}(\mathcal{H}) = \infty$, there exists a set of any finite size that can be shattered by $\mathcal{H}$. We choose a set $S = \{x_1, \dots, x_{2m}\} \subseteq \mathcal{X}$ of size $2m$ that is shattered by $\mathcal{H}$.

    Let $\mathcal{F} = \{0,1\}^S$ be the set of all $T \coloneq 2^{2m}$ possible binary functions on $S$. We index each function in $\mathcal{F}$ as
    \[
        \{0,1\}^S = \{f_1, \dots, f_T\}.
    \]
    By the definition of shattering, for each function $f_i \in \mathcal{F}$, there exists a hypothesis $h_i \in \mathcal{H}$ such that its restriction to $S$ is exactly $f_i$, i.e., $h_i|_S = f_i$.

    Next, we use this shattered set to define a family of probability distributions. Let $\mathcal{Z}_S = S \times \{0,1\}$ be the finite sample space restricted to the shattered set $S$. For each $1 \leq i \leq T$, we define a distribution $\mathbb{D}_i$ on the measurable space $(\mathcal{Z}_S, \mathcal{P}(\mathcal{Z}_S))$. This distribution's probability mass function is given by:
    \[
        \mathbb{D}_i(\{z\}) =
        \begin{cases}
            \frac{1}{2m} & \text{if } z \in \Gamma(f_i), \\
            0 & \text{otherwise},
        \end{cases}
    \]
    where $\Gamma(f_i) = \{(x, f_i(x)) \mid x \in S\}$ is the graph of the function $f$ over the set $S$. In other words, $\mathbb{D}_i$ is the discrete uniform distribution on the $2m$ points that constitute the graph of $f_i$. Note that, by definition, $\operatorname{er}_{\mathbb{D}_i}(f_i) = 0$ for all $i \in \{1, \dots, T\}$.

    The PAC condition for our chosen $\varepsilon=1/8$ and a distribution $\mathbb{D}_i$ simplifies to requiring that $\operatorname{er}_{\mathbb{D}_i}(\mathcal{A}(\overline{z})) \le 1/8$. Our strategy is to show that no single algorithm $\mathcal{A}$ can perform well simultaneously for all these distributions. We will show that there is at least one function $f \in \mathcal{F}$ for which the algorithm's expected error is high. The core of the argument is that a sample of size $m$ is insufficient to distinguish between the $2^{2m}$ possible underlying functions, as it is likely to reveal the labels of at most $m$ of the $2m$ points in $S$.

    The following lemma formalises this idea. It states that there must be at least one distribution $\mathbb{D}_j$ for which the learning algorithm $\mathcal{A}$ has a high expected error, where $j \in \{1, \dots, T\}$.

    \begin{lemma}
        \label{lem:no-free-lunch}
        Let $\mathcal{A}_S(\overline{z}) = \mathcal{A}(\overline{z})|_S$ be the restriction of the learned hypothesis to the set $S$. Then the following inequality holds:
        \[
            \max_{1 \leq i \leq T} \Bigl( \mathbb{E}_{\overline{z} \sim \mathbb{D}_i^m} \left[ \operatorname{er}_{\mathbb{D}_i}(\mathcal{A}_S(\overline{z})) \right]\Bigr) \ge \frac{1}{4}.
        \]
    \end{lemma}

    \begin{subproof}[Proof of Lemma~\ref{lem:no-free-lunch}]

        Note that, $\operatorname{er}_{\mathbb{D}_i}(\mathcal{A}(\square)): \mathcal{Z}_S^m\to [0,1]$ is positive and the bounded random variable. Hence, by the definition of well-definedness of expected value~\cite[Def 8.3]{MeasureTheoryLeGall},
        \[
            \mathbb{E}_{\overline{z} \sim \mathbb{D}_i^m}(\operatorname{er}_{\mathbb{D}_i}(\mathcal{A}(\square)))
        \]
        is well-defined.

        Moreover, the maximum of a set of values is always greater than or equal to their average. Therefore, we can find lower-bound for the maximum expected error by the average expected error taken over all $1 \leq i \leq T$:
        \[
            \max_{1 \leq i \leq T} \Bigl( \mathbb{E}_{\overline{z} \sim \mathbb{D}_i^m} \left[ \operatorname{er}_{\mathbb{D}_i}(\mathcal{A}_S(\overline{z})) \right]\Bigr) \ge \frac{1}{T} \sum_{i = 1}^{T} \mathbb{E}_{\overline{z} \sim \mathbb{D}_i^m} \left[ \operatorname{er}_{\mathbb{D}_i}(\mathcal{A}_S(\overline{z})) \right].
        \]
        We first analyze the average on the right-hand side. A sample $\overline{z}$ drawn from $\mathbb{D}_i^m$ consists of $m$ pairs $(x,y)$ where each $x$ is drawn uniformly from $S$ and $y=f_i(x)$. Let $S^m$ denote the set of all possible sequences of $m$ instances drawn from $S$ and $|S^m| = (2m)^m$. Note that, we draw these $m$ instances from $S$ with replacements, hence some $m$-tuples in $S^m$ contain duplicate instances. This observation will be used later in the proof.

        For any sequence of instances $\overline{x} = (x_{j_1}, \dots, x_{j_m}) \in S^m$, denote the corresponding labeled sample under $f_i\in \mathcal{F}$ by
        \[
            \overline{z}_{\overline{x}}^{i} \coloneq ((x_{j_1}, f_i(x_{j_1})), \dots, (x_{j_m}, f_i(x_{j_m}))).
        \]
        The probability of drawing this specific sample is $\mathbb{D}_i^m(\{\overline{z}_{\overline{x}}^{i}\}) = (1/2m)^m$. Therefore,
        \[
            \mathbb{E}_{\overline{z} \sim \mathbb{D}_i^m} \left[ \operatorname{er}_{\mathbb{D}_i}(\mathcal{A}_S(\overline{z})) \right] = \sum_{\overline{x} \in S^m} \frac{1}{(2m)^m} \operatorname{er}_{\mathbb{D}_i}(\mathcal{A}_S(\overline{z}_{\overline{x}}^{i})))
        \]
        for all $i \in \{1, \dots, T\}$.

        We can now rewrite the expectation as a sum over all possible instance sequences $\overline{x} \in S^m$ and then swap the order of summation:
        \begin{align*}
            \frac{1}{T} \sum_{i=1}^{T} \mathbb{E}_{\overline{z} \sim \mathbb{D}_i^m} \left[ \operatorname{er}_{\mathbb{D}_i}(\mathcal{A}_S(\overline{z})) \right] &= \frac{1}{T}  \sum_{i=1}^{T} \left( \sum_{\overline{x} \in S^m} \frac{1}{(2m)^m} \operatorname{er}_{\mathbb{D}_i}(\mathcal{A}_S(\overline{z}_{\overline{x}}^{i}))) \right) \\
            &= \frac{1}{(2m)^m} \sum_{\overline{x} \in S^m} \left( \frac{1}{T} \sum_{i=1}^{T} \operatorname{er}_{\mathbb{D}_i}(\mathcal{A}_S(\overline{z}_{\overline{x}}^{i})) \right). \\
            &\geq \min_{\overline{x} \in S^m} \frac{1}{T}  \sum_{i=1}^{T} \operatorname{er}_{\mathbb{D}_i}(\mathcal{A}_S(\overline{z}_{\overline{x}}^{i}))
        \end{align*}

        Fix an arbitrary sequence of instances $\overline{x}'=(x_1',\dots, x_m') \in S^m$. Let
        \[
            \{v_1, \dots, v_p\} \coloneq S \setminus \{x_1', \dots, x_m'\}.
        \]
        Since, $\overline{x}'$ might contain some duplicates, $|S| = 2m \leq p \leq m$. Then for any $f \in \mathcal{F}$ and $i \in \{1, \dots, T\}$, the true error of $f$ with respect to $\mathbb{D}_i$ is given by the probability of $f(x) \neq f_i(x)$ for $x \in S$:
        \[
            \operatorname{er}_{\mathbb{D}_i}(f) = \mathbb{D}_i(\mathcal{Z}_S\setminus\Gamma(f)) = \frac{1}{2m}\sum_{x \in S}\mathds{1}_{\mathcal{Z}_S\setminus\Gamma(f_i)}(x, f(x)).
        \]
        Since, $\mathds{1}_{\mathcal{Z}_S\setminus\Gamma(f_i)}(x, f(x)) \geq 0$ for all $x \in S$ and $\{v_1, \dots, v_p\} \subseteq S$, we get
        \[
            \operatorname{er}_{\mathbb{D}_i}(f) \geq \frac{1}{2m}\sum_{r=1}^p \mathds{1}_{\mathcal{Z}_S\setminus\Gamma(f_i)}(v_r, f(v_r)).
        \]
        By using, $\frac{1}{m} \geq \frac{1}{p}$,
        \[
            \operatorname{er}_{\mathbb{D}_i}(f) \geq \frac{1}{2p}\sum_{r=1}^p \mathds{1}_{\mathcal{Z}_S\setminus\Gamma(f_i)}(v_r, f(v_r)).
        \]
        For the rest of the proof, we denote $h' \coloneq \mathcal{A}_S(\overline{z}_{\overline{x}'}^{i})$ for our fixed $\overline{x}'$ for notational clarity.
        \[
            \begin{aligned}
                &\frac{1}{T}\sum_{i=1}^T\operatorname{er}_{\mathbb{D}_i}(h') \\
                &\geq \frac{1}{T}\sum_{i=1}^T \Bigl(\frac{1}{2p}\sum_{r=1}^p \mathds{1}_{\mathcal{Z}_S\setminus\Gamma(f_i)}(v_r, h'(v_r))\Bigr) \\
                &= \frac{1}{2}\Biggl(\frac{1}{p}\sum_{r=1}^p \Bigl(\frac{1}{T}\sum_{i=1}^T \mathds{1}_{\mathcal{Z}_S\setminus\Gamma(f_i)}(v_r, h'(v_r))\Bigr)\Biggr) \\
                &\geq \frac{1}{2} \min_{1 \leq r \leq p}\Bigl(\frac{1}{T}\sum_{i=1}^T \mathds{1}_{\mathcal{Z}_S\setminus\Gamma(f_i)}(v_r, h'(v_r))\Bigr),
            \end{aligned}
        \]
        where the last inequality follows from the basic fact that minimum of finite set of numbers is bounded above by the average of those numbers.
        Now, we fix an arbitrary $r \in \{1, \dots, p\}$ and evaluate the term inside the minimum:
        \[
            \frac{1}{T}\sum_{i=1}^T \mathds{1}_{\mathcal{Z}_S\setminus\Gamma(f_i)}(v_r, h'(v_r)).
        \]
        The key insight is that the hypothesis $h'$ depends on the sample $\overline{z}_{\overline{x}'}^{i}$, which in turn depends on the labels that $f_i$ assigns to the points in $\overline{x}'$. The point $v_r$ is not in the sequence $\overline{x}'$.

        We can partition the set of all functions $\mathcal{F}=\{f_1, \dots, f_T\}$ into $T/2$ disjoint pairs. Let $(f_j, f_k)$ be such a pair, constructed so that they differ only at the point $v_r$:
        \[
            f_j(x) \neq f_k(x) \iff x = v_r.
        \]
        Since $f_j$ and $f_k$ agree on all points in $\overline{x}'$, the labeled samples they generate are identical, i.e., $\overline{z}_{\overline{x}'}^{j} = \overline{z}_{\overline{x}'}^{k}$. As the learning algorithm $\mathcal{A}$ is a deterministic function, it must produce the same hypothesis for both samples. Thus, the hypothesis we denoted $h'$ is the same for both $f_j$ and $f_k$.

        This common hypothesis $h'$ makes a prediction $h'(v_r)$. Since $f_j(v_r) \neq f_k(v_r)$, $h'(v_r)$ must be different from exactly one of them. This means that for each pair, exactly one of the hypotheses is wrong at $v_r$:
        \[
            \mathds{1}_{\mathcal{Z}_S\setminus\Gamma(f_j)}(v_r, h'(v_r)) + \mathds{1}_{\mathcal{Z}_S\setminus\Gamma(f_k)}(v_r, h'(v_r)) = 1.
        \]
        Summing over all $T/2$ such pairs, the total sum of the indicators over all $T$ functions is exactly $T/2$. The average is therefore:
        \[
            \frac{1}{T}\sum_{i=1}^T \mathds{1}_{\mathcal{Z}_S\setminus\Gamma(f_i)}(v_r, h'(v_r)) = \frac{T/2}{T} = \frac{1}{2}.
        \]
        Since this holds for any $r$, the minimum over $r$ is also $1/2$. Substituting this back into our main inequality, we get:
        \[
            \frac{1}{T}\sum_{i=1}^T\operatorname{er}_{\mathbb{D}_i}(h') \geq \frac{1}{2} \cdot \frac{1}{2} = \frac{1}{4}.
        \]
        This lower bound holds for any sequence $\overline{x}' \in S^m$, so the minimum over all such sequences is also at least $1/4$. This completes the proof of the lemma.
    \end{subproof}

    With Lemma~\ref{lem:no-free-lunch} established, we can fix an index $j \in \{1, \dots, T\}$ such that
    \begin{equation}
        \label{eq:fixed-j-after-lemma-1}
        \mathbb{E}_{\overline{z} \sim \mathbb{D}_j^m} \left[ \operatorname{er}_{\mathbb{D}_j}(\mathcal{A}_S(\overline{z})) \right] \ge \frac{1}{4}.
    \end{equation}
    We now extend this distribution $\mathbb{D}_j$ to a distribution $\widehat{\mathbb{D}}$ on the discrete $\sigma$-algebra $\mathcal{P}(\mathcal{Z})$ by setting
    \[
        \widehat{\mathbb{D}}(C) \coloneqq \mathbb{D}_j(C \cap \mathcal{Z}_S) \quad \text{for any } C \subseteq \mathcal{Z}.
    \]
    The distribution $\mathbb{D}$ is then defined as the restriction of $\widehat{\mathbb{D}}$ to the original $\sigma$-algebra $\Sigma_{\mathcal{Z}}$. Since $\mathbb{D}_j$ is a discrete uniform distribution, so is $\mathbb{D}$, and thus $\mathbb{D} \in \mathcal{D}$ by our initial assumption. This extension yields for any $h \in \mathcal{H}$ that
    \begin{align}
        \operatorname{er}_{\mathbb{D}}(h) &= \mathbb{D}(\mathcal{Z} \setminus \Gamma(h)) \nonumber \\
        &= \mathbb{D}_j((\mathcal{Z} \setminus \Gamma(h)) \cap \mathcal{Z}_S) \nonumber \\
        &= \mathbb{D}_j(\mathcal{Z}_S \setminus \Gamma(h|_S)) = \operatorname{er}_{\mathbb{D}_j}(h|_S). \label{eq:error-equivalence}
    \end{align}
    Thus, $\operatorname{opt}_{\mathbb{D}}(\mathcal{H}) = 0$, as $\operatorname{er}_{\mathbb{D}}(h_j) = \operatorname{er}_{\mathbb{D}_j}(f_j) = 0$ by~\ref{lem:no-free-lunch} and~\ref{eq:fixed-j-after-lemma-1}.

    The next lemma shows that the high expected error from Lemma~\ref{lem:no-free-lunch} carries over to the full hypothesis under the extended measure $\widehat{\mathbb{D}}$.

    \begin{lemma}
        \label{lem:full-error-bound-hat}
        We have
        \[
            \mathbb{E}_{\overline{z} \sim \widehat{\mathbb{D}}^m} \left[ \operatorname{er}_{\mathbb{D}}(\mathcal{A}(\overline{z})) \right] \ge \frac{1}{4}.
        \]
    \end{lemma}
    \begin{subproof}[Proof of Lemma~\ref{lem:full-error-bound-hat}]

        We show that the expectation under $\widehat{\mathbb{D}}^m$ is identical to the expectation under $\mathbb{D}_j^m$:
        \[
            \mathbb{E}_{\overline{z} \sim \widehat{\mathbb{D}}^m} \left[ \operatorname{er}_{\mathbb{D}}(\mathcal{A}(\overline{z})) \right] = \mathbb{E}_{\overline{z} \sim \mathbb{D}_j^m} \left[ \operatorname{er}_{\mathbb{D}_j}(\mathcal{A}_S(\overline{z})) \right].
        \]
        The result then follows directly from the inequality established in~\eqref{eq:fixed-j-after-lemma-1}.

        To prove the equality, we expand the left-hand side. The expectation is a sum over all samples $\overline{z} \in \mathcal{Z}^m$ with non-zero probability under $\widehat{\mathbb{D}}^m$. A sample $\overline{z}$ has non-zero probability only if all its components are in $\mathcal{Z}_S$. For such a sample, which corresponds to some instance sequence $\overline{x} \in S^m$ and the function $f_j$, we have $\widehat{\mathbb{D}}^m(\{\overline{z}_{\overline{x}}^j\}) = (1/2m)^m$.

        Using this and the error equivalence from~\eqref{eq:error-equivalence}, we obtain:
        \begin{align*}
            \mathbb{E}_{\overline{z} \sim \widehat{\mathbb{D}}^m} \left[ \operatorname{er}_{\mathbb{D}}(\mathcal{A}(\overline{z})) \right] &= \sum_{\overline{x} \in S^m} \widehat{\mathbb{D}}^m(\{\overline{z}_{\overline{x}}^j\}) \operatorname{er}_{\mathbb{D}}(\mathcal{A}(\overline{z}_{\overline{x}}^j)) \\
            &= \sum_{\overline{x} \in S^m} \frac{1}{(2m)^m} \operatorname{er}_{\mathbb{D}_j}(\mathcal{A}_S(\overline{z}_{\overline{x}}^j)) \\
            &= \mathbb{E}_{\overline{z} \sim \mathbb{D}_j^m} \left[ \operatorname{er}_{\mathbb{D}_j}(\mathcal{A}_S(\overline{z})) \right].
        \end{align*}
        This completes the proof.
    \end{subproof}
    The final lemma uses this expected error bound to show that the probability of failure is high.

    \begin{lemma}
        \label{lem:pac-violation}
        The distribution $\widehat{\mathbb{D}}$ satisfies the inequality
        \[
            \widehat{\mathbb{D}}^m\left(\left\{ \overline{z} \in \mathcal{Z}^m \mid \operatorname{er}_{\mathbb{D}}(\mathcal{A}(\overline{z})) - \operatorname{opt}_{\mathbb{D}}(\mathcal{H}) > \frac{1}{8} \right\}\right) \ge \frac{1}{7}.
        \]
    \end{lemma}
    \begin{subproof}[Proof of Lemma~\ref{lem:pac-violation}]

        We recall that $\operatorname{opt}_{\mathbb{D}}(\mathcal{H}) = 0$ and, by Lemma~\ref{lem:full-error-bound-hat}, we have that $\mathbb{E}_{\overline{z} \sim \widehat{\mathbb{D}}^m} [ \operatorname{er}_{\mathbb{D}}(\mathcal{A}(\overline{z})) ] \ge 1/4$.

        Let $X(\overline{z}) = \operatorname{er}_{\mathbb{D}}(\mathcal{A}(\overline{z}))$ be the random variable for the error. Since $X$ is bounded in $[0,1]$, we can apply the one-sided Chebyshev inequality~\cite[Appendix B.1]{UnderstandinMachineLearning}. For a random variable $X \in [0,1]$ and any $a \in (0,1)$, this inequality states that $\mathbb{P}(X > a) \ge \frac{\mathbb{E}[X] - a}{1-a}$.

        Applying this with $a=1/8$ yields:
        \begin{align*}
            \widehat{\mathbb{D}}^m\left(\left\{ \overline{z} \in \mathcal{Z}^m \mid \operatorname{er}_{\mathbb{D}}(\mathcal{A}(\overline{z})) > \frac{1}{8} \right\}\right) &\ge \frac{\mathbb{E}_{\overline{z}\sim\widehat{\mathbb{D}}^m}[\operatorname{er}_{\mathbb{D}}(\mathcal{A}(\overline{z}))] - 1/8}{1 - 1/8} \\
            &\ge \frac{1/4 - 1/8}{7/8} = \frac{1/8}{7/8} = \frac{1}{7}.
        \end{align*}
        Since $\operatorname{opt}_{\mathbb{D}}(\mathcal{H}) = 0$, this is precisely the inequality stated in the lemma.
    \end{subproof}

    The PAC definition requires that for our chosen $\varepsilon=1/8$ and $\delta=1/7$, there must exist a set $C' \in \Sigma_{\mathcal{Z}}^m$ such that $C' \subseteq \{ \overline{z} \mid \operatorname{er}_{\mathbb{D}}(\mathcal{A}(\overline{z})) \le 1/8 \}$ and $\mathbb{D}^m(C') \ge 1 - 1/7$.

    However, Lemma~\ref{lem:pac-violation} shows that the complement of the set containing $C'$ has a probability of at least $1/7$ under the measure $\widehat{\mathbb{D}}^m$. Since any such $C'$ must be in $\Sigma_{\mathcal{Z}}^m$, we have $\mathbb{D}^m(C') = \widehat{\mathbb{D}}^m(C')$. This would imply that $\widehat{\mathbb{D}}^m(C') \le 1 - 1/7$, which contradicts the PAC requirement that its probability be at least $1 - 1/7$. Therefore, no such set $C'$ can exist.

    We have shown that for any sample size $m$, there exists a distribution $\mathbb{D} \in \mathcal{D}$ for which the PAC condition fails. This completes the proof by contrapositive.
\end{proof}

\subsection{Finite VC Dimension Implies PAC Learnability}

We now turn to the converse of the theorem proven in the previous section. Our goal is to show that if a hypothesis space $\mathcal{H}$ has a finite VC dimension, it is PAC learnable. Instead of a direct proof, we will establish a chain of implications that connects several key concepts. To state these equivalences precisely, we must first formally define the necessary concepts.

We begin by introducing the Uniform Convergence Property. This is a powerful statistical guarantee ensuring that, with high probability, the sample error provides a uniformly close approximation of the true error across all hypotheses in the class. As we will see, this property is the key to proving learnability.

\begin{definition}[Uniform Convergence Property]
    \label{def:ucp}
    Suppose that the map $U = U(\mathcal{H}, m, \mathbb{D}): \mathcal{Z}^m \to [0, 1]$, defined by
    \[
        \overline{z} \mapsto \sup_{h \in \mathcal{H}} |\operatorname{er}_{\mathbb{D}}(h) - \hat{\operatorname{er}}_{\overline{z}}(h)|,
    \]
    is $\Sigma_{\mathcal{Z}}^m$-measurable for any $m \ge m_{\mathcal{H}}$ and any $\mathbb{D} \in \mathcal{D}$. Then the hypothesis space $\mathcal{H}$ is said to have the \emph{uniform convergence property (UCP)} if for any $\varepsilon, \delta \in (0,1)$ there exists $m_0 = m_0(\varepsilon, \delta) \ge m_{\mathcal{H}}$ such that for any $m \ge m_0$ and any $\mathbb{D} \in \mathcal{D}$, the following inequality holds:
    \[
        \mathbb{D}^m \left( \left\{ \overline{z} \in \mathcal{Z}^m \mid \sup_{h \in \mathcal{H}} |\operatorname{er}_{\mathbb{D}}(h) - \hat{\operatorname{er}}_{\overline{z}}(h)| \le \varepsilon \right\} \right) \ge 1 - \delta.
    \]
\end{definition}

The UCP is the key to ensuring that what we learn from a finite sample (the sample error) is a reliable proxy for how well our hypothesis will perform on new data (the true error), uniformly for all hypotheses.

Next, we define the specific type of learning algorithm that will allow us to bridge the gap between UCP and PAC learnability. A natural and widely studied learning principle is \emph{Sample Error Minimization (SEM)}, which simply selects a hypothesis that best fits the training data. We consider a slightly weaker, more general condition.

\begin{definition}[Nearly Minimizing the Sample Error]
    \label{def:nmse}
    A learning function $\mathcal{A}$ for $\mathcal{H}$ is \emph{nearly minimizing the sample error (NMSE)} if for any $\varepsilon \in (0,1)$ there exists $m_0 = m_0(\varepsilon) \in \mathbb{N}$ such that for any $m \ge m_0$ and any $\overline{z} \in \mathcal{Z}^m$ the following inequality holds:
    \[
        \hat{\operatorname{er}}_{\overline{z}}(\mathcal{A}(\overline{z})) - \operatorname{opt}_{\overline{z}}(\mathcal{H}) \le \varepsilon.
    \]
\end{definition}

With these definitions in place, we can now state the full set of equivalences that constitute the Fundamental Theorem of Statistical Learning.

\begin{theorem}
    \label{thm:ftsl-equivalences}
    Let the learning framework $(\mathcal{X}, \Sigma_{\mathcal{Z}}, \mathcal{D}, \mathcal{H})$ be given as in Theorem~\ref{thm:fundamental-theorem}. The following conditions are equivalent:
    \begin{enumerate}
        \item $\mathcal{H}$ has finite VC dimension.
        \item $\mathcal{H}$ has the uniform convergence property with respect to $\mathcal{D}$.
        \item Any learning function for $\mathcal{H}$ that is nearly minimizing the sample error (NMSE) is PAC with respect to $\mathcal{D}$.
        \item $\mathcal{H}$ is PAC learnable with respect to $\mathcal{D}$.
    \end{enumerate}
\end{theorem}

\begin{remark}
    In the previous section, we proved the implication $(4) \implies (1)$. In this section, we will complete the proof of the theorem by establishing the implications $(2) \implies (3)$ and $(3) \implies (4)$. The implication $(1) \implies (2)$ is a cornerstone result of statistical learning theory, showing that a finite combinatorial dimension is sufficient to guarantee uniform statistical convergence. Its proof relies on techniques which are beyond the scope of this dissertation. We will therefore refer to~\cite[\S 3]{KrappWirth2021} for its proof.
\end{remark}

\subsubsection{UCP Implies PAC Learnability for NMSE Algorithms}

We now prove the implication $(2) \implies (3)$ from Theorem~\ref{thm:ftsl-equivalences}. The core idea is that if the sample error is a uniformly good approximation of the true error (the UCP), then an algorithm that finds a hypothesis with a nearly minimal sample error (NMSE) must also be finding a hypothesis with a nearly minimal true error.

\begin{theorem}
    Let the learning framework be as defined in Theorem~\ref{thm:fundamental-theorem}. If the hypothesis space $\mathcal{H}$ has the uniform convergence property, then any learning function for $\mathcal{H}$ that is NMSE is PAC.
\end{theorem}

\begin{proof}

    Let $\mathcal{A}$ be an NMSE learning function for $\mathcal{H}$, and assume that $\mathcal{H}$ has the UCP. Let $\varepsilon, \delta \in (0,1)$ be given. Our goal is to find a sample size $m_0$ such that for all $m \ge m_0$, the PAC condition holds.

    Since $\mathcal{H}$ is well-behaved, there exists an $m_{\mathcal{H}}$ such that the map $U$ is measurable for $m \ge m_{\mathcal{H}}$. Because $\mathcal{H}$ has the UCP and $\mathcal{A}$ is NMSE, we can choose accuracy parameters for each property. Let's use $\varepsilon/4$.
    \begin{itemize}
        \item From the UCP definition, there exists a sample size $m_0^{\text{UCP}}(\varepsilon/4, \delta)$.
        \item From the NMSE definition, there exists a sample size $m_0^{\text{NMSE}}(\varepsilon/4)$.
    \end{itemize}
    We define our overall sample size threshold as $m_0 \coloneqq \max\{m_{\mathcal{H}}, m_0^{\text{UCP}}(\varepsilon/4, \delta), m_0^{\text{NMSE}}(\varepsilon/4)\}$.

    Now, let $m \ge m_0$ and let $\mathbb{D} \in \mathcal{D}$ be any distribution. By the UCP guarantee, we know there is a set
    \[
        C \coloneqq \left\{ \overline{z} \in \mathcal{Z}^m \mid \sup_{h \in \mathcal{H}} |\operatorname{er}_{\mathbb{D}}(h) - \hat{\operatorname{er}}_{\overline{z}}(h)| \le \frac{\varepsilon}{4} \right\}
    \]
    such that $C \in \Sigma_{\mathcal{Z}}^m$ and $\mathbb{D}^m(C) \ge 1-\delta$. This set $C$ will be the one we use to satisfy the PAC definition.

    It remains to show that for any sample $\overline{z} \in C$, the excess error is bounded by $\varepsilon$. Let $\overline{z} \in C$. The condition for being in $C$ implies that for all $h \in \mathcal{H}$:
    \begin{equation} \label{eq:ucp-bound-for-all-h}
    |\operatorname{er}_{\mathbb{D}}(h) - \hat{\operatorname{er}}_{\overline{z}}(h)| \le \frac{\varepsilon}{4}.
    \end{equation}
    Furthermore, by the definition of the infimum, for any $\varepsilon' > 0$, there exists a hypothesis that is nearly optimal. Let us choose $\varepsilon' = \varepsilon/4$. Then there exists a hypothesis $h_\varepsilon \in \mathcal{H}$ such that
    \[
        \operatorname{er}_{\mathbb{D}}(h_\varepsilon) \le \operatorname{opt}_{\mathbb{D}}(\mathcal{H}) + \frac{\varepsilon}{4}.
    \]
    Now we construct a chain of inequalities for the error of the hypothesis $\mathcal{A}(\overline{z})$ returned by the algorithm on the sample $\overline{z}$:
    \begin{align*}
        \operatorname{er}_{\mathbb{D}}(\mathcal{A}(\overline{z})) &\le \hat{\operatorname{er}}_{\overline{z}}(\mathcal{A}(\overline{z})) + \frac{\varepsilon}{4} && \text{(by \eqref{eq:ucp-bound-for-all-h} applied to } h = \mathcal{A}(\overline{z})\text{)} \\
        &\le \left( \operatorname{opt}_{\overline{z}}(\mathcal{H}) + \frac{\varepsilon}{4} \right) + \frac{\varepsilon}{4} && \text{(since } \mathcal{A} \text{ is NMSE and } m \ge m_0\text{)} \\
        &= \min_{h \in \mathcal{H}} \hat{\operatorname{er}}_{\overline{z}}(h) + \frac{\varepsilon}{2} \\
        &\le \hat{\operatorname{er}}_{\overline{z}}(h_\varepsilon) + \frac{\varepsilon}{2} && \text{(by definition of minimum)} \\
        &\le \left( \operatorname{er}_{\mathbb{D}}(h_\varepsilon) + \frac{\varepsilon}{4} \right) + \frac{\varepsilon}{2} && \text{(by \eqref{eq:ucp-bound-for-all-h} applied to } h = h_\varepsilon\text{)} \\
        &\le \left( \operatorname{opt}_{\mathbb{D}}(\mathcal{H}) + \frac{\varepsilon}{4} \right) + \frac{3\varepsilon}{4} && \text{(by choice of } h_\varepsilon\text{)} \\
        &= \operatorname{opt}_{\mathbb{D}}(\mathcal{H}) + \varepsilon.
    \end{align*}
    We have shown that for any $\overline{z} \in C$, it holds that $\operatorname{er}_{\mathbb{D}}(\mathcal{A}(\overline{z})) - \operatorname{opt}_{\mathbb{D}}(\mathcal{H}) \le \varepsilon$. Since $C \subseteq \{ \overline{z} \mid \operatorname{er}_{\mathbb{D}}(\mathcal{A}(\overline{z})) - \operatorname{opt}_{\mathbb{D}}(\mathcal{H}) \le \varepsilon \}$, $C \in \Sigma_{\mathcal{Z}}^m$, and $\mathbb{D}^m(C) \ge 1-\delta$, the learning function $\mathcal{A}$ is PAC.
\end{proof}

\subsubsection{Existence of an NMSE Algorithm}

Finally, we prove the implication $(3) \implies (4)$ from Theorem~\ref{thm:ftsl-equivalences}. The statement (3) asserts that if a learning function is NMSE, then it is PAC. To show that the hypothesis space $\mathcal{H}$ is PAC learnable (4), we only need to demonstrate that at least one such NMSE learning function is guaranteed to exist for any hypothesis space.

\begin{proposition}
    Let $\mathcal{X}$ be a non-empty set and let $\emptyset \neq \mathcal{H} \subseteq \{0,1\}^\mathcal{X}$ be a hypothesis space. Then there exists a learning function for $\mathcal{H}$ that minimizes the sample error (SEM) and is therefore NMSE.
\end{proposition}

\begin{proof}
    Let $m \in \mathbb{N}$ and let $\overline{z} \in \mathcal{Z}^m$ be an arbitrary sample. The sample error $\hat{\operatorname{er}}_{\overline{z}}(h)$ can only take values in the finite set $\{0, 1/m, 2/m, \dots, 1\}$. Therefore, the set of all possible sample errors for a given $\overline{z}$, $\{\hat{\operatorname{er}}_{\overline{z}}(h) \mid h \in \mathcal{H}\}$, is a finite subset of these values.

    This means that the infimum of the sample error is always achieved by at least one hypothesis. That is,
    \[
        \operatorname{opt}_{\overline{z}}(\mathcal{H}) = \inf_{h \in \mathcal{H}} \hat{\operatorname{er}}_{\overline{z}}(h) = \min_{h \in \mathcal{H}} \hat{\operatorname{er}}_{\overline{z}}(h).
    \]
    For each sample $\overline{z}$, we can therefore choose a hypothesis $h_{\overline{z}} \in \mathcal{H}$ such that $\hat{\operatorname{er}}_{\overline{z}}(h_{\overline{z}}) = \operatorname{opt}_{\overline{z}}(\mathcal{H})$.

    We define a learning function $\mathcal{A}$ by setting $\mathcal{A}(\overline{z}) = h_{\overline{z}}$. This algorithm is known as \emph{Sample Error Minimization (SEM)}. We now show that it is NMSE. For any $\varepsilon \in (0,1)$, we have:
    \[
        \hat{\operatorname{er}}_{\overline{z}}(\mathcal{A}(\overline{z})) - \operatorname{opt}_{\overline{z}}(\mathcal{H}) = \operatorname{opt}_{\overline{z}}(\mathcal{H}) - \operatorname{opt}_{\overline{z}}(\mathcal{H}) = 0 \le \varepsilon.
    \]
    This inequality holds for any sample size $m \ge 1$. Therefore, the function $\mathcal{A}$ is NMSE.
\end{proof}

Since we have shown that an NMSE learning function always exists, and by the premise of implication (3) this function is PAC, it follows that there exists a PAC learning function for $\mathcal{H}$. This means $\mathcal{H}$ is PAC learnable, completing the proof of $(3) \implies (4)$ and thus the proof of the Fundamental Theorem of Statistical Learning.