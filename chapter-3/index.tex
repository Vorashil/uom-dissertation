\chapter{Learning Problem}

In this chapter, we build upon the discussion of VC-dimension and NIP properties from the previous chapter to explore their role within statistical learning theory. Our aim is to highlight the deep connections between model theory and learning theory, showing how combinatorial properties of definable sets naturally interact with notions of learnability. This chapter closely follows the perspective of Krapp and Wirth \cite{KrappWirth2021}, who provide a measure-theoretic analysis of the Fundamental Theorem of Statistical Learning. Their work presents the measurability assumptions required for rigorous proofs of PAC learnability, and we use their framework to bridge the concepts introduced earlier in model theory with the central results of learning theory.

\section{Learning Framework}

In statistical learning theory, a \emph{learning problem (LP)} provides a formal framework for inferring predictive rules from data~\cite{StatisticalLearningTheory}. This dissertation focuses on the setting of binary classification, where such a rule is a \emph{hypothesis}—a function that assigns a binary label to each data point, or \emph{instance}. The learning framework is defined by four core components: an instance space containing all possible instances, a hypothesis space of candidate functions, a sample space of labeled instances, and a set of probability distributions that model the data-generating process. The ultimate objective is to design a \emph{learning function} that, given a sample of data, selects a hypothesis that generalises well to new, unseen examples.


Formally, a learning problem is given by a tuple
\[
    (\mathcal{X}, \Sigma_{\mathcal{Z}}, \mathcal{D}, \mathcal{H}),
\]
where the components are defined as follows.

\begin{itemize}
    \item $\mathcal{X}$ is a non-empty set, called the \emph{instance space}. Elements of $\mathcal{X}$ represent the objects or inputs under consideration.

    \item $\Sigma_{\mathcal{Z}}$ is a $\sigma$-algebra on the \emph{sample space} $\mathcal{Z}$. The sample space is defined as
    \[
        \mathcal{Z} = \mathcal{X} \times \{0,1\},
    \]
    where $\{0,1\}$ denotes the set of labels.
    We require that $\mathcal{P}_{\mathrm{fin}}(\mathcal{Z}) \subseteq \Sigma_{\mathcal{Z}}$, meaning that every finite subset of $\mathcal{Z}$ is measurable.

    \item $\mathcal{D}$ is a subset of the set $\mathcal{D}^*$, which contains all \emph{probability distributions} defined on the measurable space $(\mathcal{Z}, \Sigma_{\mathcal{Z}})$.

    \item $\mathcal{H}$ is a non-empty set, called the \emph{hypothesis space}, consisting of candidate functions that map inputs to labels:
    \[
        \mathcal{H} \subseteq \{0,1\}^\mathcal{X} = \{h : \mathcal{X} \to \{0,1\}\}.
    \]
    An element $h \in \mathcal{H}$ is called a \emph{hypothesis}.
    The \emph{graph} of $h$ is defined by
    \[
        \Gamma(h) = \{(x,y) \in \mathcal{Z} \mid h(x) = y\} \subseteq \mathcal{Z}.
    \]
\end{itemize}

\medskip

Learning in this framework is based on processing finitely many samples
\[
    (x_1,y_1),\dots,(x_m,y_m) \in \mathcal{Z},
\]
which are assumed to be drawn independently at random according to some distribution $\mathbb{D} \in \mathcal{D}$.
Equivalently, the whole sample may be viewed as an element of $\mathcal{Z}^m$, distributed according to the product measure $\mathbb{D}^m$ on the measurable space $(\mathcal{Z}^m,\Sigma_{\mathcal{Z}}^m)$.

To ensure that such finite samples are measurable, it is natural to require that singletons $\{z\}$ are measurable for every $z \in \mathcal{Z}$. Indeed, measurability of singletons implies that every finite set of sample points is measurable, since finite unions of measurable sets remain measurable. This condition is automatically satisfied, for example, when $\mathcal{Z}$ is a Hausdorff space equipped with its Borel $\sigma$-algebra.

\begin{remark}
    \label{rem:hyposthesis-is-measurable}
    In this chapter, we assume that for every hypothesis $h \in \mathcal{H}$, its graph is measurable, i.e.
    \[
        \Gamma(h) \in \Sigma_{\mathcal{Z}}.
    \]
    This assumption will help us to measure the probability of a hypothesis $h \in \mathcal{H}$ to misclassify instances in~\ref{subsec:evaluating-hypothesis}.
\end{remark}

\begin{remarknl}[Agnostic vs. deterministic learning models]
    In our framework, probability distributions $\mathbb{D}$ are defined on the sample space $\mathcal{Z} = \mathcal{X}\times\{0,1\}$.
    This means that, for a fixed instance $x \in \mathcal{X}$, we have $(x,0)$ and $(x,1)$ as separate sample points, and it's possible that they both have non-zero probability simultanously according to some distribution defined on $(\mathcal{Z}, \Sigma_{\mathcal{Z}})$.
    In other words, the label of $x$ is itself random and not determined by a fixed target function.
    This setting is known as the \emph{agnostic}~\cite[p.45, \S 3.2.1]{UnderstandinMachineLearning}.
    It allows us to capture noisy labels and corrupted data, reflecting the fact that real-world classification tasks may not admit a perfect underlying rule.

    By contrast, in the \emph{deterministic model}~\cite{LearnabilityDeterministic}, the distribution $\mathbb{P}$ is only placed on the instance space $\mathcal{X}$ itself.
    The labels are then determined by a fixed but unknown target function $t \in \{0,1\}^\mathcal{X}$.
    Thus the data are of the form $(x_i,t(x_i))$, where the $x_i$ are drawn according to some probability distribution defined on $\sigma$-algebra on $\mathcal{X}$.

    The distinction between these two paradigms lies in whether label uncertainty is treated as intrinsic (agnostic case) or entirely due to the randomness of instance selection (deterministic case).
\end{remarknl}

\subsection{Evaluating a Learned Hypothesis}\label{subsec:evaluating-hypothesis}

We now make the discussion of learning more rigorous and to introduce the central notions needed for the formal study of Probably Approximately Correct (PAC) learning. PAC learning is not itself a learning function—it is a framework that specifies conditions under which any learning function succeeds. To arrive at this definition, we first need a precise way of evaluating the quality of hypotheses chosen by a learning function. In particular, we distinguish between the \emph{true error}, which measures performance with respect to the underlying distribution, and the \emph{sample error}, which measures performance on a finite dataset. These notions allow us to formalise the idea that a hypothesis may look good on the training data yet perform poorly on unseen examples. We also define the notion of \emph{optimal error}, which represents the best performance achievable within a given hypothesis space. Together, these definitions provide the language required to analyse concrete learning strategies. As an illustration, we will consider the sample error minimisation (SEM~\cite{KrappWirth2021}) framework (also known as empirical risk minimisation (ERM) in~\cite{UnderstandinMachineLearning}), where the learning function selects the hypothesis that minimises the sample error. While intuitive and computationally straightforward, this strategy highlights the gap between sample and true performance, motivating the need for stronger guarantees such as those provided by the PAC framework, which will be developed in the following sections.


%\begin{example}
%    \label{ex:learning-function}
%    Let $\mathcal{X}=\{a,b,c\}$ be a finite instance space, so the sample space is $\mathcal{Z}=\mathcal{X}\times\{0,1\}$.
%    The hypothesis space $\mathcal{H}$ consists of all functions $h:\mathcal{X}\to\{0,1\}$, of which there are $2^{|\mathcal{X}|}=8$ in total.
%    For illustration, we display a few of them in the table below:
%
%    \[
%        \begin{array}{c|ccc}
%            h & a & b & c \\
%            \hline
%            h_0 & 0 & 0 & 0 \\
%            h_1 & 1 & 1 & 1 \\
%            h_{ab} & 0 & 0 & 1 \\
%            h_{bc} & 1 & 0 & 0 \\
%        \end{array}
%    \]
%
%    Suppose we observe a finite sample
%    \[
%        \overline{z} = \bigl((a,0),(b,0),(c,1)\bigr) \in \mathcal{Z}^3,
%    \]
%    which is drawn according to some distribution $\mathbb{D}\in\mathcal{D}$.
%    A learning function $\mathcal{A}$ then selects a hypothesis from $\mathcal{H}$ that is consistent with this sample.
%    In this case, $\mathcal{A}(\overline{z})=h_{ab}$ would be a natural choice, since $h_{ab}$ agrees with the observed labels on $a$, $b$, and $c$.
%
%\end{example}
%
%This example~\ref{ex:learning-function} illustrates the general mechanism of learning: a learning function $\mathcal{A}$ takes a finite sample and outputs a hypothesis $h\in\mathcal{H}$.


%\subsection{Error of a Hypothesis}\label{subsec:error-of-hypothesis}

\subsubsection{Performance of a hypothesis}

The performance of a hypothesis is evaluated by how well it predicts labels of unseen data drawn from the distribution.
This is formalised through the notion of \emph{error}.
We distinguish between the \emph{true error}, which depends on the underlying distribution, and the \emph{sample error}, which is based only on the observed data. Recall that, by our assumption in the remark~\ref{rem:hyposthesis-is-measurable}, the graph of a hypothesis is measurable. This is required for the definition of true error to be well-defined.

\begin{definition}[True error]
    Let $\mathbb{D} \in \mathcal{D}$ be a distribution on $(\mathcal{Z},\Sigma_{\mathcal{Z}})$
    and let $h \in \mathcal{H}$ be a hypothesis.
    The \emph{true error} of $h$ with respect to $\mathbb{D}$ is defined as
    \[
        \operatorname{er}_{\mathbb{D}}(h) \coloneqq \mathbb{D}(\{(x,y)\in\mathcal{Z} \mid h(x)\neq y\}) = \mathbb{D}(\mathcal{Z}\setminus\Gamma(h)).
    \]
\end{definition}

%\noindent
%Thus, $\operatorname{er}_{\mathbb{D}}(h)$ is the probability of misclassification under the distribution $\mathbb{D}$.
%It quantifies the long-run frequency with which $h$ makes mistakes if we were to draw infinitely many samples.

\medskip

The following definition, invokes the other assumption about singletons $\{z\}\in \mathcal{Z}$ being measurable.

\begin{definition}[Sample error]
    Let $z = (z_1,\dots,z_m)\in\mathcal{Z}^m$ be a finite sample and $h\in\mathcal{H}$.
    The \emph{sample error} (also called \emph{empirical error}) of $h$ on $z$ is defined as
    \[
        \hat{\operatorname{er}}_{z}(h) := \frac{1}{m}\sum_{i=1}^m \ell(h,z_i),
    \]
    where the \emph{loss function} $\ell:\mathcal{H}\times \mathcal{Z}\to\{0,1\}$ is given by
    \[
        \ell(h,(x,y)) =
        \begin{cases}
            1 & \text{if } h(x)\neq y, \\
            0 & \text{otherwise}.
        \end{cases}
    \]
\end{definition}

The following lemma, shows that the sample error of a hypothesis $h \in \mathcal{H}$ is measurable function, assuming $\Gamma(h) \in \Sigma_{\mathcal{Z}}$.

\begin{lemma}[Measurability of the Sample Error Map]
    \label{lem:sample-error-measurable}
    Let $h \in \mathcal{H}$ be a hypothesis with a measurable graph $\Gamma(h)$. The sample error map,
    \[
        z \mapsto \hat{\operatorname{er}}_{z}(h),
    \]
    is a $\Sigma_{\mathcal{Z}}^m$-measurable function from $\mathcal{Z}^m$ to $[0,1]$.
\end{lemma}

\begin{proof}

    The sample error map $z \mapsto \frac{1}{m}\sum_{i=1}^m \ell(h,z_i)$ is a linear combination of the functions $f_i(z) = \ell(h,z_i)$. Each function $f_i$ is the composition $\ell(h, \cdot) \circ \pi_i$ of the loss function with the $i$-th coordinate projection map $\pi_i: \mathcal{Z}^m \to \mathcal{Z}$.

    The loss function $\ell(h, \cdot)$ is measurable because it is the indicator of the measurable set $\mathcal{Z}\setminus\Gamma(h)$. The projection map $\pi_i$ is measurable by the definition of the product $\sigma$-algebra $\Sigma_{\mathcal{Z}}^m$. The composition of measurable functions is measurable, and so is a finite sum of such functions~\ref{subsec:measurable-functions}. Therefore, the sample error map is measurable~\cite[Prop 2.4]{FollandRealAnalysis}.
    \qedhere
\end{proof}

The sample error $\hat{\operatorname{er}}_z(h)$ measures how often $h$ disagrees with the observed labels in the dataset $\overline{z} \in \mathcal{Z}$. Unlike the true error, it is always computable from the available data. The natural question is how the sample error relates to the true error. The next lemma establishes a basic relationship: the sample error is an unbiased estimator of the true error.

\begin{lemma}[Sample error is an unbiased estimator]
    \label{lem:sample-error-unbiased}
    Let $h \in \mathcal{H}$ be a hypothesis with a measurable graph $\Gamma(h)$. For a sample $z = (z_1, \dots, z_m)$ drawn according to the product distribution $\mathbb{D}^m$, the expected value of the sample error equals the true error:
    \[
        \mathbb{E}_{z\sim\mathbb{D}^m}\bigl[\hat{\operatorname{er}}_{z}(h)\bigr] = \operatorname{er}_{\mathbb{D}}(h).
    \]
\end{lemma}

\begin{proof}

    By linearity of expectation and the definition of sample error, we have
    \[
        \mathbb{E}_{z\sim\mathbb{D}^m}\bigl[\hat{\operatorname{er}}_{z}(h)\bigr] = \frac{1}{m}\sum_{i=1}^m \mathbb{E}_{z_i\sim\mathbb{D}}\bigl[\ell(h,z_i)\bigr].
    \]
    For any $i \in \{1, \dots, m\}$, this expectation is taken over the single-variable distribution $\mathbb{D}$:
    \[
        \mathbb{E}_{z_i\sim\mathbb{D}}\bigl[\ell(h,\cdot)\bigr] = \mathbb{E}_{z_i\sim\mathbb{D}}\bigl[\mathds{1}_{\mathcal{Z}\setminus\Gamma(h)}\bigr] = \mathbb{D}(\mathcal{Z}\setminus\Gamma(h))=\operatorname{er}_{\mathbb{D}}(h).
    \]
    Substituting this into the sum completes the proof:
    \[
        \mathbb{E}_{z\sim\mathbb{D}^m}\bigl[\hat{\operatorname{er}}_{z}(h)\bigr] = \frac{1}{m}\sum_{i=1}^m \operatorname{er}_{\mathbb{D}}(h) = \operatorname{er}_{\mathbb{D}}(h).
    \]
    \qedhere
\end{proof}

\begin{remarknl}[Why unbiasedness is not enough for learning]
    \label{rem:unbiased-not-enough}
    
    While Lemma~\ref{lem:sample-error-unbiased} shows that the sample error is an unbiased estimator—meaning it is correct "on average"—this property alone is insufficient for learning. To understand why, consider the following critical points:
    
    First, the unbiasedness property holds for each hypothesis $h$ individually. When we fix a specific $h$ and draw many independent samples, the average of their sample errors converges to the true error by the weak law of large numbers~\cite[Chap.~9]{MeasureTheoryLeGall}. However, in learning, we do not fix $h$ in advance. Instead, a learning algorithm selects $h$ based on the observed data, typically by choosing the hypothesis with the smallest sample error.
    
    This data-dependent selection creates a fundamental problem: the hypothesis that looks best on the sample may have been "lucky" with that particular dataset. For instance, consider a hypothesis $h^*$ that happens to correctly classify all points in our sample purely by chance, yielding $\hat{\operatorname{er}}_z(h^*) = 0$. While the expected value $\mathbb{E}[\hat{\operatorname{er}}_z(h^*)]$ equals the true error $\operatorname{er}_{\mathbb{D}}(h^*)$, for this specific sample $z$, the gap $|\hat{\operatorname{er}}_z(h^*) - \operatorname{er}_{\mathbb{D}}(h^*)|$ could be large.
    
    The problem becomes more severe when we have many hypotheses. With a large hypothesis space $\mathcal{H}$, it becomes increasingly likely that at least one hypothesis will appear misleadingly good on any given sample. This is why we need a stronger property that controls the approximation error simultaneously for all hypotheses—this is precisely what uniform convergence provides, as we will see in Section~\ref{sec:pac-learning}.
\end{remarknl}

\begin{definition}[Optimal error]
    The \emph{optimal true error} achievable by the hypothesis space $\mathcal{H}$ under $\mathbb{D}$ is
    \[
        \operatorname{opt}_{\mathbb{D}}(\mathcal{H}) := \inf_{h\in\mathcal{H}} \operatorname{er}_{\mathbb{D}}(h).
    \]
    Similarly, given a finite sample $z \in \mathcal{Z}^m$, the \emph{optimal sample error} is
    \[
        \operatorname{opt}_{z}(\mathcal{H}) := \min_{h\in\mathcal{H}} \hat{\operatorname{er}}_{z}(h).
    \]
\end{definition}

\medskip

The true error is the performance of a hypothesis on the underlying distribution, but it is typically inaccessible.
The sample error, by contrast, is observable but may deviate from the true error due to randomness in sampling.
The optimal errors describe the best performance achievable by any hypothesis in $\mathcal{H}$, either with respect to $\mathbb{D}$ (idealised) or with respect to the observed dataset (computable).

\medskip

Formally, the learning function $\mathcal{A}: \cup_{n \in \mathbb{N}} \mathcal{Z}^n \to \mathcal{H}$ takes the finite sample $\overline{z} \in \mathcal{Z}^m$ and returns a candidate hypothesis $\mathcal{A}(\overline{z}) \in \mathcal{H}$ that should explain the observed labels and hopefully predict new ones correctly. What distinguishes the PAC viewpoint is a quantitative requirement: for given $\varepsilon,\delta\in(0,1)$, if the sample size $m$ is large enough, then with probability at least $1-\delta$ (over the random draw of the sample) the hypothesis $\mathcal{A}(\overline{z})$ misclassifies new points with probability at most $\varepsilon$. Here$\varepsilon$ is the \emph{accuracy parameter} (“approximately”),
while $\delta$ is the \emph{confidence parameter} (“probably”).~\cite[Sec 3.1, p.43]{UnderstandinMachineLearning}
A fully formal definition of PAC learning will be given in~\ref{sec:pac-learning} and will be central to the rest of the dissertation.






\section{PAC Learnability and the Fundamental Theorem}

In this section, we formalise the notion of learnability through the \emph{Probably Approximately Correct (PAC)} framework. This model, first introduced by Valiant~\cite{Valiant1984} and later extended to the agnostic setting, provides a rigorous definition for what it means for a learning algorithm to succeed. We will then state the Fundamental Theorem of Statistical Learning, which provides a remarkable equivalence between PAC learnability and the combinatorial property of having a finite VC dimension.

\subsection{Probably Approximately Correct (PAC) Learning}

We begin with the formal definition of PAC learnability, which captures the requirements for a learning function to generalise well from a finite sample.

\begin{definition}[PAC Learnability]
    \label{def:pac-learnability}
    A learning function $\mathcal{A}$ for $\mathcal{H}$ is called \emph{probably approximately correct (PAC)} (with respect to $\mathcal{D}$) if it satisfies the following condition:
    \begin{quote}
        For any $\varepsilon, \delta \in (0,1)$ there exists $m_0 = m_0(\varepsilon, \delta) \in \mathbb{N}$ such that for any $m \ge m_0$ and any $\mathbb{D} \in \mathcal{D}$ there exists a set $C = C(\varepsilon, \delta, m, \mathbb{D}) \in \Sigma_{\mathcal{Z}}^m$ such that
        \[
            C \subseteq \left\{ \overline{z} \in \mathcal{Z}^m \mid \operatorname{er}_{\mathbb{D}}(\mathcal{A}(\overline{z})) - \operatorname{opt}_{\mathbb{D}}(\mathcal{H}) \le \varepsilon \right\}
        \]
        and $\mathbb{D}^m(C) \ge 1 - \delta$.
    \end{quote}
    The hypothesis space $\mathcal{H}$ is called \emph{PAC learnable} (with respect to $\mathcal{D}$) if there exists a learning function for $\mathcal{H}$ that is PAC with respect to $\mathcal{D}$.
\end{definition}

In essence, PAC learnability guarantees that a learning algorithm can, with high probability ($1-\delta$), find a hypothesis whose performance is close (within $\varepsilon$) to the best possible performance within the hypothesis class, provided it is given a sufficiently large sample. The term $\operatorname{er}_{\mathbb{D}}(\mathcal{A}(\overline{z})) - \operatorname{opt}_{\mathbb{D}}(\mathcal{H})$ is the \emph{excess error}, which measures how much worse our learned hypothesis is compared to the optimal one in $\mathcal{H}$. The definition is robust as it must hold for any data distribution $\mathbb{D} \in \mathcal{D}$.

\subsection{The Fundamental Theorem of Statistical Learning}

Having defined what it means for a hypothesis space to be learnable, we now turn to the central result that characterizes it. The Fundamental Theorem of Statistical Learning establishes a deep connection between the statistical property of PAC learnability and the combinatorial property of finite VC dimension. The theorem asserts that, under certain regularity conditions, a hypothesis space is PAC learnable if and only if its VC dimension is finite.

Our focus will be on proving one direction of this equivalence: that PAC learnability implies a finite VC dimension. For the converse—that a finite VC dimension guarantees PAC learnability—we will provide a high-level sketch of the argument, which relies on the \emph{uniform convergence property}, and refer the interested reader to foundational texts like~\cite{AnthonyBartlett1999} or~\cite{UnderstandinMachineLearning} for a complete proof.

Before stating the theorem, we must introduce two key concepts: a specific class of distributions needed for the proof and a technical condition on the hypothesis space.

\begin{definition}[Discrete Uniform Distribution]
    Let $(\Omega, \Sigma)$ be a measurable space. A probability measure $\mathbb{P}: \Sigma \to [0,1]$ is a \emph{discrete uniform distribution} if it is of the form
    \[
        \mathbb{P} = \frac{1}{\ell} \sum_{j=1}^{\ell} \delta_{\omega_j},
    \]
    where $\ell \in \mathbb{N}$ and $\omega_1, \dots, \omega_\ell \in \Omega$ are distinct points. Here, $\delta_{\omega_j}$ denotes the Dirac measure concentrated at $\omega_j$.
\end{definition}

This definition formalises the notion of a uniform distribution on a finite set of points, as introduced in Example~\ref{ex:classical-distributions}, using the language of Dirac measures. The significance of this type of distribution lies in its role as a "test case" for learnability. The proof of the Fundamental Theorem, particularly the direction we will demonstrate, relies on constructing specific probability distributions to challenge a learning algorithm. Discrete uniform distributions are ideal for this purpose because they allow us to focus all the probability mass on a carefully chosen finite set of points. If a hypothesis space can be learned, it must be able to handle these simple yet potentially adversarial scenarios. Therefore, the statement of the theorem will require that the set of distributions $\mathcal{D}$ is rich enough to contain all such discrete uniform distributions.

Next, we introduce a set of technical measurability conditions on the hypothesis space, which we bundle under the term "well-behaved". These conditions are essential for the rigorous development of the theory, ensuring that the key quantities involved in the proofs are mathematically well-defined.

\begin{definition}[Well-Behaved Hypothesis Space]
    \label{def:well-behaved}
    A hypothesis space $\emptyset \neq \mathcal{H} \subseteq \{0,1\}^\mathcal{X}$ is called \emph{well-behaved} (with respect to $\mathcal{D}$) if it satisfies the following conditions:
    \begin{enumerate}
        \item For every hypothesis $h \in \mathcal{H}$, its graph $\Gamma(h)$ is measurable.
        \item There exists an integer $m_{\mathcal{H}} \in \mathbb{N}$ such that for all $m \ge m_{\mathcal{H}}$:
        \begin{itemize}
            \item The map $V = V(\mathcal{H}, m): \mathcal{Z}^{2m} \to [0,1]$, defined by
            \[
                (\overline{z}, \overline{z}') \mapsto \sup_{h \in \mathcal{H}} |\hat{\operatorname{er}}_{\overline{z}}(h) - \hat{\operatorname{er}}_{\overline{z}'}(h)|,
            \]
            is $\Sigma_{\mathcal{Z}}^{2m}$-measurable.
            \item The map $U = U(\mathcal{H}, m, \mathbb{D}): \mathcal{Z}^m \to [0,1]$, defined by
            \[
                \overline{z} \mapsto \sup_{h \in \mathcal{H}} |\operatorname{er}_{\mathbb{D}}(h) - \hat{\operatorname{er}}_{\overline{z}}(h)|,
            \]
            is $\Sigma_{\mathcal{Z}}^m$-measurable for any $\mathbb{D} \in \mathcal{D}$.
        \end{itemize}
    \end{enumerate}
\end{definition}

The well-behavedness condition is a technical prerequisite for the Fundamental Theorem. The measurability of the map $U$, which captures the worst-case deviation between true and sample error, is essential for proving generalization bounds. The measurability of the map $V$ is crucial for a standard proof technique known as \emph{symmetrization}, which is used to show that a finite VC dimension implies PAC learnability (the direction we do not prove here). While technical, these conditions are satisfied by most hypothesis spaces encountered in practice.

We are now ready to state the main theorem of this chapter, which brings together all the concepts we have introduced.

\begin{theorem}[Fundamental Theorem of Statistical Learning]
    \label{thm:fundamental-theorem}
    Let $\mathcal{X}$ be a non-empty set, let $\Sigma_{\mathcal{Z}}$ be a $\sigma$-algebra on $\mathcal{Z} = \mathcal{X} \times \{0,1\}$ with $\mathcal{P}_{\mathrm{fin}}(\mathcal{Z}) \subseteq \Sigma_{\mathcal{Z}}$ and let $\mathcal{D}$ be a set of distributions on $(\mathcal{Z},\Sigma_{\mathcal{Z}})$ containing all discrete uniform distributions. Further, let $\emptyset \neq \mathcal{H} \subseteq \{0,1\}^\mathcal{X}$ be a hypothesis space that is well-behaved with respect to $\mathcal{D}$. Then $\mathcal{H}$ is PAC learnable with respect to $\mathcal{D}$ if and only if $\operatorname{vc}(\mathcal{H}) < \infty$.
\end{theorem}

This theorem provides a complete characterization of learnability in the PAC model. It tells us that the ability to learn from data is equivalent to a purely combinatorial property of the hypothesis space: having a finite VC dimension. In the next section, we will prove the "only if" direction of this theorem.
\section{Proof of the Fundamental Theorem of Statistical Learning}

In this section, we provide the proof for the "only if" direction of the Fundamental Theorem of Statistical Learning (\ref{thm:fundamental-theorem}). We will then briefly outline the argument for the converse, which relies on the Uniform Convergence Property.

\subsection{PAC Learnability Implies Finite VC Dimension}

We will prove the following statement, which is one half of the Fundamental Theorem.

\begin{theorem}
    Let the learning framework $(\mathcal{X}, \Sigma_{\mathcal{Z}}, \mathcal{D}, \mathcal{H})$ be given as in Theorem~\ref{thm:fundamental-theorem}. If $\mathcal{H}$ is PAC learnable, then $\operatorname{vc}(\mathcal{H}) < \infty$.
\end{theorem}

\begin{proof}

    The proof proceeds by contrapositive. We assume that $\mathcal{H}$ has an infinite VC dimension and show that it cannot be PAC learnable. To show that $\mathcal{H}$ is not PAC learnable, we must negate its definition. That is, we must show that there exist fixed accuracy and confidence parameters, $\varepsilon > 0$ and $\delta > 0$, such that for any sample size $m \in \mathbb{N}$, there exists a distribution $\mathbb{D} \in \mathcal{D}$ for which any learning algorithm $\mathcal{A}$ fails to meet the PAC guarantee. Specifically, the probability of the algorithm returning a hypothesis with an excess error greater than $\varepsilon$ will be at least $\delta$.

    Let us fix the PAC parameters $\varepsilon = 1/8$ and $\delta = 1/7$. Let $\mathcal{A}$ be an arbitrary learning function and $m \in \mathbb{N}$ be any sample size. Our goal is to find a distribution $\mathbb{D} \in \mathcal{D}$ such that the hypothesis $\mathcal{A}(\overline{z})$ returned by the algorithm on a sample $\overline{z} \sim \mathbb{D}^m$ fails the PAC condition, i.e.,
    \[
        \mathbb{D}^m\left(\left\{ \overline{z} \in \mathcal{Z}^m \mid \operatorname{er}_{\mathbb{D}}(\mathcal{A}(\overline{z})) - \operatorname{opt}_{\mathbb{D}}(\mathcal{H}) > \frac{1}{8} \right\}\right) \ge \frac{1}{7}.
    \]
    This will demonstrate that no sample size $m_0$ can satisfy the PAC definition for these fixed $\varepsilon$ and $\delta$, proving that $\mathcal{H}$ is not PAC learnable.

    First, we use the assumption of an infinite VC dimension to construct a specific learning problem that is designed to be difficult for any algorithm. Since $\operatorname{vc}(\mathcal{H}) = \infty$, there exists a set of any finite size that can be shattered by $\mathcal{H}$. We choose a set $S = \{x_1, \dots, x_{2m}\} \subseteq \mathcal{X}$ of size $2m$ that is shattered by $\mathcal{H}$.

    Let $\mathcal{F} = \{0,1\}^S$ be the set of all $T \coloneq 2^{2m}$ possible binary functions on $S$. We index each function in $\mathcal{F}$ as
    \[
        \{0,1\}^S = \{f_1, \dots, f_T\}.
    \]
    By the definition of shattering, for each function $f_i \in \mathcal{F}$, there exists a hypothesis $h_i \in \mathcal{H}$ such that its restriction to $S$ is exactly $f_i$, i.e., $h_i|_S = f_i$.

    Next, we use this shattered set to define a family of probability distributions. Let $\mathcal{Z}_S = S \times \{0,1\}$ be the finite sample space restricted to the shattered set $S$. For each $1 \leq i \leq T$, we define a distribution $\mathbb{D}_i$ on the measurable space $(\mathcal{Z}_S, \mathcal{P}(\mathcal{Z}_S))$. This distribution's probability mass function is given by:
    \[
        \mathbb{D}_i(\{z\}) =
        \begin{cases}
            \frac{1}{2m} & \text{if } z \in \Gamma(f_i), \\
            0 & \text{otherwise},
        \end{cases}
    \]
    where $\Gamma(f_i) = \{(x, f_i(x)) \mid x \in S\}$ is the graph of the function $f$ over the set $S$. In other words, $\mathbb{D}_i$ is the discrete uniform distribution on the $2m$ points that constitute the graph of $f_i$. Note that, by definition, $\operatorname{er}_{\mathbb{D}_i}(f_i) = 0$ for all $i \in \{1, \dots, T\}$.

    The PAC condition for our chosen $\varepsilon=1/8$ and a distribution $\mathbb{D}_i$ simplifies to requiring that $\operatorname{er}_{\mathbb{D}_i}(\mathcal{A}(\overline{z})) \le 1/8$. Our strategy is to show that no single algorithm $\mathcal{A}$ can perform well simultaneously for all these distributions. We will show that there is at least one function $f \in \mathcal{F}$ for which the algorithm's expected error is high. The core of the argument is that a sample of size $m$ is insufficient to distinguish between the $2^{2m}$ possible underlying functions, as it is likely to reveal the labels of at most $m$ of the $2m$ points in $S$.

    The following lemma formalises this idea. It states that there must be at least one distribution $\mathbb{D}_j$ for which the learning algorithm $\mathcal{A}$ has a high expected error, where $j \in \{1, \dots, T\}$.

    \begin{lemma}
        \label{lem:no-free-lunch}
        Let $\mathcal{A}_S(\overline{z}) = \mathcal{A}(\overline{z})|_S$ be the restriction of the learned hypothesis to the set $S$. Then the following inequality holds:
        \[
            \max_{1 \leq i \leq T} \Bigl( \mathbb{E}_{\overline{z} \sim \mathbb{D}_i^m} \left[ \operatorname{er}_{\mathbb{D}_i}(\mathcal{A}_S(\overline{z})) \right]\Bigr) \ge \frac{1}{4}.
        \]
    \end{lemma}

    \begin{subproof}[Proof of Lemma~\ref{lem:no-free-lunch}]

        Note that, $\operatorname{er}_{\mathbb{D}_i}(\mathcal{A}(\square)): \mathcal{Z}_S^m\to [0,1]$ is positive and the bounded random variable. Hence, by the definition of well-definedness of expected value~\cite[Def 8.3]{MeasureTheoryLeGall},
        \[
            \mathbb{E}_{\overline{z} \sim \mathbb{D}_i^m}(\operatorname{er}_{\mathbb{D}_i}(\mathcal{A}(\square)))
        \]
        is well-defined.

        Moreover, the maximum of a set of values is always greater than or equal to their average. Therefore, we can find lower-bound for the maximum expected error by the average expected error taken over all $1 \leq i \leq T$:
        \[
            \max_{1 \leq i \leq T} \Bigl( \mathbb{E}_{\overline{z} \sim \mathbb{D}_i^m} \left[ \operatorname{er}_{\mathbb{D}_i}(\mathcal{A}_S(\overline{z})) \right]\Bigr) \ge \frac{1}{T} \sum_{i = 1}^{T} \mathbb{E}_{\overline{z} \sim \mathbb{D}_i^m} \left[ \operatorname{er}_{\mathbb{D}_i}(\mathcal{A}_S(\overline{z})) \right].
        \]
        We first analyze the average on the right-hand side. A sample $\overline{z}$ drawn from $\mathbb{D}_i^m$ consists of $m$ pairs $(x,y)$ where each $x$ is drawn uniformly from $S$ and $y=f_i(x)$. Let $S^m$ denote the set of all possible sequences of $m$ instances drawn from $S$ and $|S^m| = (2m)^m$. Note that, we draw these $m$ instances from $S$ with replacements, hence some $m$-tuples in $S^m$ contain duplicate instances. This observation will be used later in the proof.

        For any sequence of instances $\overline{x} = (x_{j_1}, \dots, x_{j_m}) \in S^m$, denote the corresponding labeled sample under $f_i\in \mathcal{F}$ by
        \[
            \overline{z}_{\overline{x}}^{i} \coloneq ((x_{j_1}, f_i(x_{j_1})), \dots, (x_{j_m}, f_i(x_{j_m}))).
        \]
        The probability of drawing this specific sample is $\mathbb{D}_i^m(\{\overline{z}_{\overline{x}}^{i}\}) = (1/2m)^m$. Therefore,
        \[
            \mathbb{E}_{\overline{z} \sim \mathbb{D}_i^m} \left[ \operatorname{er}_{\mathbb{D}_i}(\mathcal{A}_S(\overline{z})) \right] = \sum_{\overline{x} \in S^m} \frac{1}{(2m)^m} \operatorname{er}_{\mathbb{D}_i}(\mathcal{A}_S(\overline{z}_{\overline{x}}^{i})))
        \]
        for all $i \in \{1, \dots, T\}$.

        We can now rewrite the expectation as a sum over all possible instance sequences $\overline{x} \in S^m$ and then swap the order of summation:
        \begin{align*}
            \frac{1}{T} \sum_{i=1}^{T} \mathbb{E}_{\overline{z} \sim \mathbb{D}_i^m} \left[ \operatorname{er}_{\mathbb{D}_i}(\mathcal{A}_S(\overline{z})) \right] &= \frac{1}{T}  \sum_{i=1}^{T} \left( \sum_{\overline{x} \in S^m} \frac{1}{(2m)^m} \operatorname{er}_{\mathbb{D}_i}(\mathcal{A}_S(\overline{z}_{\overline{x}}^{i}))) \right) \\
            &= \frac{1}{(2m)^m} \sum_{\overline{x} \in S^m} \left( \frac{1}{T} \sum_{i=1}^{T} \operatorname{er}_{\mathbb{D}_i}(\mathcal{A}_S(\overline{z}_{\overline{x}}^{i})) \right). \\
            &\geq \min_{\overline{x} \in S^m} \frac{1}{T}  \sum_{i=1}^{T} \operatorname{er}_{\mathbb{D}_i}(\mathcal{A}_S(\overline{z}_{\overline{x}}^{i}))
        \end{align*}

        Fix an arbitrary sequence of instances $\overline{x}'=(x_1',\dots, x_m') \in S^m$. Let
        \[
            \{v_1, \dots, v_p\} \coloneq S \setminus \{x_1', \dots, x_m'\}.
        \]
        Since, $\overline{x}'$ might contain some duplicates, $|S| = 2m \leq p \leq m$. Then for any $f \in \mathcal{F}$ and $i \in \{1, \dots, T\}$, the true error of $f$ with respect to $\mathbb{D}_i$ is given by the probability of $f(x) \neq f_i(x)$ for $x \in S$:
        \[
            \operatorname{er}_{\mathbb{D}_i}(f) = \mathbb{D}_i(\mathcal{Z}_S\setminus\Gamma(f)) = \frac{1}{2m}\sum_{x \in S}\mathds{1}_{\mathcal{Z}_S\setminus\Gamma(f_i)}(x, f(x)).
        \]
        Since, $\mathds{1}_{\mathcal{Z}_S\setminus\Gamma(f_i)}(x, f(x)) \geq 0$ for all $x \in S$ and $\{v_1, \dots, v_p\} \subseteq S$, we get
        \[
            \operatorname{er}_{\mathbb{D}_i}(f) \geq \frac{1}{2m}\sum_{r=1}^p \mathds{1}_{\mathcal{Z}_S\setminus\Gamma(f_i)}(v_r, f(v_r)).
        \]
        By using, $\frac{1}{m} \geq \frac{1}{p}$,
        \[
            \operatorname{er}_{\mathbb{D}_i}(f) \geq \frac{1}{2p}\sum_{r=1}^p \mathds{1}_{\mathcal{Z}_S\setminus\Gamma(f_i)}(v_r, f(v_r)).
        \]
        For the rest of the proof, we denote $h' \coloneq \mathcal{A}_S(\overline{z}_{\overline{x}'}^{i})$ for our fixed $\overline{x}'$ for notational clarity.
        \[
            \begin{aligned}
                &\frac{1}{T}\sum_{i=1}^T\operatorname{er}_{\mathbb{D}_i}(h') \\
                &\geq \frac{1}{T}\sum_{i=1}^T \Bigl(\frac{1}{2p}\sum_{r=1}^p \mathds{1}_{\mathcal{Z}_S\setminus\Gamma(f_i)}(v_r, h'(v_r))\Bigr) \\
                &= \frac{1}{2}\Biggl(\frac{1}{p}\sum_{r=1}^p \Bigl(\frac{1}{T}\sum_{i=1}^T \mathds{1}_{\mathcal{Z}_S\setminus\Gamma(f_i)}(v_r, h'(v_r))\Bigr)\Biggr) \\
                &\geq \frac{1}{2} \min_{1 \leq r \leq p}\Bigl(\frac{1}{T}\sum_{i=1}^T \mathds{1}_{\mathcal{Z}_S\setminus\Gamma(f_i)}(v_r, h'(v_r))\Bigr),
            \end{aligned}
        \]
        where the last inequality follows from the basic fact that minimum of finite set of numbers is bounded above by the average of those numbers.
        Now, we fix an arbitrary $r \in \{1, \dots, p\}$ and evaluate the term inside the minimum:
        \[
            \frac{1}{T}\sum_{i=1}^T \mathds{1}_{\mathcal{Z}_S\setminus\Gamma(f_i)}(v_r, h'(v_r)).
        \]
        The key insight is that the hypothesis $h'$ depends on the sample $\overline{z}_{\overline{x}'}^{i}$, which in turn depends on the labels that $f_i$ assigns to the points in $\overline{x}'$. The point $v_r$ is not in the sequence $\overline{x}'$.

        We can partition the set of all functions $\mathcal{F}=\{f_1, \dots, f_T\}$ into $T/2$ disjoint pairs. Let $(f_j, f_k)$ be such a pair, constructed so that they differ only at the point $v_r$:
        \[
            f_j(x) \neq f_k(x) \iff x = v_r.
        \]
        Since $f_j$ and $f_k$ agree on all points in $\overline{x}'$, the labeled samples they generate are identical, i.e., $\overline{z}_{\overline{x}'}^{j} = \overline{z}_{\overline{x}'}^{k}$. As the learning algorithm $\mathcal{A}$ is a deterministic function, it must produce the same hypothesis for both samples. Thus, the hypothesis we denoted $h'$ is the same for both $f_j$ and $f_k$.

        This common hypothesis $h'$ makes a prediction $h'(v_r)$. Since $f_j(v_r) \neq f_k(v_r)$, $h'(v_r)$ must be different from exactly one of them. This means that for each pair, exactly one of the hypotheses is wrong at $v_r$:
        \[
            \mathds{1}_{\mathcal{Z}_S\setminus\Gamma(f_j)}(v_r, h'(v_r)) + \mathds{1}_{\mathcal{Z}_S\setminus\Gamma(f_k)}(v_r, h'(v_r)) = 1.
        \]
        Summing over all $T/2$ such pairs, the total sum of the indicators over all $T$ functions is exactly $T/2$. The average is therefore:
        \[
            \frac{1}{T}\sum_{i=1}^T \mathds{1}_{\mathcal{Z}_S\setminus\Gamma(f_i)}(v_r, h'(v_r)) = \frac{T/2}{T} = \frac{1}{2}.
        \]
        Since this holds for any $r$, the minimum over $r$ is also $1/2$. Substituting this back into our main inequality, we get:
        \[
            \frac{1}{T}\sum_{i=1}^T\operatorname{er}_{\mathbb{D}_i}(h') \geq \frac{1}{2} \cdot \frac{1}{2} = \frac{1}{4}.
        \]
        This lower bound holds for any sequence $\overline{x}' \in S^m$, so the minimum over all such sequences is also at least $1/4$. This completes the proof of the lemma.
    \end{subproof}

    With Lemma~\ref{lem:no-free-lunch} established, we can fix an index $j \in \{1, \dots, T\}$ such that
    \begin{equation}
        \label{eq:fixed-j-after-lemma-1}
        \mathbb{E}_{\overline{z} \sim \mathbb{D}_j^m} \left[ \operatorname{er}_{\mathbb{D}_j}(\mathcal{A}_S(\overline{z})) \right] \ge \frac{1}{4}.
    \end{equation}
    We now extend this distribution $\mathbb{D}_j$ to a distribution $\widehat{\mathbb{D}}$ on the discrete $\sigma$-algebra $\mathcal{P}(\mathcal{Z})$ by setting
    \[
        \widehat{\mathbb{D}}(C) \coloneqq \mathbb{D}_j(C \cap \mathcal{Z}_S) \quad \text{for any } C \subseteq \mathcal{Z}.
    \]
    The distribution $\mathbb{D}$ is then defined as the restriction of $\widehat{\mathbb{D}}$ to the original $\sigma$-algebra $\Sigma_{\mathcal{Z}}$. Since $\mathbb{D}_j$ is a discrete uniform distribution, so is $\mathbb{D}$, and thus $\mathbb{D} \in \mathcal{D}$ by our initial assumption. This extension yields for any $h \in \mathcal{H}$ that
    \begin{align}
        \operatorname{er}_{\mathbb{D}}(h) &= \mathbb{D}(\mathcal{Z} \setminus \Gamma(h)) \nonumber \\
        &= \mathbb{D}_j((\mathcal{Z} \setminus \Gamma(h)) \cap \mathcal{Z}_S) \nonumber \\
        &= \mathbb{D}_j(\mathcal{Z}_S \setminus \Gamma(h|_S)) = \operatorname{er}_{\mathbb{D}_j}(h|_S). \label{eq:error-equivalence}
    \end{align}
    Thus, $\operatorname{opt}_{\mathbb{D}}(\mathcal{H}) = 0$, as $\operatorname{er}_{\mathbb{D}}(h_j) = \operatorname{er}_{\mathbb{D}_j}(f_j) = 0$ by~\ref{lem:no-free-lunch} and~\ref{eq:fixed-j-after-lemma-1}.

    The next lemma shows that the high expected error from Lemma~\ref{lem:no-free-lunch} carries over to the full hypothesis under the extended measure $\widehat{\mathbb{D}}$.

    \begin{lemma}
        \label{lem:full-error-bound-hat}
        We have
        \[
            \mathbb{E}_{\overline{z} \sim \widehat{\mathbb{D}}^m} \left[ \operatorname{er}_{\mathbb{D}}(\mathcal{A}(\overline{z})) \right] \ge \frac{1}{4}.
        \]
    \end{lemma}
    \begin{subproof}[Proof of Lemma~\ref{lem:full-error-bound-hat}]

        We show that the expectation under $\widehat{\mathbb{D}}^m$ is identical to the expectation under $\mathbb{D}_j^m$:
        \[
            \mathbb{E}_{\overline{z} \sim \widehat{\mathbb{D}}^m} \left[ \operatorname{er}_{\mathbb{D}}(\mathcal{A}(\overline{z})) \right] = \mathbb{E}_{\overline{z} \sim \mathbb{D}_j^m} \left[ \operatorname{er}_{\mathbb{D}_j}(\mathcal{A}_S(\overline{z})) \right].
        \]
        The result then follows directly from the inequality established in~\eqref{eq:fixed-j-after-lemma-1}.

        To prove the equality, we expand the left-hand side. The expectation is a sum over all samples $\overline{z} \in \mathcal{Z}^m$ with non-zero probability under $\widehat{\mathbb{D}}^m$. A sample $\overline{z}$ has non-zero probability only if all its components are in $\mathcal{Z}_S$. For such a sample, which corresponds to some instance sequence $\overline{x} \in S^m$ and the function $f_j$, we have $\widehat{\mathbb{D}}^m(\{\overline{z}_{\overline{x}}^j\}) = (1/2m)^m$.

        Using this and the error equivalence from~\eqref{eq:error-equivalence}, we obtain:
        \begin{align*}
            \mathbb{E}_{\overline{z} \sim \widehat{\mathbb{D}}^m} \left[ \operatorname{er}_{\mathbb{D}}(\mathcal{A}(\overline{z})) \right] &= \sum_{\overline{x} \in S^m} \widehat{\mathbb{D}}^m(\{\overline{z}_{\overline{x}}^j\}) \operatorname{er}_{\mathbb{D}}(\mathcal{A}(\overline{z}_{\overline{x}}^j)) \\
            &= \sum_{\overline{x} \in S^m} \frac{1}{(2m)^m} \operatorname{er}_{\mathbb{D}_j}(\mathcal{A}_S(\overline{z}_{\overline{x}}^j)) \\
            &= \mathbb{E}_{\overline{z} \sim \mathbb{D}_j^m} \left[ \operatorname{er}_{\mathbb{D}_j}(\mathcal{A}_S(\overline{z})) \right].
        \end{align*}
        This completes the proof.
    \end{subproof}
    The final lemma uses this expected error bound to show that the probability of failure is high.

    \begin{lemma}
        \label{lem:pac-violation}
        The distribution $\widehat{\mathbb{D}}$ satisfies the inequality
        \[
            \widehat{\mathbb{D}}^m\left(\left\{ \overline{z} \in \mathcal{Z}^m \mid \operatorname{er}_{\mathbb{D}}(\mathcal{A}(\overline{z})) - \operatorname{opt}_{\mathbb{D}}(\mathcal{H}) > \frac{1}{8} \right\}\right) \ge \frac{1}{7}.
        \]
    \end{lemma}
    \begin{subproof}[Proof of Lemma~\ref{lem:pac-violation}]

        We recall that $\operatorname{opt}_{\mathbb{D}}(\mathcal{H}) = 0$ and, by Lemma~\ref{lem:full-error-bound-hat}, we have that $\mathbb{E}_{\overline{z} \sim \widehat{\mathbb{D}}^m} [ \operatorname{er}_{\mathbb{D}}(\mathcal{A}(\overline{z})) ] \ge 1/4$.

        Let $X(\overline{z}) = \operatorname{er}_{\mathbb{D}}(\mathcal{A}(\overline{z}))$ be the random variable for the error. Since $X$ is bounded in $[0,1]$, we can apply the one-sided Chebyshev inequality~\cite[Appendix B.1]{UnderstandinMachineLearning}. For a random variable $X \in [0,1]$ and any $a \in (0,1)$, this inequality states that $\mathbb{P}(X > a) \ge \frac{\mathbb{E}[X] - a}{1-a}$.

        Applying this with $a=1/8$ yields:
        \begin{align*}
            \widehat{\mathbb{D}}^m\left(\left\{ \overline{z} \in \mathcal{Z}^m \mid \operatorname{er}_{\mathbb{D}}(\mathcal{A}(\overline{z})) > \frac{1}{8} \right\}\right) &\ge \frac{\mathbb{E}_{\overline{z}\sim\widehat{\mathbb{D}}^m}[\operatorname{er}_{\mathbb{D}}(\mathcal{A}(\overline{z}))] - 1/8}{1 - 1/8} \\
            &\ge \frac{1/4 - 1/8}{7/8} = \frac{1/8}{7/8} = \frac{1}{7}.
        \end{align*}
        Since $\operatorname{opt}_{\mathbb{D}}(\mathcal{H}) = 0$, this is precisely the inequality stated in the lemma.
    \end{subproof}

    The PAC definition requires that for our chosen $\varepsilon=1/8$ and $\delta=1/7$, there must exist a set $C' \in \Sigma_{\mathcal{Z}}^m$ such that $C' \subseteq \{ \overline{z} \mid \operatorname{er}_{\mathbb{D}}(\mathcal{A}(\overline{z})) \le 1/8 \}$ and $\mathbb{D}^m(C') \ge 1 - 1/7$.

    However, Lemma~\ref{lem:pac-violation} shows that the complement of the set containing $C'$ has a probability of at least $1/7$ under the measure $\widehat{\mathbb{D}}^m$. Since any such $C'$ must be in $\Sigma_{\mathcal{Z}}^m$, we have $\mathbb{D}^m(C') = \widehat{\mathbb{D}}^m(C')$. This would imply that $\widehat{\mathbb{D}}^m(C') \le 1 - 1/7$, which contradicts the PAC requirement that its probability be at least $1 - 1/7$. Therefore, no such set $C'$ can exist.

    We have shown that for any sample size $m$, there exists a distribution $\mathbb{D} \in \mathcal{D}$ for which the PAC condition fails. This completes the proof by contrapositive.
\end{proof}
\section{PAC Learnability of Definable Sets in O-minimal Structures}

In this section, we connect the model-theoretic properties developed in Chapter~\ref{ch:nip-property} with the learning theory framework established here. Our primary goal is to apply the Fundamental Theorem of Statistical Learning to hypothesis spaces that are definable over first-order structures, with a particular focus on o-minimal structures. These structures are naturally equipped with an order topology, allowing the measure-theoretic conditions of our learning framework to be satisfied by considering the associated Borel $\sigma$-algebra.

We will show that for such structures, the non-independence property (NIP) provides a powerful tool to bound the VC-dimension of definable hypothesis spaces, which in turn guarantees PAC learnability. This will serve as the capstone result, demonstrating a deep and fruitful interaction between model theory and statistical learning. For a comprehensive introduction to the model-theoretic concepts used, we refer the reader to~\cite{Marker2002}.

We begin by formally defining what we mean by a definable hypothesis space.

\begin{definition}[Definable Hypothesis Space]
    \label{def:definable-hypothesis-space}
    Let $n, \ell \in \mathbb{N}$, let $\mathcal{L}$ be a language and let $\mathcal{M}$ be an $\mathcal{L}$-structure. Moreover, let $\mathcal{X} \subseteq \mathcal{M}^n$ be a non-empty set definable over $\mathcal{M}$, and let $\varphi(x_1, \dots, x_n; p_1, \dots, p_\ell)$ be an $\mathcal{L}$-formula. Then the hypothesis space $\mathcal{H}^\varphi \subseteq \{0,1\}^{\mathcal{M}^n}$ is given by
    \[
        \mathcal{H}^\varphi \coloneqq \{ \mathds{1}_{\varphi(\mathcal{M}; \underline{w})} \mid \underline{w} \in \mathcal{M}^\ell \},
    \]
    and the hypothesis space $\mathcal{H}_\mathcal{X}^\varphi \subseteq \{0,1\}^\mathcal{X}$ is given by
    \[
        \mathcal{H}_\mathcal{X}^\varphi \coloneqq \mathcal{H}^\varphi|_\mathcal{X} = \{ h|_\mathcal{X} \mid h \in \mathcal{H}^\varphi \}.
    \]
\end{definition}

\subsection{Definable Sets, NIP, and VC-Dimension}

The definition of a definable hypothesis space is powerful because it encompasses many models used in practice. A key example is neural networks.

\begin{example}[Neural Networks as Definable Hypothesis Spaces]
    Many common machine learning models, including artificial neural networks with fixed architectures, can be described by formulas in specific mathematical structures. For instance, consider the structure of the real exponential field, $\mathbb{R}_{\text{exp}} = (\mathbb{R}, +, \cdot, \exp, <)$. Common activation functions like the sigmoid function, $\sigma(x) = 1/(1+e^{-x})$, and the ReLU function, $\text{ReLU}(x) = \max\{0,x\}$, are definable in this structure. Consequently, a neural network with a fixed architecture can be represented by a single formula $\varphi(\underline{x}; \underline{p})$, where the inputs $\underline{x}$ are the features and the parameters $\underline{p}$ represent all the weights and biases of the network. The set of all functions computable by this network is precisely the definable hypothesis space $\mathcal{H}^\varphi$.
\end{example}

The connection between this model-theoretic setup and learnability is established by linking the Non-Independence Property (NIP) to the VC-dimension. In Chapter~\ref{ch:nip-property}, we established this link under the name of VC-Dependence Duality (Proposition~\ref{prop:vc-dependence-duality}). Recall that a formula $\varphi(\underline{x}; \underline{p})$ defines a family of sets $\{\varphi(\mathcal{M}; \underline{w}) \mid \underline{w} \in \mathcal{M}^\ell\}$. We say this family is dependent if it does not contain independent sequences of all finite lengths. This duality states that this family of sets is dependent if and only if the corresponding hypothesis space $\mathcal{H}^\varphi$ is a VC-class (i.e., has finite VC-dimension). This provides the crucial bridge between the two fields.

A central result from Chapter~\ref{ch:nip-property} is that any o-minimal structure has the NIP property. Specifically, we proved that every definable relation in an o-minimal structure is dependent (Corollary~\ref{cor:o-minimal-is-nip}). Combining these facts gives the following powerful result.

\begin{theorem}
    \label{thm:o-minimal-finite-vc}
    Let $\mathcal{M}$ be an o-minimal structure. Then any hypothesis space $\mathcal{H}_\mathcal{X}^\varphi$ definable over $\mathcal{M}$ has a finite VC-dimension.
\end{theorem}

\subsubsection{Bridging Model Theory and Statistical Learning}

We can now see the complete picture. The Fundamental Theorem of Statistical Learning (Theorem~\ref{thm:fundamental-theorem}) provides a direct link between a statistical property (PAC learnability) and a combinatorial one (finite VC-dimension). On the other hand, the results from model theory, culminating in Theorem~\ref{thm:o-minimal-finite-vc}, provide a link between a logical property (definability in an o-minimal structure) and the same combinatorial property (finite VC-dimension).

By chaining these results together, we can conclude that hypothesis spaces that are definable in o-minimal structures are PAC learnable, provided the necessary measure-theoretic conditions of the learning framework are met. This provides a remarkably high-level method for guaranteeing the learnability of complex models, such as neural networks, by analyzing the logical structure in which they are defined. In the remainder of this section, we will make this connection rigorous by carefully verifying the measurability requirements for this setting.

\subsection{The Main Result}

We are now ready to state the main result of this chapter, which formally connects PAC learnability to definability in o-minimal structures over the reals. To do so, we must first specify the measure-theoretic setting.

For any $k \in \mathbb{N}$, the space $\mathbb{R}^k$ is equipped with its standard topology. The \emph{Borel $\sigma$-algebra} on $\mathbb{R}^k$, denoted $\mathcal{B}(\mathbb{R}^k)$, is the smallest $\sigma$-algebra containing all open sets. When our instance space $\mathcal{X}$ is a subset of $\mathbb{R}^n$, we equip it with the \emph{trace $\sigma$-algebra}:
\[
    \mathcal{B}(\mathcal{X}) \coloneqq \mathcal{B}(\mathbb{R}^n) \cap \mathcal{X} = \{ B \cap \mathcal{X} \mid B \in \mathcal{B}(\mathbb{R}^n) \}.
\]
A crucial link between the logical property of definability and the measure-theoretic property of being a Borel set is provided by the Cell Decomposition Theorem, a cornerstone of o-minimality mentioned in Chapter~\ref{ch:nip-property}.

\begin{lemma}
    \label{lem:definable-is-borel}
    Let $\mathcal{R}$ be an o-minimal expansion of the real ordered field $\mathbb{R}_{\text{or}}$. If a set $A \subseteq \mathbb{R}^n$ is definable over $\mathcal{R}$, then $A$ is a Borel set, i.e., $A \in \mathcal{B}(\mathbb{R}^n)$.
\end{lemma}
\begin{proof}

    By the Cell Decomposition Theorem, any definable set $A$ has a finite partition into cells. Each cell, being homeomorphic to an open hypercube, is a Borel set. Since the collection of Borel sets is closed under finite unions, the set $A$ must also be a Borel set.
\end{proof}

The theorem also requires the technical condition of completeness for the probability space. A probability space $(\Omega, \Sigma, \mathbb{P})$ is called \emph{complete} if its $\sigma$-algebra $\Sigma$ contains all subsets of any set $N \in \Sigma$ that has measure zero ($\mathbb{P}(N)=0$). This ensures that the measure theory is robust to null sets.

With this, we can state the main theorem. We recall that $\mathbb{R}_{\text{or}} = (\mathbb{R}, +, \cdot, <)$ denotes the real ordered field.

\begin{theorem}
    \label{thm:main-pac-o-minimal}
    Let $\mathcal{R}$ be an o-minimal expansion of the real ordered field $\mathbb{R}_{\text{or}}$. Let $\mathcal{X} \subseteq \mathbb{R}^n$ be a non-empty definable set, and let $\varphi(\underline{x}; \underline{p})$ be an $\mathcal{L}(\mathcal{R})$-formula defining the hypothesis space $\mathcal{H}_\mathcal{X}^\varphi$.

    Let the learning framework be $(\mathcal{X}, \Sigma_{\mathcal{Z}}, \mathcal{D}, \mathcal{H}_\mathcal{X}^\varphi)$, where the sample space is $\mathcal{Z} = \mathcal{X} \times \{0,1\}$. Assume that the sigma-algebra $\Sigma_{\mathcal{Z}}$ contains the Borel $\sigma$-algebra on $\mathcal{Z}$ ($\mathcal{B}(\mathcal{Z}) \subseteq \Sigma_{\mathcal{Z}}$), and that for any $\mathbb{D} \in \mathcal{D}$ and any $m \in \mathbb{N}$, the product space $(\mathcal{Z}^m, \Sigma_{\mathcal{Z}}^m, \mathbb{D}^m)$ is a complete probability space.

    Then the hypothesis space $\mathcal{H}_\mathcal{X}^\varphi$ is PAC learnable with respect to $\mathcal{D}$.
\end{theorem}

\begin{proof}

    Since $\mathcal{R}$ is an o-minimal structure, it has the NIP property, which means every definable relation is dependent (Corollary~\ref{cor:o-minimal-is-nip}). By the VC-Dependence Duality (Proposition~\ref{prop:vc-dependence-duality}), this is equivalent to the statement that any definable hypothesis space has a finite VC-dimension. Therefore, we have $\operatorname{vc}(\mathcal{H}_\mathcal{X}^\varphi) < \infty$.

    Our aim is to apply the Fundamental Theorem of Statistical Learning (Theorem~\ref{thm:fundamental-theorem}) to conclude that $\mathcal{H}_\mathcal{X}^\varphi$ is PAC learnable. To do so, we must verify that the hypothesis space is well-behaved with respect to $\mathcal{D}$.

    First, we show that for any hypothesis $h_{\underline{w}} \in \mathcal{H}_\mathcal{X}^\varphi$ (corresponding to parameters $\underline{w} \in \mathbb{R}^\ell$), its graph $\Gamma(h_{\underline{w}})$ is a measurable set in $\Sigma_{\mathcal{Z}}$. The graph is the set of points $(\underline{x}, y) \in \mathcal{Z}$ satisfying the formula:
    \[
        (\varphi(\underline{x}; \underline{w}) \land y=1) \lor (\neg\varphi(\underline{x}; \underline{w}) \land y=0).
    \]
    This is a first-order formula in the language of $\mathcal{R}$, so the graph $\Gamma(h_{\underline{w}})$ is a definable set. By Lemma~\ref{lem:definable-is-borel}, every definable set is a Borel set. Since we assumed $\mathcal{B}(\mathcal{Z}) \subseteq \Sigma_{\mathcal{Z}}$, it follows that $\Gamma(h_{\underline{w}}) \in \Sigma_{\mathcal{Z}}$.

    Second, we must verify that the maps $U$ and $V$ from Definition~\ref{def:well-behaved} are $\Sigma_{\mathcal{Z}}^m$-measurable and $\Sigma_{\mathcal{Z}}^{2m}$-measurable, respectively. To this end, we first show that the map
    \[
        f_U: \mathcal{Z}^m \times \mathbb{R}^\ell \to [0,1], \quad (\overline{z}, \underline{w}) \mapsto |\operatorname{er}_{\mathbb{D}}(h_{\underline{w}}) - \hat{\operatorname{er}}_{\overline{z}}(h_{\underline{w}})|
    \]
    is $(\Sigma_{\mathcal{Z}}^m \otimes \mathcal{B}(\mathbb{R}^\ell))$-measurable, and that the map
    \[
        f_V: \mathcal{Z}^{2m} \times \mathbb{R}^\ell \to \{k/m \mid k \in \{0, \dots, m\}\}, \quad (\overline{z}, \overline{z}', \underline{w}) \mapsto |\hat{\operatorname{er}}_{\overline{z}'}(h_{\underline{w}}) - \hat{\operatorname{er}}_{\overline{z}}(h_{\underline{w}})|
    \]
    is $(\Sigma_{\mathcal{Z}}^{2m} \otimes \mathcal{B}(\mathbb{R}^\ell))$-measurable. After establishing this joint measurability, we will use the completeness of the underlying probability spaces which allows us to deduce the measurability of a function defined as a supremum over an uncountable parameter space, since this is precisely how the maps $U$ and $V$ are constructed.

    We begin by establishing the measurability of the two main components of the functions $f_U$ and $f_V$: the true error and the sample error, viewed as functions of the parameters $\underline{w}$.

    \begin{enumerate}
        \item \textbf{Measurability of the true error map.}
        First, consider the set of all pairs $(\underline{z}, \underline{w})$ for which the hypothesis parametrised by $\underline{w}$ misclassifies the point $\underline{z}$:
        \[
            M \coloneqq \{(\underline{z}, \underline{w}) \in \mathcal{Z} \times \mathbb{R}^\ell \mid \underline{z} \notin \Gamma(h_{\underline{w}}) \}.
        \]
        Since $\mathcal{Z}$ and $\Gamma(h_{\underline{w}})$ are definable over $\mathcal{R}$ for any $\underline{w}$, the set $M$ is also definable over $\mathcal{R}$. By Lemma~\ref{lem:definable-is-borel}, it follows that $M$ is a Borel set, i.e., $M \in \mathcal{B}(\mathcal{Z} \times \mathbb{R}^\ell)$. As the Borel $\sigma$-algebra on a product space is the product of the Borel $\sigma$-algebras, we have $M \in \mathcal{B}(\mathcal{Z}) \otimes \mathcal{B}(\mathbb{R}^\ell)$.

        By a standard result from measure theory (a consequence of Fubini's theorem, see e.g.,~\cite[Proposition 3.3.2]{Bogachev2007}), the function that maps a parameter $\underline{w}$ to the measure of the corresponding slice of $M$ is measurable. Therefore, the map
        \[
            \underline{w} \mapsto \mathbb{D}(\{\underline{z} \mid (\underline{z}, \underline{w}) \in M\}) = \operatorname{er}_{\mathbb{D}}(h_{\underline{w}})
        \]
        is $\mathcal{B}(\mathbb{R}^\ell)$-measurable.

        \item \textbf{Joint measurability of the sample error map.}
        Next, we consider the map $(\overline{z}, \underline{w}) \mapsto \hat{\operatorname{er}}_{\overline{z}}(h_{\underline{w}})$. To show it is measurable, we verify that the preimages of generating sets for the Borel $\sigma$-algebra on its codomain are measurable. For any $k \in \{1, \dots, m\}$, consider the set
        \[
            P_k \coloneqq \{(\overline{z}, \underline{w}) \in \mathcal{Z}^m \times \mathbb{R}^\ell \mid \hat{\operatorname{er}}_{\overline{z}}(h_{\underline{w}}) \ge k/m \}.
        \]
        This condition is equivalent to stating that at least $k$ of the sample points in $\overline{z} = (\underline{z}_1, \dots, \underline{z}_m)$ are misclassified by $h_{\underline{w}}$. This can be expressed by the following $\mathcal{L}(\mathcal{R})$-formula:
        \[
            \psi_k(\underline{z}_1, \dots, \underline{z}_m; \underline{w}) \coloneqq \bigvee_{I \subseteq \{1,\dots,m\}, |I| \ge k} \left( \bigwedge_{i \in I} (\underline{z}_i, \underline{w}) \in M \right).
        \]
        Since $M$ is a definable set, $\psi_k$ is a definable formula. Therefore, the set $P_k$ is definable over $\mathcal{R}$. By Lemma~\ref{lem:definable-is-borel}, $P_k$ is a Borel set, $P_k \in \mathcal{B}(\mathcal{Z}^m \times \mathbb{R}^\ell)$. Since the preimages of the generating sets are measurable, the map $(\overline{z}, \underline{w}) \mapsto \hat{\operatorname{er}}_{\overline{z}}(h_{\underline{w}})$ is $(\mathcal{B}(\mathcal{Z}^m) \otimes \mathcal{B}(\mathbb{R}^\ell))$-measurable. As we assume $\mathcal{B}(\mathcal{Z}^m) \subseteq \Sigma_{\mathcal{Z}}^m$, the map is also $(\Sigma_{\mathcal{Z}}^m \otimes \mathcal{B}(\mathbb{R}^\ell))$-measurable.
    \end{enumerate}

    The measurability of the map $\underline{w} \mapsto \operatorname{er}_{\mathbb{D}}(h_{\underline{w}})$ and the joint measurability of the map $(\overline{z}, \underline{w}) \mapsto \hat{\operatorname{er}}_{\overline{z}}(h_{\underline{w}})$ imply that the functions $f_U$ and $f_V$ are measurable with respect to their respective product $\sigma$-algebras. This is because they are formed by taking differences and absolute values of measurable functions, which are operations that preserve measurability.

    Since we assume that the probability space $(\mathcal{Z}^k, \Sigma_{\mathcal{Z}}^k, \mathbb{D}^k)$ is complete for any $k \in \mathbb{N}$, we can apply a standard result from empirical process theory (see, e.g.,~\cite[p. 197]{PollardConvergenceOfStochasticProcesses}) to deduce that the supremum over the parameter space of a jointly measurable function is itself measurable. This allows us to conclude that the map
    \[
        U: \mathcal{Z}^m \to [0,1], \quad \overline{z} \mapsto \sup_{\underline{w} \in \mathbb{R}^\ell} f_U(\overline{z}, \underline{w})
    \]
    is $\Sigma_{\mathcal{Z}}^m$-measurable, and that the map
    \[
        V: \mathcal{Z}^{2m} \to \{k/m \mid k \in \{0, \dots, m\}\}, \quad (\overline{z}, \overline{z}') \mapsto \sup_{\underline{w} \in \mathbb{R}^\ell} f_V(\overline{z}, \overline{z}', \underline{w})
    \]
    is $\Sigma_{\mathcal{Z}}^{2m}$-measurable.

    Hence, the hypothesis space $\mathcal{H}_\mathcal{X}^\varphi$ is well-behaved with respect to $\mathcal{D}$. Having established both that $\operatorname{vc}(\mathcal{H}_\mathcal{X}^\varphi) < \infty$ and that the space is well-behaved, applying the Fundamental Theorem of Statistical Learning (Theorem~\ref{thm:fundamental-theorem}) yields that $\mathcal{H}_\mathcal{X}^\varphi$ is PAC learnable with respect to $\mathcal{D}$.
\end{proof}